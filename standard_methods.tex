\section{csvimport}

This method is used to import text files in tabular format.  It reads
plain text files as well as \texttt{gzip} compressed files.

\subsection{Arguments}

\begin{tabular}{ll}
  \texttt{filename} & \\ %                  : OptionString,
  \texttt{separator} & \\ %                 : ',',
  \texttt{labelsonfirstline} & \\ %         : True,
  \texttt{labels} & \\ %                    : [], # Mandatory if not labelsonfirstline, always sets labels if set.
  \texttt{hashlabel} & \\ %                 : None,
  \texttt{quot} & \\ %_support'             : False, # 'foo',''bar'' style CSV
  \texttt{rename} & \\ %                    : {},    # Labels to replace (if they are in the file) (happens first)
  \texttt{discard} & \\ %                   : set(), # Labels to not include (if they are in the file)
  \texttt{allow\_bad} & \\ %               : False, # Still succeed if some lines have too few/many fields.
\end{tabular}

\begin{tabular}{ll}
  \texttt{previous}
\end{tabular}


\clearpage
\section{dataset\_type}

A dataset imported with \texttt{csvimport} may be typed using
\texttt{dataset\_type}.  This method parses the input data...



\subsection{Arguments}

Datasets\\

\begin{tabular}{ll}
  \texttt{source} & \\
  \texttt{previous} & \\
\end{tabular}\\

\noindent Options\\

\begin{tabular}{ll}
  \texttt{column2type}               & \\%{'COLNAME': TYPENAME},
  \texttt{defaults}                  & \\%{}, # {'COLNAME': value}, unspecified -> method fails on unconvertible unless filter\_bad
  \texttt{rename}                    & \\%{}, # {'OLDNAME': 'NEWNAME'} doesn't shadow OLDNAME.
  \texttt{caption}                   & \\%'typed dataset',
  \texttt{discard\_untyped}           & \\%bool, # Make unconverted columns inaccessible (``new'' dataset)
  \texttt{filter\_bad}                & \\%False, # Implies discard_untyped
  \texttt{numeric\_comma}             & \\%False, # floats as ``3,14''
\end{tabular}


\subsection{Example Invocation}
An example invocation is the following

\begin{python}
urd.build('dataset_type', ...,
  options=dict(
    column2type=dict(
      auct_start_dt='datetime:%Y-%m-%d',
      brand='json',
      item_id='number',
      comp='unicode:utf-8',
    ),
  )
\end{python}



\subsection{Typing}
This section describes all typing options in detail.

\subsubsection{Numbers}
The \emph{number} type is int or float.\\

\begin{tabular}{ll}
  \texttt{number}     & \texttt{int} or \texttt{float}\\
  \texttt{number:int} & \texttt{int}, converts floats to int.\\
\end{tabular}



\subsubsection{Float Point Numbers}
Floating point numbers may be stored as 32 or 64 bits.  In addition,
there are six parsing options that are useful in different scenarios.
The \emph{ignore} option ignores any trailing characters after the
number.  Then there are \emph{exact} that causes error if the number
does not fit, and \emph{saturate} that silently saturates a
non-fitting number.  These can also be used in combination, see table
below for all alternatives\\

\begin{tabular}{lll}
\texttt{float32} & \texttt{float64} & \emph{default}\\
\texttt{float32i} & \texttt{float64i} & \emph{ignore}, will discard trailing garbage\\
\texttt{float32e} & \texttt{float64e} & \emph{exact}, error if parsed number does not fit in type \\
\texttt{float32s} & \texttt{float64s} & \emph{saturate}, saturate to min/max if number does not fit in type \\
\texttt{float32ei} & \texttt{float64ei} & \emph{exact} + \emph{ignore} \\
\texttt{float32si} & \texttt{float64si} & \emph{saturate} + \emph{ignore} \\
\end{tabular}

\subsubsection{Integers}
Integers are stored as either 32 or 64 bits.  Parsing takes base into
account, so in addition to decimal numbers, it is also straightforward
to parse octal and hexadecimal numbers.  The \emph{ignore} option
causes parsing to ignore trailing garbage characters.\\

\begin{tabular}{lll}
  \texttt{int32\_0}   & \texttt{int64\_0}   & \emph{auto}, avoid and use a deteministic type if possible \\
  \texttt{int32\_0i}  & \texttt{int64\_0i}  & \emph{auto}, ignore trailing garbage \\
  \texttt{int32\_8}   & \texttt{int64\_8}   & \emph{octal} \\
  \texttt{int32\_8i}  & \texttt{int64\_8i}  & \emph{octal}, ignore trailing garbage \\
  \texttt{int32\_10}  & \texttt{int64\_10}  & \emph{decimal} \\
  \texttt{int32\_10i} & \texttt{int64\_10i} & \emph{decimal}, ignore trailing garbage \\
  \texttt{int32\_16}  & \texttt{int64\_16}  & \emph{hexadecimal} \\
  \texttt{int32\_16i} & \texttt{int64\_16i} & \emph{hexadecimal}, ignore trailing garbage \\
\end{tabular}



\subsection{Integers Stored as Floats}

There are also a parsing options for integers that are represented in
a floating point format in the source data.  This is useful if integer
data is stored with decimals, such as \texttt{5.0}.  In pseudocode,
the parsing basically runs \texttt{int(float(value))} for each such
value.\\

\begin{tabular}{lll}
  \texttt{floatint32e} & \texttt{floatint64e}  & \emph{exact}, error if parsed number does not fit in type\\
  \texttt{floatint32s} & \texttt{floatint64s}  & \emph{saturate}, saturate to min/max if number does not fit in type\\
  \texttt{floatint32ei}& \texttt{floatint64ei} & \emph{exact} + \emph{ignore}\\
  \texttt{floatint32si}& \texttt{floatint64si} & \emph{saturate} + \emph{ignore}\\
\end{tabular}



\subsection{Convert to Boolean}
It is common that a column holds values that are to be interpreted as
either \texttt{False} or \texttt{True}.  There are two types that is
useful in this context.  The first one is \texttt{strbool}, that works like this\\

\begin{tabular}{ll}
  \texttt{strbool} & \texttt{False} if value in (\texttt{'false', '0', 'f', 'no', 'off', 'nil', 'null', ''})\\
                   & \texttt{True} otherwise
\end{tabular}
\\
The other is \texttt{floatbool} that is True when the float has bits
set to one and zero otherwise.  There is also a \texttt{floatbooli}
that ignores trailing garbage characters.\\

\begin{tabular}{ll}
  \texttt{floatbool}  & True if float has bits set to one, zero otherwise\\
  \texttt{floatbooli} & same + \emph{ignore}\\
\end{tabular}



\subsection{Time and Date}
There are three types relating to time available, \texttt{date},
\texttt{time}, and \texttt{datetime}.  Each of these has a
corresponding version that ignores trailing garbage characters.
All time types require a format specification as described below\\

\begin{tabular}{ll}
  \texttt{date:*}      & a date with format specifier\\
  \texttt{datei:*}     & same + \emph{ignore}\\
  \texttt{time:*}      & a time with format specifier\\
  \texttt{timei:*}     & same + \emph{ignore}\\
  \texttt{datetime:*}  & a date + time with format specifier\\
  \texttt{datetimei:*} & same + \emph{ignore}\\
\end{tabular}\\

The format is standard Python time formats, see for example\\
\\
\begin{python}
   # will match for example '2017-03-22'
   auct_start_dt='date:%Y-%m-%d'  
   # will match for example '183000', i.e. half past six in the evening
   tod='time:%H%M%S'
   # will match for example '2017-03-22 18:30:15'
   timestamp='datetime:'%Y-%m-%d %H:%M:%S'
\end{python}


\subsection{Strings and Byte Sequences}
There are a number of ways to read string and byte data, depending on
how the raw input data is to be interpreted.  Here are the basic
types, and variations and options will be described below.\\

\begin{tabular}{ll}
  \texttt{bytes}      & list of bytes\\
  \texttt{bytesstrip} & list of bytes, strip characters 8-13,32 from start and end\\
  \texttt{ascii}      & list of ascii characters\\
  \texttt{asciistrip} & list of ascii characters, strip characters 8-13,32 from start and end\\
  \texttt{unicode:*}    & list of unicode characters\\
\end{tabular}\\

\noindent The following also takes argument\\

\begin{tabular}{ll}
  \texttt{unicodestrip:*}  & list of unicode characters, strip characters 8-13,32 from start and end\\
\end{tabular}\\

\noindent where the argument is one of \texttt{replace}, \texttt{ignore}, and \texttt{strict}.
\texttt{strict} will cause error, \texttt{ignore} will erase illegal characters.\\

\begin{tabular}{ll}
  \texttt{ascii:*}      & list of ascii characters\\
  \texttt{asciistrip:*} & list of ascii characters, strip characters 8-13,32 from start and end\\
\end{tabular}\\

\noindent where the argument is one of \texttt{replace}, \texttt{encode}, and \texttt{strict}.
\texttt{strict} will cause error, \texttt{ignore} will erase illegal characters,
and \texttt{encode} is reversible encoding.


\clearpage
\section{dataset\_csvexport}

\clearpage
\section{dataset\_rehash}

Dataset rehash will create a new dataset based on its \texttt{source}
dataset.  The new dataset will be hashed on the column specified by
the \texttt{hashlabel} input option.

\subsection{Input Datasets and Options}

Datasets\\

\begin{tabular}{ll}
  \texttt{source}   & source dataset to rehash\\
  \texttt{previous} & previous dataset that will be chained to\\
\end{tabular}\\


\noindent Options\\

\begin{tabular}{ll}
  \texttt{hashlabel} & column for hashing, required.\\
  \texttt{length}    & Go back at most this many datasets. Default is -1, which goes back to previous.source\\
  \texttt{caption}   & Optional caption.  A reasonable caption is created automatically if left blank\\
  \texttt{as\_chain}  & True generates one dataset per slice, False generates one dataset.  Default \texttt{False}.\\
\end{tabular}


\subsection{Example Invocation}
\begin{python}
urd.build('dataset_rehash',
  datasets=dict(
    source=jid,
  ), 
  options=dict(
    hashlabel = 'start_date',
  )
)
\end{python}



\subsection{Hashing Details}
Hashing on a column means that each row will be copied to a slice that
is determined from the value of the \texttt{hashlabel} column.  In
particular, the value of the row's \texttt{hashlabel} column will be
fed through a hashing function, and the output modulo the number of
slices will be the target slice number.  In meta-language, it is
something like this
\\
\begin{python}
from gzutil import siphash
target_sliceno = siphash(cols[hashlabel])
\end{python}


\subsection{Note on Chains}
\subsubsection{1.}
Note that default operation is to rehash a complete chain of datasets
from \texttt{source} back to \texttt{previous.source}.  This is
controlled by the \texttt{length} option.

\subsubsection{2.}
Internally, \texttt{dataset\_rehash} always generates one dataset per
slice in a chain.  This is also what is returned if \texttt{as\_chain
  == True}.  Otherwise, all datasets are concatenated into one.  Thus,
there is a choice of either having the output as a chain of datasets
or as a single dataset.  The chain will execute faster, since the
concatenation step is omitted.


\clearpage
\subsection{Internal Operation}
Figure~\ref{fig:dataset_rehash} shows how rehashing happens...

Each \texttt{analysis} reads one complete slice of the source dataset
and hashes it into a new dataset.  This means that there is one new
dataset created per slice.  Each of these datasets are chained and
named using an integer counter starting at zero.  The last slice,
however, is named \texttt{default}.

If the \texttt{as\_chain} is set, \texttt{dataset\_rehash} is finished.
Otherwise, in \texttt{synthesis}, each of the datasets are read in
turn and concatenated into a single dataset.  When this is finished,
all datasets except the concatenated one is removed from disk.


\begin{figure}[h!]
  \begin{center}
    \input{dataset_rehash.pdf_t}
  \end{center}
  \caption{\texttt{dataset\_rehash} data flow}
  \label{fig:dataset_rehash}
\end{figure}


\clearpage
\section{dataset\_filter\_columns}
The method \texttt{dataset\_filter\_columns} is used to remove columns
from a dataset.  This method is typically run before methods that
operate on all columns of a dataset when only a subset of the columns
are required.  A typical example is \texttt{dataset\_rehash} that
operates on all columns of a dataset.  If a subset such as only one of
the columns is to be rehashed, they can be cut out using
\texttt{dataset\_filter\_columns}.

Note that no data is copied using this method.  It only updates soft
links.  So execution is typically significantly below a second.





\clearpage
\section{dataset\_sort}
The method \texttt{dataset\_sort} is used to sort relatively large
datasets.  One or more columns may be selected for sorting, and it
will sort one column at a time.

The method works by reading the columns to sort by, and create an
indexing column that stipulates the sorting order.  Each column is
then read in turn and sorted according to the sorting column.

The method has limited sorting capability.  Internally, it sorts one
column at a time, and it requires to hold that complete column plus an
indexing column in memory simultaneously.


\clearpage
\section{dataset\_datesplit, dataset\_datesplit\_discarded}

\clearpage
\section{dataset\_checksum, dataset\_checksum\_chain}


