\label{chap:urd_basic}

Build scripts are used to execute jobs and control the jobflow on the
Accelerator.  This chapter describes the basics of job building.  More
advanced features, using the Urd server, are presented in
chapter~\ref{chap:urd}.



\section{Build Scripts}
\index{build scripts}
Build scripts are stored in method package directories and have names
that start with the string \texttt{build\_}.  They are executed using
the \texttt{run} command.  For example, this command
\begin{shell}
ax run example
\end{shell}
will look for a file named \texttt{build\_example.py} and execute it.
(The \texttt{run} command is described in
section~\ref{sec:exec_runner}.)

The \texttt{run} command will load the build script and execute
its \texttt{main()} function.  The function takes a \textsl{mandatory}
argument named \texttt{urd}, so a basic build script looks like this
\index{build script!\texttt{main} function}
\begin{python}
def main(urd):
    ...
\end{python}
The \texttt{run} command inserts an object of the \texttt{Urd} class
as the argument to the \texttt{main} function.  This \texttt{urd}
object has a number of member functions and attributes useful for job
building and tracking.  For tracking purposes, it remembers all jobs
that are built, together with their input parameters and some other
meta information.  The \texttt{Urd class} is described on
page~\ref{}.



\subsection{Building a Job: \texttt{urd.build()}}
The \texttt{.build()} function is used to build a job from a method
(i.e.\ source file).  For example, the most simple build script that
executes method \texttt{method1} is
\begin{python}
def main(urd):
    urd.build('method1')
\end{python}
The full syntax for the \texttt{build} function is as follows
\index{\texttt{urd.build}}
\begin{python}
job = urd.build(method,
                options={}, datasets={}, jobs={},
                name='', caption='', workdir=None)
\end{python}
All parameters, except the name of the method, are optional, and
the \texttt{options}, \texttt{datasets}, and \texttt{jobs}
parameters must correspond to what is defined in the method to be
executed.

\textsl{Note}, however, that  if there are no collisions, any \options, \datasets, or \jobs
could be specified immediately after the \texttt{method} name as \textsl{plain
keyword arguments}.  Examples of this will follow.

When the job is completed, Urd will record it using the name of the
method as key, unless the \texttt{name=} is specified.
The \texttt{name} parameter is particularly useful to tell jobs apart
that are based on the same method.  A common case would be
the \texttt{csvimport} method, for example.  It is also possible to
assign a caption to a job, but this has no functional benefits.

When the job has been successfully built, the \texttt{build} function
will return a reference of type \texttt{Job}.  The \texttt{Job} class
contains member functions and attributes that can be used to extract
information, such as generated files or text written
to \texttt{stdout}, from the job.  The \texttt{Job} class is described
in detail on page~\ref{}.

Similarly, if the job to be built already exists in a
configured \textsl{workdir}, the \texttt{build} function will
immediately return the reference \texttt{Job} object without executing
anything.



\subsection{Connecting Jobs}
Jobs are connected by feeding the output job reference from
the \texttt{.build()} function as input parameter to a
new \texttt{.build()}.  For example
\begin{python}
def main(urd):
    job_import  = urd.build('csvimport', filename='inputfile.txt')
    job_process = urd.build('process',   source=job_import)
\end{python}
In the example above, the first job, \texttt{csvimport}, imports the
file ``\texttt{inputfile.txt}''.  The second job, \texttt{process},
takes imported dataset as input for further processing.



\subsection{Replaying Build Scripts}

When the example build script from the previous section is run, both
the \texttt{csvimport} and the \texttt{process} jobs will be built.
But what happens if the same build script is run a second time?
Remember now that the Accelerator stores all jobs in its associated
workdir.  If there has been no change since the last run, the
Accelerator will immediately find the job reference to
the \texttt{csvimport} without executing it.  This reference will be
input to the \texttt{.build()} call of the second method, and since
the Accelerator has seen this call before too, it will immediately
look up the reference to this job instead of executing it.  A second
run of the build script will only take a fraction of a second to
execute, but it will still return all job references.

On the other hand, if something has been modified, such as a method's
source code or any of the input parameters, the affected job(s) will
be re-executed.  For example, assume that the input to
the \texttt{csvimport} job is modified.  This will cause this method
to be executed again, leading to a new job with a new jobid.  This job
reference is input to the \texttt{process} job, causing it too to be
re-executed too.  Only those jobs affected by the modification will be
re-executed!

\textsl{A successful ``replay'' of a build script ensures the integrity and
dependencies of all involved calculations.}  If there are no changes,
the same result remains.  If, however, some of the code has been
modified, the Accelerator will compute new jobs to reflect the new
situation.  The result may be different, and the user is notified.


\section{Working with Build History:  \texttt{urd.joblist}}
\label{sec:joblist}

Information about previously executed jobs is stored in
the \texttt{urd.joblist} variable.  This variable is of
type \texttt{JobList}, which is basically a standard ordered
Python \texttt{list} with some additional features for searching,
profiling and pretty-printing.  The \texttt{JobList} class is explained on page~\ref{pageellersectionreferenceva}



\subsection{Printing a \texttt{JobList}:  \texttt{urd.joblist.pretty}}
Create a \texttt{JobList} and pretty-print it
\begin{python}
def main(urd):
    job1 = urd.build('first')
    job2 = urd.build('second', first=job1)
    print(urd.joblist.pretty)
\end{python}
which results in
\begin{shell}
JobList(
   [  0]  first : TEST-38
   [  1] second : TEST-39
)
\end{shell}
(The actual jobids will most likely be different.)  The name in
the \texttt{joblist} is either the name of the method, or, if present,
the name given explicitly using the \texttt{urd.build(name=)} option.



\subsection{Finding Jobs in a Joblist}
There are several ways to extract jobs or list of jobs from
a \texttt{joblist}.


\subsubsection*{Using \texttt{find} to Extract Jobs to a New Joblist}
The \texttt{find()} function finds matching jobs and returns them in a
new \texttt{JobList}.  For example,
\begin{python}
jl = urd.joblist.find('csvimport')
\end{python}
will create a \texttt{JobList} of all \texttt{csvimport} jobs
in \texttt{urd.joblist}.


\subsubsection*{Using \texttt{get} to Potentially Extract a Single Job}
The \texttt{get} function will return a job reference to the most
recent matching job.  For example,
\begin{python}
job = urd.joblist.get('csvimport')
\end{python}
If no matching job is found, \texttt{get} will return \pyNone.


\subsubsection*{Using Square Brackets to Extract a Singe Job}
Accessing jobs directly with a key like this
\begin{python}
job = urd.joblist['csvimport']
\end{python}
is similar to \texttt{get}, but will return an error if a matching job
is not found.


\subsection{Return a \texttt{JobList} as a \texttt{tuple}}
The \texttt{as\_tuples} function will return the joblist as a list of
tuples,
\begin{python}
x = urd.joblist.as_tuples
\end{python}
will return something like
\begin{verbatim}
[('csvimport', 'test-0'), ...]
\end{verbatim}


\subsection{Indexing and Slicing a \texttt{JobList}}
Since the \texttt{Joblist} is derived from a Python \texttt{list},
individual items and slices can be accessed just like a \texttt{list},
for example
\begin{python}
joblst = jl[3]
\end{python}
or
\begin{python}
joblst = jl[-2:]
\end{python}









\section{Configuration Information:  \texttt{urd.info}}
The dictionary \texttt{urd.info} contains configuration information
from the Accelerator daemon.  In particular, it contains these fields
\starttabletwo
\texttt{slices} & Configured number of slices.\\
\texttt{urd} & An URL to the Urd server.\\
\texttt{result\_directory} & see section~\ref{sec:configfile}.\\
\texttt{input\_directory} & see section~\ref{sec:configfile}.\\
\stoptabletwo



\section{Summary}
In a build script, the \texttt{urd} object has functionality for
building and retrieving jobs.  A job is built
using \texttt{urd.build()}, and references to all built jobs are
stored in \texttt{urd.joblist}.  These references could be fed as
input parameters to new jobs so that the output from one job could be
used as input by another.  The \texttt{urd.joblist} variable is
basically of type \texttt{list}, but with extra functionality to find
previous jobs.
