\label{chap:urd_basic}

Build scripts are used to instruct the Accelerator about which jobs to
build.  This chapter describes the basics of job building.  More
advanced features, using the \texttt{urd} server, is presented in
chapter~\ref{chap:urd}.


\section{Build Scripts}
\index{build scripts}
Build scripts are executed by the \texttt{automatarunner} executable.
A build script must contain the function \texttt{main}, since this is
called by the runner, like this
\index{build script!\texttt{main} function}
\begin{python}
def main(urd):
    ...
\end{python}
At run time, the \texttt{runner} inserts an object of the \texttt{Urd}
class as argument to the \texttt{main} function.  This \texttt{urd}
object has a number of member functions and attributes useful for job
building and tracking.  For tracking purposes, it remembers all jobs
that are built, together with their input parameters and some more
meta information.  This chapter will only cover the basic
possibilities provided by Urd, and a more comprehensive view will be
provided in the next chapter.



\subsection{Building a Job: \texttt{urd.build()}}
The \texttt{build} function is used to build a job from a method
(i.e.\ source file).  Here is an example of how to build the method
\texttt{method1}:
\begin{python}
def main(urd):
    urd.build('method1')
\end{python}
The full syntax for the \texttt{build} function is as follows
\index{\texttt{urd.build}}
\begin{python}
jobid = urd.build(method,
                  options={}, datasets={}, jobids={},
                  name='', caption='', workdir=None)
\end{python}
All paramters, except the name of the method, are optional, and
the \texttt{options}, \texttt{datasets}, and \texttt{jobids}
parameters must correspond to what is defined in the method to be
executed.  Urd will record the job based on the name of the method,
unless the \texttt{name=} is specified.  It is, for example, common to
build several \texttt{csvimport} jobs, and using \texttt{name=} they
can be told apart easily.  It is also possible to assign a caption to
a job, but this has no functional benefits.

When the job has been successfully built, the \texttt{build} function
will return a reference, a \textsl{jobid} to the job.  Similarly, if
the job already existed in an available \textsl{workdir}, the
\texttt{build} function immediately returns the jobid to that job
without executing anything.



\subsection{Connecting Jobs}
It is straightforward to connect jobs by feeding the output jobid from
the \texttt{build} function into a new job build.  For example
\begin{python}
jid_filter = urd.build('filter', datasets=dict(source=<some_input>))
jid_reduce = urd.build('reduce', datasets=dict(source=jid_filter))
\end{python}
In the example above, the first job, \texttt{filter}, creates a new
dataset from its input dataset.  This is then forwarded to the second
job, \texttt{reduce}, using the jobid reference \texttt{jid\_filter}.

What happens if the same build script is run a second time?  If there
has been no changes since the last run, the Accelerator will very
quickly find the jobids to the existing jobs and exit, thus both
finding the jobs as well as confirming the links and dependencies
between them.  On the other hand, if something has been modified, such
as a method's source code or any input parameters, the affected job(s)
will be re-executed.  For example, assume that the input to
the \texttt{filter} job is modified.  This will cause this method to
be executed again, leading to a new job with a new jobid.  This jobid
is input to the \texttt{reduce} job, causing it too to be re-executed.
Modifying the input to a sequence of jobs will cause all jobs to be
re-executed.  Modifying the parameters to, say, only the last job will
cause just that job to be re-executed.



\subsection{Building Chained Jobs: \texttt{urd.build\_chained()}}
There is a special version of \texttt{build()} that can be used for
linking a set of dataset-creating jobs.  This function was created for
the purpose of having build scripts that imported a large set of files
in a \texttt{for}-loop.  The call looks like this
\begin{python}
jobid = urd.build_chained('method1', name='myjob')
\end{python}
and it takes the same options as the standard \texttt{build} method,
with the exception that name is mandatory.  The method must also have
a ``\texttt{previous}'' key in its \texttt{datasets} parameter.  The
way it works is that the Accelerator will look up the latest job
having the same name and the new job to be built will have a link to
this job inserted in the \texttt{dataset.previous} parameter.



\subsection{Replaying Build Scripts}
This was mentioned in the previous section, but it is so fundamental
that it will be repeated here.  Executing a build script a second time
will not cause any new jobs to be executed, assuming nothing has been
changed.  Instead, the Accelerator will fill in the jobids of the
existing jobs so that processing can continue immediately.  A
successful ``replay'' of a build script ensures the integrity and
dependencies of the calculations.  If nothing has changed, the same
result remains.  If, however, some of the code has been modified, the
Accelerator will compute new jobs to reflect the new situation.  The
result may be different, and the user is notified.




\section{Inspecting Build History:  \texttt{urd.joblist}}
Information about previously executed jobs is stored in
the \texttt{urd.joblist} variable.  This variable is of
type \texttt{JobList}, which is basically a standard
ordered Python \texttt{list} with some additional features,
such as pretty-printing:
\begin{python}
def main(urd):
    jid1 = urd.build('first')
    jid2 = urd.build('second', jobids=dict(first=jid1))
    print(urd.joblist.pretty)
\end{python}
which results in
\begin{shell}
JobList(
   [  0]  first : TEST-38
   [  1] second : TEST-39
)
\end{shell}
Any job in the list can be accessed by its name, so for example the
jobid to the \texttt{first} job could be retrieved by
\begin{python}
urd.joblist['first'].jobid
\end{python}
Using the joblist, we can modify our example to
\begin{python}
def main(urd):
    urd.build('first')
    urd.build('second', jobids=dict(first=urd.joblist['first'].jobid))
\end{python}
The ``\texttt{.jobid}'' is a bit clumsy, but currently required to get
the actual jobid for the job.  There is also a \texttt{.method} that
will return the name of the last method in the joblist.

Note that if there are several jobs created by the same method, they
will have the same key in the joblist.  This situation could be
resolved using unique names to the \texttt{name=} option to
the \texttt{build()}-function as mentioned earlier.

An important special case is that the \textsl{latest} jobid is always
accessible using \texttt{urd.joblist.jobid}, so the code example could
also be written
\begin{python}
def main(urd):
    urd.build('first')
    urd.build('second', jobids=dict(first=urd.joblist.jobid))
\end{python}

\comment{.find() vs joblist[blah]}
It is also possible to look up jobs by name in the \texttt{joblist}.
This is done using the \texttt{find} function
\begin{python}
urd.joblist.find('csvimport')
\end{python}
If there is more than one job matching, the last one will be returned.
There is also a \texttt{.all} method that will return the joblist as a
comma separated list of jobids
\begin{python}
print(urd.joblist.all)
\end{python}
\begin{shell}
TEST-38,TEST-39
\end{shell}

Finally, it may be worth noting that
the \texttt{Joblist} is actually a \texttt{list}, so it is possible to
access individual items and slices like this
\begin{python}
jl[3]
\end{python}
and linke this
\begin{python}
jl[-2:]
\end{python}


\section{Configuration Information:  \texttt{urd.info}}
The dictionary \texttt{urd.info} contains configuration information
from the Accelerator daemon.  In particular, it contains these fields
\starttabletwo
\RPtwo \texttt{slices} & Configured number of slices.\\[1ex]
\RPtwo \texttt{urd} & An URL to the Urd server, if configured,
  \pyNone otherwise.\\[1ex]
\RPtwo \texttt{result\_directory} & see section~\ref{}.\\[1ex]
\RPtwo \texttt{common\_directory} & see section~\ref{}.\\[1ex]
\RPtwo \texttt{source\_directory} & see section~\ref{}.\\[1ex]
\stoptabletwo



\section{Summary}
The \texttt{urd} object has functionality for building and retreiveing
jobs.  A job is built using \texttt{urd.build()}, and references to
all built jobs will be stored in \texttt{urd.joblist}.  These
references could be fed as input parameters to new jobs so that the
output from one job could be used as input by another.
The \texttt{urd.joblist} variable is basically of type \texttt{list},
but with extra functionality to find previous jobs and their jobids.
