\section{Introduction}

The dataset is the prefered way to store large amounts of data.  The
dataset is the container for fast and simple access to data.  Data in
a dataset are stored as a matrix in rows and columns.  Using the
dataset, data is simple to access at very high speed.

Datasets are created by methods, and any job may contain zero, one, or
more datasets.  The most obvious use of a dataset is the cvsimport
method that creates a dataset from an input file, but much more
advanced useage is possible since a job may contain more than one
dataset.  This allows for efficient storage and access of data in some
common practical situations.  For example, a filtering job may split
the input dataset in two or more output datasets that can be accessed
independently.

For performance reasons, datasets are typically split into several
slices, where each data row exists in exactly one of the slices.  The
actual slicing may be carried out in any fashion, like round robin, or
even random, but an interesting approach is to slice according to the
hash value of a certain column.  Slicing according to a hashed column
ensures that all rows with a certain hash column value ends up in the
same slice.

This chapter covers ...
chaining, where datasets grows in number of lines, and column
appending, where datasets grow in number of columns.



\section{Dataset access}

A method's dataset inputs are specified in the \texttt{datasets}
tuple.  In a running job, the datasets in the input tuple are objects
of the \texttt{Dataset} class.  For example

\begin{python}
datasets = ('tlog',)

def synthesis():
  print(datasets.tlog.filename)
\end{python}
In this example, the \texttt{tlog} input dataset is an object of class
\texttt{Dataset} when the method is running, so printing its filename
is straightforward.  Most common dataset operations are available as
member functions to the \texttt{Dataset} class, so this automatic
instantiation is very convenient.

A dataset object can also be instantiated from a jobid, like this

\begin{python}
from dataset import Dataset
...
d = Dataset('foo-0')
\end{python}
In this case, \texttt{d} will be an object based on the default
dataset residing at jobid \texttt{foo-0}.  If the dataset is stored by
a different name, it may be accessed like this

\begin{python}
d = Dataset('foo-0/bar')
# or
d = Dataset(('foo-0', 'bar'))
\end{python}
where the latter approach is a bit more flexible in that it uses
parameters for both jobid and name.




\newpage
\section{Dataset Properties}

The dataset class has a number of member functions and attributes that
is intended to make it simple to work with datasets.



\subsubsection{Column Names}

All columns in a dataset may be aquired using the \texttt{columns} property, like this

\begin{python}
datasets = ('source',)

def synthesis():
  print(datasets.source.columns.keys())
  # ['GTIN', 'date', 'locale', 'subsource']
\end{python}



\subsubsection{Column Properties}

For each column, the name, type, and if applicable, the min and max
values are accessible like this

\begin{python}
# each key, i.e. column, has a number of properties, of which the
# most important ones are shown below
print(datasets.source.columns['locale'].type)
# ascii
print(datasets.source.columns['locale'].name)
# locale
print(datasets.source.columns['locale'].min)
# 3
print(datasets.source.columns['locale'].max)
# 9
\end{python}

\subsubsection{Rows per Slice}

It may be interesting to see how many rows there are per slice in a
dataset.  This information is available as a list, for example

\begin{python}
print(datasets.source.lines)
# [5771, 6939, 6212, 6312, 6702, 6341, 5988, 6195,
#  6741, 6587, 6518, 5840, 6327, 5933, 6745, 6673,
#  6536, 6405, 6259, 6455, 6036, 6088, 6937, 6245,
#  6418, 6437, 6360, 6106, 6878]
\end{python}
The first item in the list is the number of rows in slice 0, and so
fourth.  From an efficiency perspective, the variance of the numbers
in this list should be low, so that all slices will contain about the
same amount of data.

\subsubsection{Dataset Shape (I.e.\ Number of Columns and Total Number
  of Lines)}

The shape if the dataset, i.e.\ the number of columns times the total
number of rows, is available from the shape attribute

\begin{python}
print(datasets.source.shape)
# (4, 184984)
\end{python}
the second number is exactly the sum of the number of lines for each
slice from above.



\subsubsection{Hashlabel}

If the dataset is hashed on a particular column, this column is stored
in the hashlabel attribute

\begin{python}
print(datasets.source.hashlabel)
# GTIN
\end{python}



\subsubsection{Filename and Caption}

The dataset may have a filename associated to it.  This makes sense in
situations for example where the dataset is created from an input
datafile using \texttt{csvimport} or similar.  The filename is
accessable like this

\begin{python}
print(datasets.source.filename)
# /data/incoming/raw_repository_5391.gz
\end{python}
It is possible to set a caption when creating a dataset.  The caption
is entirely user-defined and has no function in the Accelerator.  The
caption is accessible like this

\begin{python}
print(datasets.source.caption)
# flattening
\end{python}



\subsubsection{Chains}

When a dataset is created, it is optional to input a link to another
dataset using the parameter \texttt{previous}.  This has effect on the
dataset iterators, which may continue to iterate over dataset
boundaries when one dataset is exhausted and continue to the next.
This will be described in another section in more detail.  The
previous dataset is available as an attribute

\begin{python}
print(datasets.source.previous)
# neu4-4893/default
\end{python}



\newpage
\section{Column Typing}
The dataset columns are typed.  All types are shown in table~\ref{tab:types}.
Some of the types are explained in more detail in the following.

\begin{table}[h!]
 \begin{tabular*}{\textwidth}{l @{\extracolsep{\fill}} l}
  \hline
   \textbf{typename}   & \textbf{explanation} \\[1ex]
   \texttt{number}     &  float or int\\
   \texttt{number:int} &  int\\[1ex]
   \texttt{float64}   &  64 bit (double) float\\
   \texttt{float32}   &  32 bit float\\
   \texttt{int64}     &  64 bit signed integer\\
   \texttt{int32}     &  32 bit integer\\[1ex]
   \texttt{bool}      &  True or False\\[1ex]
   \texttt{date}      &  date\\
   \texttt{time}      &  time\\
   \texttt{datetime}  &  complete date and time object\\[1ex]
   \texttt{bytes}     &  raw input, avoid \\
   \texttt{ascii}     &  ascii is faster in python2, otherwise use unicode\\
   \texttt{unicode}   &  use for strings\\[1ex]
   \texttt{parsed:number}   & int, float or string parsing into \texttt{number} \\
   \texttt{parsed:float64}  & int, float or string parsing into \texttt{float64} \\
   \texttt{parsed:float32}  & int, float or string parsing into \texttt{float32} \\
   \texttt{parsed:int64}    & int, float or string parsing into \texttt{int64} \\
   \texttt{parsed:int32}    & int, float or string parsing into \texttt{int32} \\[1ex]
   \texttt{json}            &  a datastructure that is jsonable\\
   \texttt{parsed:json}     &  string containing parseable json\\[1ex]
   \hline
 \end{tabular*}
 \caption{Available dataset column types.}
 \label{tab:types}
\end{table}


\subsection{Arbitrary precision numbers:  \texttt{number}}
The type \texttt{number} is integer when possible and float otherwise.
it can handle very large numbers, up to $\pm (2^{1007}-1)$.  Integers
are forced using \texttt{number:int}, and it accepts trailing decimal
zeroes like \texttt{7.0}, \texttt{4.000} etc.  This is useful when
typing datafiles where numbers are integers but have trailing zero
decimals.

The number type occupies a minimum of nine bytes, where eight is for
the number and the additional byte is a marker.


\subsection{Standard fixed size numbers}
The common \texttt{int} and \texttt{float} types in 32 and 64 bit
versions are available for useage when the range of the data is known.


\subsection{The boolean type}
The \texttt{bool} type is used to store logical \texttt{True} or
\texttt{False} values only.


\subsection{Types relating to time}
The \texttt{date}, \texttt{time}, and \texttt{datetime} are compatible
with Python's corresponding classes.  \texttt{datetime} is the
combination of \texttt{date} and \texttt{time}.  A column that is
typed to any of these may directly take advantage of the high level
time related methods, like for example

\begin{python}
  for ts in datasets.source.iterate(sliceno, 'timestamp'):
    print(ts.strftime('%Y-%m-%d')
\end{python}


\subsection{String types}
There are three string types, \texttt{bytes}, \texttt{ascii}, and
\texttt{unicode}.  Unicode is the prefered choice, and executes faster
in Python~3.

\textbf{To be completed.}

\subsection{JSON types}
It is possible to store a more complex data structure in JSON format.
There are two choice, the \texttt{json} type and the
\texttt{parsed:json} type.  The first takes a JSON-able datatype as
input, and the latter inputs a string that is serialized to JSON
internally.



\newpage
\section{Create New Dataset}
Datasets are either created in prepare + analysis, or in just
synthesis.

\subsection{Create in prepare + analysis}
A simple example writing three columns to the default dataset is
presented next.  Note that the example creates a dataset chain,
linking the currently created dataset to the dataset that is input
with the name previous.

\begin{python}
from dataset import DatasetWriter
datasets = ('previous',)

def prepare():
  dw = DatasetWriter(
    hashlabel = 'X',
    previous = datasets.previous,
  )
  dw.add('X', number)
  dw.add('Y', number)
  dw.add('Z', number)
  return dw

def analysis(sliceno, prepare_res):
  dw = prepare_res
  ...
  for x, y, z in data:
    dw.write(x, y, z)
\end{python}
Note that the order of the variables in the dw.write function call is
the same as the order of the add calls in prepare\footnote{in case
  write is called with a dict, the order is unknown, but then names
  are looked up using the dict keys.}.

DatasetWriter takes a number of optional arguments such as caption and
filename.  The argument ``name'' specifies the name of the dataset,
which is set to be ``default'' when unassigned.  Several datasets can
be created in the same method using more than one datasetwriter
instance with different ``name''s.

There is some flexibility in the way the write function may be called

\begin{python}
  dw.write_dict({column: value})
  dw.write_list([value, value, ...])
  dw.write(value, value, ...)
  # or even
  dw.writers[name].write(value)  # return True if hashed to correct slice
\end{python}



\subsection{Creating Hashed datasets}

If hashlabel is set, one can use dw.hashcheck(value) to check if value
belongs to the slice.  It is also possible to just call the writer
since it will discard anything not belonging to the correct slice.



\subsection{Create in synthesis}

There are two options if the dataset is to be created in synthesis.
One in to set the slice number first

\begin{python}
  dw.set_slice(sliceno)
\end{python}
while the other is to use one of these functions

\begin{python}
  dw.get_split_write_dict()({column: value})
  dw.get_split_write_list()([value, value, ...])
  dw.get_split_write()(value, value, ...)
\end{python}
that will assign the data to the correct slice automatically.

\subsection{Placeholder:  Creating datasets more manually}


\newpage
\section{Appending new columns to an existing Dataset}

With minimal overhead, the dataset supports adding new columns to an
existing dataset.  This is implemented by storing the new column data
together with a pointer to the dataset

Appending new columns work exactly the same way as creating a dataset,
with the exception that a link to a dataset that is to be appended to
is input to the writer constructor.  The following example appends one
column to an existing dataset while maintaining the chain.  Note that
appending does only apply to one dataset, and not to the complete
chain.

\begin{python}
datasets = ("source", "previous",)

def prepare():
  dw = dataset.DatasetWriter(
    parent=datasets.source,
    previous=datasets.previous,
    caption="flattening_attempt_1"
  )
  dw.add(name, type)
  return dw

def analysis(sliceno, prepare_res):
  dw = prepare_res
  ...
  dw.write(value)
\end{python}
Note that an error is issued if the total number of appended lines
does not match the number of lines in the parent dataset.
