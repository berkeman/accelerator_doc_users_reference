
This chapter presents an overview of the Accelerator's features in a
rather non-formal way.  It is based on an article published on the
eBay Tech Blog website.



\section{High Level View}
The Accelerator is a client-server based application, and from a high
level, it can be visualised like in figure~\ref{fig:overview}.
\begin{figure}[h!]
  \begin{center}
    %%%%%%%%%%%%%%%%%%%%%%%%%%%%%%%%%%%%%%%%%%%%%%%%%%%%%%%%%%%%%%%%%%%%%%%%%%%%
%                                                                          %
% Copyright (c) 2018 eBay Inc.                                             %
% Modifications copyright (c) 2019-2021 Anders Berkeman                    %
%                                                                          %
% Licensed under the Apache License, Version 2.0 (the "License");          %
% you may not use this file except in compliance with the License.         %
% You may obtain a copy of the License at                                  %
%                                                                          %
%  http://www.apache.org/licenses/LICENSE-2.0                              %
%                                                                          %
% Unless required by applicable law or agreed to in writing, software      %
% distributed under the License is distributed on an "AS IS" BASIS,        %
% WITHOUT WARRANTIES OR CONDITIONS OF ANY KIND, either express or implied. %
% See the License for the specific language governing permissions and      %
% limitations under the License.                                           %
%                                                                          %
%%%%%%%%%%%%%%%%%%%%%%%%%%%%%%%%%%%%%%%%%%%%%%%%%%%%%%%%%%%%%%%%%%%%%%%%%%%%


This chapter presents an overview of the Accelerator's features in a
rather non-formal way.  It is based on an article published on the
eBay Tech Blog website.



\section{High Level View}
The Accelerator is a client-server based application, and from a high
level, it can be visualised like in figure~\ref{fig:overview}.

\begin{figure}[h!]
  \begin{center}
    %%%%%%%%%%%%%%%%%%%%%%%%%%%%%%%%%%%%%%%%%%%%%%%%%%%%%%%%%%%%%%%%%%%%%%%%%%%%
%                                                                          %
% Copyright (c) 2018 eBay Inc.                                             %
% Modifications copyright (c) 2019-2021 Anders Berkeman                    %
%                                                                          %
% Licensed under the Apache License, Version 2.0 (the "License");          %
% you may not use this file except in compliance with the License.         %
% You may obtain a copy of the License at                                  %
%                                                                          %
%  http://www.apache.org/licenses/LICENSE-2.0                              %
%                                                                          %
% Unless required by applicable law or agreed to in writing, software      %
% distributed under the License is distributed on an "AS IS" BASIS,        %
% WITHOUT WARRANTIES OR CONDITIONS OF ANY KIND, either express or implied. %
% See the License for the specific language governing permissions and      %
% limitations under the License.                                           %
%                                                                          %
%%%%%%%%%%%%%%%%%%%%%%%%%%%%%%%%%%%%%%%%%%%%%%%%%%%%%%%%%%%%%%%%%%%%%%%%%%%%


This chapter presents an overview of the Accelerator's features in a
rather non-formal way.  It is based on an article published on the
eBay Tech Blog website.



\section{High Level View}
The Accelerator is a client-server based application, and from a high
level, it can be visualised like in figure~\ref{fig:overview}.

\begin{figure}[h!]
  \begin{center}
    %%%%%%%%%%%%%%%%%%%%%%%%%%%%%%%%%%%%%%%%%%%%%%%%%%%%%%%%%%%%%%%%%%%%%%%%%%%%
%                                                                          %
% Copyright (c) 2018 eBay Inc.                                             %
% Modifications copyright (c) 2019-2021 Anders Berkeman                    %
%                                                                          %
% Licensed under the Apache License, Version 2.0 (the "License");          %
% you may not use this file except in compliance with the License.         %
% You may obtain a copy of the License at                                  %
%                                                                          %
%  http://www.apache.org/licenses/LICENSE-2.0                              %
%                                                                          %
% Unless required by applicable law or agreed to in writing, software      %
% distributed under the License is distributed on an "AS IS" BASIS,        %
% WITHOUT WARRANTIES OR CONDITIONS OF ANY KIND, either express or implied. %
% See the License for the specific language governing permissions and      %
% limitations under the License.                                           %
%                                                                          %
%%%%%%%%%%%%%%%%%%%%%%%%%%%%%%%%%%%%%%%%%%%%%%%%%%%%%%%%%%%%%%%%%%%%%%%%%%%%


This chapter presents an overview of the Accelerator's features in a
rather non-formal way.  It is based on an article published on the
eBay Tech Blog website.



\section{High Level View}
The Accelerator is a client-server based application, and from a high
level, it can be visualised like in figure~\ref{fig:overview}.

\begin{figure}[h!]
  \begin{center}
    \input{figures/overview.pdftex_t}
    \caption{High level view of the Accelerator framework.  See text
      for details.}
    \label{fig:overview}
  \end{center}
\end{figure}

In the figure, the client side, with shell commands and web interface,
is on the left, and the server side, with job databases is to the
right.  The Accelerator executes \textsl{build scripts}, that issue
\textsl{jobs} on the server.  These jobs are stored in the server's
job databases for later retreival.  There are two databases, one for
storing everything related to a job's execution, called the
\textsl{workdir}, and one transaction log used for storing
meta-information about built jobs.  The workdir will contain all
inputs, source code, parameters, and outputs of all executed jobs,
whereas the job log database ensures reproducibility and transparency,
and it will be further discussed in chapter~\ref{chap:urd}.



\section{Jobs}
The basic operation of the Accelerator is to execute small Python
programs called \textsl{methods}.  A method is nothing but a Python
program with one or a few special functions that are used to execute
code sequentially or in parallel and to pass parameters and results.
A method that has completed execution is called a \textsl{job}.

Jobs are stored in \textsl{job directories}.  A dedicated directory in
the workdir will be created for each new job, and this directory will
contain all information regarding the job, such as its input
parameters, stored files, return values, profiling information, Python
interpreter version and path, and more.

The Accelerator has a database that keeps track of all jobs that has
been run.  This avoids unnecessary re-computing in favour of re-using
previously computed results.  Re-using jobs does not only speed up
processing and encourage incremental design, but also makes it
transparent which code and which data that was used for any particular
result, minimising these kind of uncertainties to zero.


\subsection{A Very Simple Job:  ``Hello, World''}
Here's an example of a very simple method.  It does not take any input
parameters and does almost nothing, it will just return the string
``\texttt{hello world}'' and exit.
\begin{python}
def synthesis():
    return "hello world"
\end{python}
Everything inside the \texttt{synthesis()}-function is run once
without any parallelisation.  In order to get this method to execute,
it is called from a \textsl{build script} looking something like this
\begin{python}
def main(urd)
    job = urd.build('hello_world')
    print(job.load())
\end{python}
The \texttt{urd} object made available by the \texttt{main()}-function
contains functions for job building and organisation, and is described
in chapter~\ref{chap:urd} and section~\ref{sec:classes:urd}.  This
build script is then run by a shell command like this
\begin{shell}
ax run
\end{shell}
The \texttt{build()}-call will instruct the server to execute the
\texttt{hello\_world} method.  During its execution, a job directory
will be created that contains everyting associated with the build
process.  When the \texttt{build()}-call is completed, a job object, of type
\texttt{Job}, is returned to the program.  This object provides a
convenient interface to the data in the corresponding job directory,
and contains member functions such as \texttt{.load()}, that is used
in the example to read back the returned value from the job.


\subsection{Jobs Can Only be Run Once}
If the build script is executed again, the \texttt{hello\_world} job
will not be re-built, simply because the Accelerator remembers that
the job has been built in the past, and its associated information is
stored in a job directory.  Instead, the Accelerator immediately
returns a job object representing the previous run.  This means that
from a user's perspective, there is no visible difference between
running a job for the first time or re-using results from an existing
run!  In order to have the method executing again, either the source
code or input parameters has to change.  If there are changes, the
method will be re-executed, and a new job will be created that
reflects these changes.


\subsection{Back to the ``Hello, World'' Example}
Figure~\ref{fig:execflow-hello-world} illustrates the dispatch of the
\texttt{hello\_world} method.  The created job gets the \textsl{jobid}
\texttt{test-0}, and parts of the corresponding job directory
information is shown in green.  (Jobids are job identifiers, that are
named by their corresponding \textsl{workdir} plus an integer counter
value.)  The job directory contains several files, of which the most
important are
\begin{itemize}
\item[] \texttt{setup.json}, containing job meta information;
\item[] \texttt{result.pickle}, containing the returned data; and
\item[] \texttt{method.tar.gz}, containing the method's source code.
\end{itemize}

\begin{figure}[b]
  \begin{center}
    \input{figures/job0.pdftex_t}
    \caption{A simple hello world program, represented as graph and
      work directory.}
    \label{fig:execflow-hello-world}
  \end{center}
\end{figure}

The \texttt{Job} object provides a convenient way to access files and
data stored in this directory.  For example, as we've already seen,
the job's return value can be loaded into a variable using the
\texttt{.load()} function.



\subsection{Workdirs and Sharing Jobs}

Workdirs are used to store jobs.  There can be more than one workdir,
and different workdirs can be used separate jobs into different
physical locations.  The Accelerator can be set up to have any number
of workdirs associated, but only one is used for writing.

If the same workdir is entered into two or more different user's
configuration files, the workdir and its contents will be shared
between the users.  Each Accelerator server will update its knowledge
about the contents of all workdirs before executing a build script, to
make sure that the latest jobs are taken into account.  In addition,
the job log database, as described in chapter~\ref{chap:urd}, is
designed for efficient re-use and sharing of particularly interesting
jobs.


\subsection{Linking Jobs}

Using jobs, complex tasks can be split into several smaller
operations.  Jobs can be connected so that the next job will depend on
the result of a previous job or set of jobs, and so on.

To continue the simple hello world example, assume for a second that
the \texttt{hello\_world}-job is computationally expensive, and that
it returns a result that is to be used as input to further processing.
To keep things simple, this further processing is represented by
printing the result to standard output.  A new method
\texttt{print\_result} is created, and it goes like this
\begin{python}
jobs = ('hello_world_job',)

def synthesis():
    data = jobs.hello_world_job.load() 
    print(data)
\end{python}
This method expects the \texttt{hello\_world\_job} input parameter to
be provided at execution time, and it is accomplished by the following
extended build script
\begin{python}
def main(urd):
    job1 = urd.build('hello_world')
    job2 = urd.build('print_result', hello_world_job=job1)
\end{python}
The \texttt{print\_result} method then loads the result from the
provided job and prints its contents to \texttt{stdout}.  Note that
this method does not return anything.

Figure~\ref{fig:execflow-print-result} illustrates the situation.
(Note the direction of the arrow: the second job, \texttt{test-1} has
\texttt{test-0} as input parameter, but \texttt{test-0} does not know
of any jobs run in the future.  Hence, arrows point to previous jobs.)

\begin{figure}[b]
  \begin{center}
    \input{figures/job0job1.pdftex_t}
    \caption{Job \texttt{test-0}, is used as input to the
      \texttt{print\_result} job.}
    \label{fig:execflow-print-result}
  \end{center}
\end{figure}

The example shows how a complex task may be split into several jobs,
each reading intermediate results from previous jobs.  The Accelerator
will keep track of all job dependencies, so there is no doubt which
jobs that are run when and on which data.  Furthermore, since the
Accelerator remembers if a job has been executed before, it will link
and re-use previous jobs when possible.  This may bring a significant
improvement in execution speed.  Furthermore, a re-used job is a proof
of that the code, input- and output data is unchanged and connected.


\section{Datasets: Storing Data}

The \texttt{dataset} is the Accelerator's default storage type for
small or large quantities of data, designed for parallel processing
and high performance.  Datasets are built on top of jobs, so
\emph{datasets are created by methods and stored in job directories,
  just like any job result.}

Internally, data in a dataset is stored in a row-column format, and is
typically \emph{sliced} into a fixed number of slices to allow
efficient parallel access, see figure~\ref{fig:dataset}. Columns are
accessed independently, so there is no overhead in reading a single or
a set of columns.  Only the data relevant for a task is read from
disk.


\begin{figure}[h!]
  \begin{center}
    \input{figures/dataset_files.pdftex_t}
    \caption{A dataset containing three columns, $A$, $B$, and $C$
      stored using two slices.  Each dotted box corresponds to a file,
      so there are two files for each column, allowing for parallel
      read of the data using two processes.}
    \label{fig:dataset}
  \end{center}
\end{figure}

Furthermore, datasets may be \textsl{hash partitioned}, so that
slice-membership is based on the hash value of a given column.
Partitioning base on, for example, a column containing some ID string
will puy all rows corresponding to any particular ID in a single slice
only.  In many practical applications, hash partitioning makes
parallel processes independent, minimising the need for complicated
merging operations.  This is explained further in
section~\ref{sec:slicing_and_hashing}.



\subsection{Importing Data}

A project typically starts with \textsl{importing} some data from a
file on disk.  The bundled method \texttt{csvimport} is designed to
parse a plethora of ``comma separated values''-file formats and store
the data as a dataset.  See figure~\ref{fig:dataset_csvimport}.
\begin{figure}[b]
  \begin{center}
    \input{figures/import_file1.pdftex_t}
    \caption{Importing \texttt{file0.txt}.}
    \label{fig:dataset_csvimport}
  \end{center}
\end{figure}
The method takes several input options in addition to the mandatory
filename to control the import process.  Here is an example
invocation
\begin{python}
def main(urd):
    jid = urd.build('csvimport', filename='file0.txt')
\end{python}
When executed, the created dataset will be stored in the resulting job
directory, and the name of the dataset will by default be the jobid
plus the string \texttt{default}.  For example, if the
\texttt{csvimport} jobid is \texttt{imp-0}, the dataset will be
referenced by \texttt{imp-0/default}.  In this case, and always when
there is no ambiguity, the jobid alone (\texttt{imp-0}) could be used
too, for simplicity.  In general, a job could contain any number of
datasets, but a single dataset is a common case.




\subsection{Linking Datasets, Chaining}

Just like jobs can be linked to each other, datasets can link to each
other too.  Since datasets are build on top of jobs, this is
straightforward.  Assume the file \texttt{file0.txt} is imported into
dataset \texttt{imp-0/default}, and that there is more data like it
stored in the file \texttt{file1.txt}.  The second file is imported
with a link to the first dataset, see
figure~\ref{fig:dataset_csvimport_chain}.
\begin{figure}[t]
  \begin{center}
    \input{figures/import_file0file1.pdftex_t}
    \caption{Chaining the import of \texttt{file1.txt} to the previous
      import of \texttt{file0.txt}.}
    \label{fig:dataset_csvimport_chain}
  \end{center}
\end{figure}
The \texttt{imp-1} (or \texttt{imp-1/default}) dataset reference can
now be used to access all data imported from \textsl{both} files!

Linking datasets containing related content is called \emph{chaining},
and this is particularly convenient when dealing with data that grows
over time.  A good example is any kind of \emph{log} data, such as
logs of transactions, user interactions, and similar.  Using chaining,
datasets can be with more rows just by linking, which is a lightweight
constant time operation.



\subsection{Adding New Columns to a Dataset}
In the previous section it was shown that datasets can be chained and
thereby grow in number of rows.  A dataset chain is created simply by
linking one dataset to the other, so the overhead is minimal.  In this
section it is shown that by the same principles, it is equally simple
to add new columns to existing datasets.  Adding columns is a common
operation and the Accelerator handles this situation efficiently using
links.

The idea is very simple.  Assume a ``source'' dataset to which one or
more new columns should be added.  A new dataset is created containing
\textsl{only} the new column(s), and while creating it, the
constructor is instructed to link all the source dataset's columns to
the new dataset such that the new dataset appears to contain all
columns from both datasets.  (Note that this linking is similar to but
different from chaining.)

Accessing the new dataset will transparently access all the columns in
both the new and the source dataset in parallel, making it
indistinguishable from a single dataset.  See
Figure~\ref{fig:dep_dataset_append_column}.

\begin{figure}[b]
  \begin{center}
    \input{figures/dataset_append_column.pdftex_t}
    \caption{Adding one new column to the source dataset.}
    \label{fig:dep_dataset_append_column}
  \end{center}
\end{figure}

A common case is to compute new columns based on existing ones.  In
this case, values are written to the new columns in the new dataset
while reading from the iterator iterating over the existing columns in
the source dataset.  This will be discussed in detail in
section~\ref{sec:appending_new_columns}



\subsection{Multiple Datasets in a Job}

Typically, a method creates a single dataset in the job directory, but
there is no limit to how many datasets that could be created and
stored in a single job directory.  This leads to some interesting
applications.

One application for keeping multiple datasets in a job is when data is
split into subsets based on some condition.  This could, for example,
be when a dataset is split into a training set and a test set.  One
way to achieve this using the Accelerator is by creating a Boolean
column that tells if the current row is train or test data, followed
by a job that splits the dataset in two based on the value on that
column.  See Figure~\ref{fig:dep_dataset_csvimport_chain}.

\begin{figure}[h!]
  \hspace{1cm}
  \input{figures/filter_dataset.pdftex_t}
  \caption{\texttt{job-1} separates the dataset
    \texttt{job-0/default} into two new datasets, named
    \texttt{job-1/train} and \texttt{job-1/test}.}
  \label{fig:dep_dataset_csvimport_chain}
\end{figure}

In the setup of figure~\ref{fig:dep_dataset_csvimport_chain} we have
full tracking from either \texttt{train} or \texttt{test} datasets.
If we want to know the source of one of these sets, we just follow the
links back to the previous jobs until we reach the source job.  In the
figure, \texttt{job-0} may for example be a \texttt{csvimport} job,
and will therefore contain the name of the input file in its
parameters.  Thus, it is straightforward to link any data to its
source.

Splitting a dataset into parts creates ``physical'' isolation while
still keeping all the data at the same place.  No data is lost in the
process, and this is good for transparency reasons.  For example, a
following method may iterate over \textsl{both} datasets in
\texttt{job-1} and by that read the complete dataset.



\subsection{Parallel Dataset Access and Hash Partitioning}
As shown earlier in this chapter, data in datasets is stored in
multiple files for two reasons.  The first reason is that we can read
only the columns that we need, without overhead, and the second reason
is that it allows fast parallel reads.  The parameter \texttt{slices}
determines how many slices that the dataset should be partitioned
into, and it also sets the number of parallel process that is used for
processing the dataset.  There is always one process for each slice of
the dataset, and each process operates on a unique part of the
dataset.

Datasets can be partitioned, or sliced, in different ways.  One
obvious way is to use round robin, where each consecutive data row is
written to the next slice, modulo the number of slices.  This leads to
``well balanced'' datasets with approximately equal number of rows per
slice.  Another alternative to slicing is to slice based on the hash
value of a particular column's values.  Using this method, all rows
with the same value in the hash column end up in the same slice.  This
is efficient for many parallel processing tasks, and will be described
in detail later on.




\subsection{Dataset Column Types}

There are large a number of useful types available for dataset
columns.  They include \textsl{floating} and \textsl{integer point
  numbers}, \textsl{Booleans}, \textsl{timestamps}, several
\textsl{string types} (handling all kinds of string encodings), and
\textsl{json} as well as \textsl{pickle} types for storing arbitrary
data collections.  Most of these types come with advanced parsers,
making importing data from text files straightforward with
deterministic handling of errors, overflows, and so on.



\subsection{Dataset Attributes}
The dataset has a number of attributes associated with it, such as
shape, number of rows, column names and types, and more.
An attribute is accessed like this
\begin{python}
datasets = ('source',)
def synthesis():
    print(datasets.source.shape)
    print(datasets.source.columns)
\end{python}
and so on.


%%%%%%%%%%%%%%%%%%%%%%%%%%%%%%%%%%%%%%%%%%%%%%%%%%%%%%%%%%%%%%%%%%%%%%%%%%%%%%%%

\section{Iterators: Working with Data}

Data in a dataset is typically accessed using an \emph{iterator} that
reads and streams one dataset slice at a time to a CPU core.  The
parallel processing capabilities of the Accelerator makes it possible
to dispatch a set of parallel iterators, one for each slice, in order
to have efficient parallel processing of the whole dataset.

This section shows how iterators are used for reading data, how to
take advantage of slicing to have parallel processing, and how to
efficiently create new datasets.


\subsection{Iterator Basics}

Basic iterator functionality is introduced here using an example.
Assume a dataset that has a column containing movie titles named
\texttt{movie}, and the problem is to extract the ten most frequently
occuring movies.  Here's a complete method solving this problem
\begin{python}
from collections import Counter
datasets = ('source',)

def synthesis():
    c = Counter(datasets.source.iterate(None, 'movie'))
    top10 = c.most_common(10)
    print(top10)
    return top10
\end{python}
This will print the ten most common movie titles and their
corresponding counts in the \texttt{source} dataset.  The code will
run on a single CPU core, because we use the single-process
\texttt{synthesis()} function, which is called and executed only once.
The \texttt{dataset.iterate()} (class-)method therefore has to read
through all slices, one at a time, in a serial fashion, and this is
reflected by the first argument to the iterator being \pyNone.  The
method also returns the variable \texttt{top10} so that it can be used
by other methods.



\subsection{Parallel Execution}
It is easy to write parallel programs using the Accelerator.  The fact
that data in a dataset is sliced into disjoint sets and files makes
parallel processing of data straightforward.  Here's a slightly
modified version of the program from the previous section that will
now execute in parallel
\begin{python}
def analysis(sliceno):
    return Counter(datasets.source.iterate(sliceno, 'movie'))

def synthesis(analysis_res)
    c = analysis_res.merge_auto()
    top10 = c.most_common(10)
    return top10
\end{python}
For larger datasets, this parallel version of the movie title counter
will run much faster.  Here, \texttt{.iterate()} is moved inside the
\texttt{analysis()} function.  This function is forked once for each
slice, and the argument \texttt{sliceno} will contain an integer
between zero and the number of slices minus one.  The returned value
from the analysis functions will be available as input to the
synthesis function in the \texttt{analysis\_res} Python iterable.  It
is possible to merge the results explicitly, but the this iterable
comes with a rather magic method \texttt{merge\_auto()}, that merges
the results from all slices into one based on the data type.  It can
for example merge \texttt{Counter}s, \texttt{set}s, and composed types
like \texttt{set}s of \texttt{Counter}s, and so on.


\subsection{Iterating over Several Columns}
Since each column is stored independently in a dataset, there is no
overhead from reading a subset of a dataset's columns.  In the
previous section we've seen how to iterate over a single column using
\texttt{.iterate()}.  Iterating over more columns is straightforward
by feeding a list of column names to \texttt{.iterate()}, like in this
example
\begin{python}
from collections import defaultdict
datasets = ('source',)

def analysis(sliceno):
    user2movieset = defaultdict(set)
    for user, movie in datasets.source.iterate(sliceno, ('user', 'movie')):
        user2movieset[user].add(movie)
    return user2movieset
\end{python}
This example creates a lookup dictionary from users to sets of movies.
Note that in this case, we would like to have the dataset hashed on
the \texttt{user} column, so that each user appears in exactly one slice.
This will make later merging (if necessary) much easier.

It is also possible to iterate over all columns by specifying an empty
list of columns or by using the value \pyNone.
\begin{python}
...
def analysis(sliceno):
    for columns in datasets.source.iterate(sliceno, None):
        ...
\end{python}
Here, \texttt{columns} will be a list of values, one for each column
in the dataset, in sorted column name order.


\subsection{Iterating over Dataset Chains}

The \texttt{iterate} function is used to iterate over a single
dataset.  There is a corresponding function, \texttt{iterate\_chain},
that is used for iterating over chains of datasets.  This function
takes a number of arguments, such as
\begin{itemize}
\item[] \texttt{length}, i.e.\ the number of datasets to iterate over.
  By default, it will iterate over all datasets in the chain.
\item[] \texttt{callbacks}, functions that can be called before and/or
  after each dataset in a chain.  Very useful for aggregating data
  between datasets.
\item[] \texttt{stop\_id} which stops iterating at a certain dataset.
  This dataset could be from \textsl{another} job's parameters, so we
  can for example iterate exactly over all new datasets not covered by
  a previous job.
\item[] \texttt{range}, which allows for iterating over a range of
  data.
\end{itemize}
The \texttt{range} options is based on the max/min values stored for
each column in the dataset.  Assuming that the chain is sorted, one
can for example set
\begin{python}
range={timestamp, ('2016-01-01', '2016-01-31')}
\end{python}
in order to get rows within the specified range only.  Using
\texttt{range=} is quite costly, since it requires each row in the
dataset chain with dates within the range to be checked against the
range criterion.  Therefore, there is a \texttt{sloppy} version that
iterates over complete datasets in the chain that contains at least
one row with a date within the range.  This runs at full speed, and is
useful, for example, to very quickly produce histograms or plots of
subsets of a huge dataset.



\subsection{Job Execution Flow and Result Passing}

Execution of code in a method is either parallel or serial depending
on which function is used to encapsulate it.  There are three
functions in a method that are called from the Accelerator when a
method is running, and they are \texttt{prepare()},
\texttt{analysis()}, and \texttt{synthesis()}.  All three may exist in
the same method, and at least one is required.  When the method
executes, they are called one after the other.
\begin{itemize}
\item[] \texttt{prepare()} is executed first.  The returned value is
  available in the variable \texttt{prepare\_res}.
\item[] \texttt{analysis()} is run in parallel processes, one for each
  slice.  It is called after completion of, and actually forked from
  \texttt{prepare()}.  Common input parameters are \texttt{sliceno},
  holding the number of the current process instance, and
  \texttt{prepare\_res}.  The return value for each process becomes
  available in the \texttt{analysis\_res} variable.
\item[] \texttt{synthesis()} is called after the last
  \texttt{analysis()}-process is completed.  It is typically used to
  aggregate parallel results created by \texttt{analysis()} and takes
  both \texttt{prepare\_res} and \texttt{analysis\_res} as optional
  parameters.  The latter is an iterator of the results from the
  parallel processes.
\end{itemize}
Figure~\ref{fig:prepanasyn} shows the execution order from top to
bottom, and the data passed between functions in coloured branches.
\texttt{prepare()} is executed first, and its return value is
available to both the \texttt{analysis()} and \texttt{synthesis()}
functions.  There are \texttt{slices} (a configurable parameter)
number of parallel \texttt{analysis()} processes, and their output is
available to the \texttt{synthesis()} function, which is executed
last.

Return values from any of the three functions may be stored in the
job's directory making them available to other jobs.


\begin{figure}[t]
  \begin{center}
    \input{figures/prepanasyn.pdftex_t}
    \caption{Execution flow and result propagation in a method.}
    \label{fig:prepanasyn}
  \end{center}
\end{figure}



\subsection{Job Parameters}
\label{sec:jobparams}
We've seen how completed jobs can be used as input to new
jobs.  Jobs are one of three kinds of input parameters that
a job can take.  Here the input parameters are summarised:
\begin{itemize}
\item[] \texttt{jobs}, a set, or list, of identifiers to previously executed jobs;
\item[] \texttt{options}, a dictionary of options; and
\item[] \texttt{datasets}, a set, or list, of input \textsl{datasets}.
\end{itemize}
See Figure~\ref{fig:execflow}.  Parameters are entered as global
variables early in the method's source.


\begin{figure}[b]
  \begin{center}
    \input{figures/execflow.pdftex_t}
    \caption{Execution flow of a method.  The method takes optionally
      three kinds of parameters: \texttt{options}, \texttt{jobs},
      and \texttt{datasets}.}
    \label{fig:execflow}
  \end{center}
\end{figure}





\section{A Class Based Programming Model}
The Accelerator is based on an class based paradigm.  Access the the
Accelerator's build in functions and parameters are typically done
through a few \textsl{objects} that are populated by the running
Accelerator.

    \caption{High level view of the Accelerator framework.  See text
      for details.}
    \label{fig:overview}
  \end{center}
\end{figure}

In the figure, the client side, with shell commands and web interface,
is on the left, and the server side, with job databases is to the
right.  The Accelerator executes \textsl{build scripts}, that issue
\textsl{jobs} on the server.  These jobs are stored in the server's
job databases for later retreival.  There are two databases, one for
storing everything related to a job's execution, called the
\textsl{workdir}, and one transaction log used for storing
meta-information about built jobs.  The workdir will contain all
inputs, source code, parameters, and outputs of all executed jobs,
whereas the job log database ensures reproducibility and transparency,
and it will be further discussed in chapter~\ref{chap:urd}.



\section{Jobs}
The basic operation of the Accelerator is to execute small Python
programs called \textsl{methods}.  A method is nothing but a Python
program with one or a few special functions that are used to execute
code sequentially or in parallel and to pass parameters and results.
A method that has completed execution is called a \textsl{job}.

Jobs are stored in \textsl{job directories}.  A dedicated directory in
the workdir will be created for each new job, and this directory will
contain all information regarding the job, such as its input
parameters, stored files, return values, profiling information, Python
interpreter version and path, and more.

The Accelerator has a database that keeps track of all jobs that has
been run.  This avoids unnecessary re-computing in favour of re-using
previously computed results.  Re-using jobs does not only speed up
processing and encourage incremental design, but also makes it
transparent which code and which data that was used for any particular
result, minimising these kind of uncertainties to zero.


\subsection{A Very Simple Job:  ``Hello, World''}
Here's an example of a very simple method.  It does not take any input
parameters and does almost nothing, it will just return the string
``\texttt{hello world}'' and exit.
\begin{python}
def synthesis():
    return "hello world"
\end{python}
Everything inside the \texttt{synthesis()}-function is run once
without any parallelisation.  In order to get this method to execute,
it is called from a \textsl{build script} looking something like this
\begin{python}
def main(urd)
    job = urd.build('hello_world')
    print(job.load())
\end{python}
The \texttt{urd} object made available by the \texttt{main()}-function
contains functions for job building and organisation, and is described
in chapter~\ref{chap:urd} and section~\ref{sec:classes:urd}.  This
build script is then run by a shell command like this
\begin{shell}
ax run
\end{shell}
The \texttt{build()}-call will instruct the server to execute the
\texttt{hello\_world} method.  During its execution, a job directory
will be created that contains everyting associated with the build
process.  When the \texttt{build()}-call is completed, a job object, of type
\texttt{Job}, is returned to the program.  This object provides a
convenient interface to the data in the corresponding job directory,
and contains member functions such as \texttt{.load()}, that is used
in the example to read back the returned value from the job.


\subsection{Jobs Can Only be Run Once}
If the build script is executed again, the \texttt{hello\_world} job
will not be re-built, simply because the Accelerator remembers that
the job has been built in the past, and its associated information is
stored in a job directory.  Instead, the Accelerator immediately
returns a job object representing the previous run.  This means that
from a user's perspective, there is no visible difference between
running a job for the first time or re-using results from an existing
run!  In order to have the method executing again, either the source
code or input parameters has to change.  If there are changes, the
method will be re-executed, and a new job will be created that
reflects these changes.


\subsection{Back to the ``Hello, World'' Example}
Figure~\ref{fig:execflow-hello-world} illustrates the dispatch of the
\texttt{hello\_world} method.  The created job gets the \textsl{jobid}
\texttt{test-0}, and parts of the corresponding job directory
information is shown in green.  (Jobids are job identifiers, that are
named by their corresponding \textsl{workdir} plus an integer counter
value.)  The job directory contains several files, of which the most
important are
\begin{itemize}
\item[] \texttt{setup.json}, containing job meta information;
\item[] \texttt{result.pickle}, containing the returned data; and
\item[] \texttt{method.tar.gz}, containing the method's source code.
\end{itemize}

\begin{figure}[b]
  \begin{center}
    \input{figures/job0.pdftex_t}
    \caption{A simple hello world program, represented as graph and
      work directory.}
    \label{fig:execflow-hello-world}
  \end{center}
\end{figure}

The \texttt{Job} object provides a convenient way to access files and
data stored in this directory.  For example, as we've already seen,
the job's return value can be loaded into a variable using the
\texttt{.load()} function.



\subsection{Workdirs and Sharing Jobs}

Workdirs are used to store jobs.  There can be more than one workdir,
and different workdirs can be used separate jobs into different
physical locations.  The Accelerator can be set up to have any number
of workdirs associated, but only one is used for writing.

If the same workdir is entered into two or more different user's
configuration files, the workdir and its contents will be shared
between the users.  Each Accelerator server will update its knowledge
about the contents of all workdirs before executing a build script, to
make sure that the latest jobs are taken into account.  In addition,
the job log database, as described in chapter~\ref{chap:urd}, is
designed for efficient re-use and sharing of particularly interesting
jobs.


\subsection{Linking Jobs}

Using jobs, complex tasks can be split into several smaller
operations.  Jobs can be connected so that the next job will depend on
the result of a previous job or set of jobs, and so on.

To continue the simple hello world example, assume for a second that
the \texttt{hello\_world}-job is computationally expensive, and that
it returns a result that is to be used as input to further processing.
To keep things simple, this further processing is represented by
printing the result to standard output.  A new method
\texttt{print\_result} is created, and it goes like this
\begin{python}
jobs = ('hello_world_job',)

def synthesis():
    data = jobs.hello_world_job.load() 
    print(data)
\end{python}
This method expects the \texttt{hello\_world\_job} input parameter to
be provided at execution time, and it is accomplished by the following
extended build script
\begin{python}
def main(urd):
    job1 = urd.build('hello_world')
    job2 = urd.build('print_result', hello_world_job=job1)
\end{python}
The \texttt{print\_result} method then loads the result from the
provided job and prints its contents to \texttt{stdout}.  Note that
this method does not return anything.

Figure~\ref{fig:execflow-print-result} illustrates the situation.
(Note the direction of the arrow: the second job, \texttt{test-1} has
\texttt{test-0} as input parameter, but \texttt{test-0} does not know
of any jobs run in the future.  Hence, arrows point to previous jobs.)

\begin{figure}[b]
  \begin{center}
    \input{figures/job0job1.pdftex_t}
    \caption{Job \texttt{test-0}, is used as input to the
      \texttt{print\_result} job.}
    \label{fig:execflow-print-result}
  \end{center}
\end{figure}

The example shows how a complex task may be split into several jobs,
each reading intermediate results from previous jobs.  The Accelerator
will keep track of all job dependencies, so there is no doubt which
jobs that are run when and on which data.  Furthermore, since the
Accelerator remembers if a job has been executed before, it will link
and re-use previous jobs when possible.  This may bring a significant
improvement in execution speed.  Furthermore, a re-used job is a proof
of that the code, input- and output data is unchanged and connected.


\section{Datasets: Storing Data}

The \texttt{dataset} is the Accelerator's default storage type for
small or large quantities of data, designed for parallel processing
and high performance.  Datasets are built on top of jobs, so
\emph{datasets are created by methods and stored in job directories,
  just like any job result.}

Internally, data in a dataset is stored in a row-column format, and is
typically \emph{sliced} into a fixed number of slices to allow
efficient parallel access, see figure~\ref{fig:dataset}. Columns are
accessed independently, so there is no overhead in reading a single or
a set of columns.  Only the data relevant for a task is read from
disk.


\begin{figure}[h!]
  \begin{center}
    \input{figures/dataset_files.pdftex_t}
    \caption{A dataset containing three columns, $A$, $B$, and $C$
      stored using two slices.  Each dotted box corresponds to a file,
      so there are two files for each column, allowing for parallel
      read of the data using two processes.}
    \label{fig:dataset}
  \end{center}
\end{figure}

Furthermore, datasets may be \textsl{hash partitioned}, so that
slice-membership is based on the hash value of a given column.
Partitioning base on, for example, a column containing some ID string
will puy all rows corresponding to any particular ID in a single slice
only.  In many practical applications, hash partitioning makes
parallel processes independent, minimising the need for complicated
merging operations.  This is explained further in
section~\ref{sec:slicing_and_hashing}.



\subsection{Importing Data}

A project typically starts with \textsl{importing} some data from a
file on disk.  The bundled method \texttt{csvimport} is designed to
parse a plethora of ``comma separated values''-file formats and store
the data as a dataset.  See figure~\ref{fig:dataset_csvimport}.
\begin{figure}[b]
  \begin{center}
    \input{figures/import_file1.pdftex_t}
    \caption{Importing \texttt{file0.txt}.}
    \label{fig:dataset_csvimport}
  \end{center}
\end{figure}
The method takes several input options in addition to the mandatory
filename to control the import process.  Here is an example
invocation
\begin{python}
def main(urd):
    jid = urd.build('csvimport', filename='file0.txt')
\end{python}
When executed, the created dataset will be stored in the resulting job
directory, and the name of the dataset will by default be the jobid
plus the string \texttt{default}.  For example, if the
\texttt{csvimport} jobid is \texttt{imp-0}, the dataset will be
referenced by \texttt{imp-0/default}.  In this case, and always when
there is no ambiguity, the jobid alone (\texttt{imp-0}) could be used
too, for simplicity.  In general, a job could contain any number of
datasets, but a single dataset is a common case.




\subsection{Linking Datasets, Chaining}

Just like jobs can be linked to each other, datasets can link to each
other too.  Since datasets are build on top of jobs, this is
straightforward.  Assume the file \texttt{file0.txt} is imported into
dataset \texttt{imp-0/default}, and that there is more data like it
stored in the file \texttt{file1.txt}.  The second file is imported
with a link to the first dataset, see
figure~\ref{fig:dataset_csvimport_chain}.
\begin{figure}[t]
  \begin{center}
    \input{figures/import_file0file1.pdftex_t}
    \caption{Chaining the import of \texttt{file1.txt} to the previous
      import of \texttt{file0.txt}.}
    \label{fig:dataset_csvimport_chain}
  \end{center}
\end{figure}
The \texttt{imp-1} (or \texttt{imp-1/default}) dataset reference can
now be used to access all data imported from \textsl{both} files!

Linking datasets containing related content is called \emph{chaining},
and this is particularly convenient when dealing with data that grows
over time.  A good example is any kind of \emph{log} data, such as
logs of transactions, user interactions, and similar.  Using chaining,
datasets can be with more rows just by linking, which is a lightweight
constant time operation.



\subsection{Adding New Columns to a Dataset}
In the previous section it was shown that datasets can be chained and
thereby grow in number of rows.  A dataset chain is created simply by
linking one dataset to the other, so the overhead is minimal.  In this
section it is shown that by the same principles, it is equally simple
to add new columns to existing datasets.  Adding columns is a common
operation and the Accelerator handles this situation efficiently using
links.

The idea is very simple.  Assume a ``source'' dataset to which one or
more new columns should be added.  A new dataset is created containing
\textsl{only} the new column(s), and while creating it, the
constructor is instructed to link all the source dataset's columns to
the new dataset such that the new dataset appears to contain all
columns from both datasets.  (Note that this linking is similar to but
different from chaining.)

Accessing the new dataset will transparently access all the columns in
both the new and the source dataset in parallel, making it
indistinguishable from a single dataset.  See
Figure~\ref{fig:dep_dataset_append_column}.

\begin{figure}[b]
  \begin{center}
    \input{figures/dataset_append_column.pdftex_t}
    \caption{Adding one new column to the source dataset.}
    \label{fig:dep_dataset_append_column}
  \end{center}
\end{figure}

A common case is to compute new columns based on existing ones.  In
this case, values are written to the new columns in the new dataset
while reading from the iterator iterating over the existing columns in
the source dataset.  This will be discussed in detail in
section~\ref{sec:appending_new_columns}



\subsection{Multiple Datasets in a Job}

Typically, a method creates a single dataset in the job directory, but
there is no limit to how many datasets that could be created and
stored in a single job directory.  This leads to some interesting
applications.

One application for keeping multiple datasets in a job is when data is
split into subsets based on some condition.  This could, for example,
be when a dataset is split into a training set and a test set.  One
way to achieve this using the Accelerator is by creating a Boolean
column that tells if the current row is train or test data, followed
by a job that splits the dataset in two based on the value on that
column.  See Figure~\ref{fig:dep_dataset_csvimport_chain}.

\begin{figure}[h!]
  \hspace{1cm}
  \input{figures/filter_dataset.pdftex_t}
  \caption{\texttt{job-1} separates the dataset
    \texttt{job-0/default} into two new datasets, named
    \texttt{job-1/train} and \texttt{job-1/test}.}
  \label{fig:dep_dataset_csvimport_chain}
\end{figure}

In the setup of figure~\ref{fig:dep_dataset_csvimport_chain} we have
full tracking from either \texttt{train} or \texttt{test} datasets.
If we want to know the source of one of these sets, we just follow the
links back to the previous jobs until we reach the source job.  In the
figure, \texttt{job-0} may for example be a \texttt{csvimport} job,
and will therefore contain the name of the input file in its
parameters.  Thus, it is straightforward to link any data to its
source.

Splitting a dataset into parts creates ``physical'' isolation while
still keeping all the data at the same place.  No data is lost in the
process, and this is good for transparency reasons.  For example, a
following method may iterate over \textsl{both} datasets in
\texttt{job-1} and by that read the complete dataset.



\subsection{Parallel Dataset Access and Hash Partitioning}
As shown earlier in this chapter, data in datasets is stored in
multiple files for two reasons.  The first reason is that we can read
only the columns that we need, without overhead, and the second reason
is that it allows fast parallel reads.  The parameter \texttt{slices}
determines how many slices that the dataset should be partitioned
into, and it also sets the number of parallel process that is used for
processing the dataset.  There is always one process for each slice of
the dataset, and each process operates on a unique part of the
dataset.

Datasets can be partitioned, or sliced, in different ways.  One
obvious way is to use round robin, where each consecutive data row is
written to the next slice, modulo the number of slices.  This leads to
``well balanced'' datasets with approximately equal number of rows per
slice.  Another alternative to slicing is to slice based on the hash
value of a particular column's values.  Using this method, all rows
with the same value in the hash column end up in the same slice.  This
is efficient for many parallel processing tasks, and will be described
in detail later on.




\subsection{Dataset Column Types}

There are large a number of useful types available for dataset
columns.  They include \textsl{floating} and \textsl{integer point
  numbers}, \textsl{Booleans}, \textsl{timestamps}, several
\textsl{string types} (handling all kinds of string encodings), and
\textsl{json} as well as \textsl{pickle} types for storing arbitrary
data collections.  Most of these types come with advanced parsers,
making importing data from text files straightforward with
deterministic handling of errors, overflows, and so on.



\subsection{Dataset Attributes}
The dataset has a number of attributes associated with it, such as
shape, number of rows, column names and types, and more.
An attribute is accessed like this
\begin{python}
datasets = ('source',)
def synthesis():
    print(datasets.source.shape)
    print(datasets.source.columns)
\end{python}
and so on.


%%%%%%%%%%%%%%%%%%%%%%%%%%%%%%%%%%%%%%%%%%%%%%%%%%%%%%%%%%%%%%%%%%%%%%%%%%%%%%%%

\section{Iterators: Working with Data}

Data in a dataset is typically accessed using an \emph{iterator} that
reads and streams one dataset slice at a time to a CPU core.  The
parallel processing capabilities of the Accelerator makes it possible
to dispatch a set of parallel iterators, one for each slice, in order
to have efficient parallel processing of the whole dataset.

This section shows how iterators are used for reading data, how to
take advantage of slicing to have parallel processing, and how to
efficiently create new datasets.


\subsection{Iterator Basics}

Basic iterator functionality is introduced here using an example.
Assume a dataset that has a column containing movie titles named
\texttt{movie}, and the problem is to extract the ten most frequently
occuring movies.  Here's a complete method solving this problem
\begin{python}
from collections import Counter
datasets = ('source',)

def synthesis():
    c = Counter(datasets.source.iterate(None, 'movie'))
    top10 = c.most_common(10)
    print(top10)
    return top10
\end{python}
This will print the ten most common movie titles and their
corresponding counts in the \texttt{source} dataset.  The code will
run on a single CPU core, because we use the single-process
\texttt{synthesis()} function, which is called and executed only once.
The \texttt{dataset.iterate()} (class-)method therefore has to read
through all slices, one at a time, in a serial fashion, and this is
reflected by the first argument to the iterator being \pyNone.  The
method also returns the variable \texttt{top10} so that it can be used
by other methods.



\subsection{Parallel Execution}
It is easy to write parallel programs using the Accelerator.  The fact
that data in a dataset is sliced into disjoint sets and files makes
parallel processing of data straightforward.  Here's a slightly
modified version of the program from the previous section that will
now execute in parallel
\begin{python}
def analysis(sliceno):
    return Counter(datasets.source.iterate(sliceno, 'movie'))

def synthesis(analysis_res)
    c = analysis_res.merge_auto()
    top10 = c.most_common(10)
    return top10
\end{python}
For larger datasets, this parallel version of the movie title counter
will run much faster.  Here, \texttt{.iterate()} is moved inside the
\texttt{analysis()} function.  This function is forked once for each
slice, and the argument \texttt{sliceno} will contain an integer
between zero and the number of slices minus one.  The returned value
from the analysis functions will be available as input to the
synthesis function in the \texttt{analysis\_res} Python iterable.  It
is possible to merge the results explicitly, but the this iterable
comes with a rather magic method \texttt{merge\_auto()}, that merges
the results from all slices into one based on the data type.  It can
for example merge \texttt{Counter}s, \texttt{set}s, and composed types
like \texttt{set}s of \texttt{Counter}s, and so on.


\subsection{Iterating over Several Columns}
Since each column is stored independently in a dataset, there is no
overhead from reading a subset of a dataset's columns.  In the
previous section we've seen how to iterate over a single column using
\texttt{.iterate()}.  Iterating over more columns is straightforward
by feeding a list of column names to \texttt{.iterate()}, like in this
example
\begin{python}
from collections import defaultdict
datasets = ('source',)

def analysis(sliceno):
    user2movieset = defaultdict(set)
    for user, movie in datasets.source.iterate(sliceno, ('user', 'movie')):
        user2movieset[user].add(movie)
    return user2movieset
\end{python}
This example creates a lookup dictionary from users to sets of movies.
Note that in this case, we would like to have the dataset hashed on
the \texttt{user} column, so that each user appears in exactly one slice.
This will make later merging (if necessary) much easier.

It is also possible to iterate over all columns by specifying an empty
list of columns or by using the value \pyNone.
\begin{python}
...
def analysis(sliceno):
    for columns in datasets.source.iterate(sliceno, None):
        ...
\end{python}
Here, \texttt{columns} will be a list of values, one for each column
in the dataset, in sorted column name order.


\subsection{Iterating over Dataset Chains}

The \texttt{iterate} function is used to iterate over a single
dataset.  There is a corresponding function, \texttt{iterate\_chain},
that is used for iterating over chains of datasets.  This function
takes a number of arguments, such as
\begin{itemize}
\item[] \texttt{length}, i.e.\ the number of datasets to iterate over.
  By default, it will iterate over all datasets in the chain.
\item[] \texttt{callbacks}, functions that can be called before and/or
  after each dataset in a chain.  Very useful for aggregating data
  between datasets.
\item[] \texttt{stop\_id} which stops iterating at a certain dataset.
  This dataset could be from \textsl{another} job's parameters, so we
  can for example iterate exactly over all new datasets not covered by
  a previous job.
\item[] \texttt{range}, which allows for iterating over a range of
  data.
\end{itemize}
The \texttt{range} options is based on the max/min values stored for
each column in the dataset.  Assuming that the chain is sorted, one
can for example set
\begin{python}
range={timestamp, ('2016-01-01', '2016-01-31')}
\end{python}
in order to get rows within the specified range only.  Using
\texttt{range=} is quite costly, since it requires each row in the
dataset chain with dates within the range to be checked against the
range criterion.  Therefore, there is a \texttt{sloppy} version that
iterates over complete datasets in the chain that contains at least
one row with a date within the range.  This runs at full speed, and is
useful, for example, to very quickly produce histograms or plots of
subsets of a huge dataset.



\subsection{Job Execution Flow and Result Passing}

Execution of code in a method is either parallel or serial depending
on which function is used to encapsulate it.  There are three
functions in a method that are called from the Accelerator when a
method is running, and they are \texttt{prepare()},
\texttt{analysis()}, and \texttt{synthesis()}.  All three may exist in
the same method, and at least one is required.  When the method
executes, they are called one after the other.
\begin{itemize}
\item[] \texttt{prepare()} is executed first.  The returned value is
  available in the variable \texttt{prepare\_res}.
\item[] \texttt{analysis()} is run in parallel processes, one for each
  slice.  It is called after completion of, and actually forked from
  \texttt{prepare()}.  Common input parameters are \texttt{sliceno},
  holding the number of the current process instance, and
  \texttt{prepare\_res}.  The return value for each process becomes
  available in the \texttt{analysis\_res} variable.
\item[] \texttt{synthesis()} is called after the last
  \texttt{analysis()}-process is completed.  It is typically used to
  aggregate parallel results created by \texttt{analysis()} and takes
  both \texttt{prepare\_res} and \texttt{analysis\_res} as optional
  parameters.  The latter is an iterator of the results from the
  parallel processes.
\end{itemize}
Figure~\ref{fig:prepanasyn} shows the execution order from top to
bottom, and the data passed between functions in coloured branches.
\texttt{prepare()} is executed first, and its return value is
available to both the \texttt{analysis()} and \texttt{synthesis()}
functions.  There are \texttt{slices} (a configurable parameter)
number of parallel \texttt{analysis()} processes, and their output is
available to the \texttt{synthesis()} function, which is executed
last.

Return values from any of the three functions may be stored in the
job's directory making them available to other jobs.


\begin{figure}[t]
  \begin{center}
    \input{figures/prepanasyn.pdftex_t}
    \caption{Execution flow and result propagation in a method.}
    \label{fig:prepanasyn}
  \end{center}
\end{figure}



\subsection{Job Parameters}
\label{sec:jobparams}
We've seen how completed jobs can be used as input to new
jobs.  Jobs are one of three kinds of input parameters that
a job can take.  Here the input parameters are summarised:
\begin{itemize}
\item[] \texttt{jobs}, a set, or list, of identifiers to previously executed jobs;
\item[] \texttt{options}, a dictionary of options; and
\item[] \texttt{datasets}, a set, or list, of input \textsl{datasets}.
\end{itemize}
See Figure~\ref{fig:execflow}.  Parameters are entered as global
variables early in the method's source.


\begin{figure}[b]
  \begin{center}
    \input{figures/execflow.pdftex_t}
    \caption{Execution flow of a method.  The method takes optionally
      three kinds of parameters: \texttt{options}, \texttt{jobs},
      and \texttt{datasets}.}
    \label{fig:execflow}
  \end{center}
\end{figure}





\section{A Class Based Programming Model}
The Accelerator is based on an class based paradigm.  Access the the
Accelerator's build in functions and parameters are typically done
through a few \textsl{objects} that are populated by the running
Accelerator.

    \caption{High level view of the Accelerator framework.  See text
      for details.}
    \label{fig:overview}
  \end{center}
\end{figure}

In the figure, the client side, with shell commands and web interface,
is on the left, and the server side, with job databases is to the
right.  The Accelerator executes \textsl{build scripts}, that issue
\textsl{jobs} on the server.  These jobs are stored in the server's
job databases for later retreival.  There are two databases, one for
storing everything related to a job's execution, called the
\textsl{workdir}, and one transaction log used for storing
meta-information about built jobs.  The workdir will contain all
inputs, source code, parameters, and outputs of all executed jobs,
whereas the job log database ensures reproducibility and transparency,
and it will be further discussed in chapter~\ref{chap:urd}.



\section{Jobs}
The basic operation of the Accelerator is to execute small Python
programs called \textsl{methods}.  A method is nothing but a Python
program with one or a few special functions that are used to execute
code sequentially or in parallel and to pass parameters and results.
A method that has completed execution is called a \textsl{job}.

Jobs are stored in \textsl{job directories}.  A dedicated directory in
the workdir will be created for each new job, and this directory will
contain all information regarding the job, such as its input
parameters, stored files, return values, profiling information, Python
interpreter version and path, and more.

The Accelerator has a database that keeps track of all jobs that has
been run.  This avoids unnecessary re-computing in favour of re-using
previously computed results.  Re-using jobs does not only speed up
processing and encourage incremental design, but also makes it
transparent which code and which data that was used for any particular
result, minimising these kind of uncertainties to zero.


\subsection{A Very Simple Job:  ``Hello, World''}
Here's an example of a very simple method.  It does not take any input
parameters and does almost nothing, it will just return the string
``\texttt{hello world}'' and exit.
\begin{python}
def synthesis():
    return "hello world"
\end{python}
Everything inside the \texttt{synthesis()}-function is run once
without any parallelisation.  In order to get this method to execute,
it is called from a \textsl{build script} looking something like this
\begin{python}
def main(urd)
    job = urd.build('hello_world')
    print(job.load())
\end{python}
The \texttt{urd} object made available by the \texttt{main()}-function
contains functions for job building and organisation, and is described
in chapter~\ref{chap:urd} and section~\ref{sec:classes:urd}.  This
build script is then run by a shell command like this
\begin{shell}
ax run
\end{shell}
The \texttt{build()}-call will instruct the server to execute the
\texttt{hello\_world} method.  During its execution, a job directory
will be created that contains everyting associated with the build
process.  When the \texttt{build()}-call is completed, a job object, of type
\texttt{Job}, is returned to the program.  This object provides a
convenient interface to the data in the corresponding job directory,
and contains member functions such as \texttt{.load()}, that is used
in the example to read back the returned value from the job.


\subsection{Jobs Can Only be Run Once}
If the build script is executed again, the \texttt{hello\_world} job
will not be re-built, simply because the Accelerator remembers that
the job has been built in the past, and its associated information is
stored in a job directory.  Instead, the Accelerator immediately
returns a job object representing the previous run.  This means that
from a user's perspective, there is no visible difference between
running a job for the first time or re-using results from an existing
run!  In order to have the method executing again, either the source
code or input parameters has to change.  If there are changes, the
method will be re-executed, and a new job will be created that
reflects these changes.


\subsection{Back to the ``Hello, World'' Example}
Figure~\ref{fig:execflow-hello-world} illustrates the dispatch of the
\texttt{hello\_world} method.  The created job gets the \textsl{jobid}
\texttt{test-0}, and parts of the corresponding job directory
information is shown in green.  (Jobids are job identifiers, that are
named by their corresponding \textsl{workdir} plus an integer counter
value.)  The job directory contains several files, of which the most
important are
\begin{itemize}
\item[] \texttt{setup.json}, containing job meta information;
\item[] \texttt{result.pickle}, containing the returned data; and
\item[] \texttt{method.tar.gz}, containing the method's source code.
\end{itemize}

\begin{figure}[b]
  \begin{center}
    \input{figures/job0.pdftex_t}
    \caption{A simple hello world program, represented as graph and
      work directory.}
    \label{fig:execflow-hello-world}
  \end{center}
\end{figure}

The \texttt{Job} object provides a convenient way to access files and
data stored in this directory.  For example, as we've already seen,
the job's return value can be loaded into a variable using the
\texttt{.load()} function.



\subsection{Workdirs and Sharing Jobs}

Workdirs are used to store jobs.  There can be more than one workdir,
and different workdirs can be used separate jobs into different
physical locations.  The Accelerator can be set up to have any number
of workdirs associated, but only one is used for writing.

If the same workdir is entered into two or more different user's
configuration files, the workdir and its contents will be shared
between the users.  Each Accelerator server will update its knowledge
about the contents of all workdirs before executing a build script, to
make sure that the latest jobs are taken into account.  In addition,
the job log database, as described in chapter~\ref{chap:urd}, is
designed for efficient re-use and sharing of particularly interesting
jobs.


\subsection{Linking Jobs}

Using jobs, complex tasks can be split into several smaller
operations.  Jobs can be connected so that the next job will depend on
the result of a previous job or set of jobs, and so on.

To continue the simple hello world example, assume for a second that
the \texttt{hello\_world}-job is computationally expensive, and that
it returns a result that is to be used as input to further processing.
To keep things simple, this further processing is represented by
printing the result to standard output.  A new method
\texttt{print\_result} is created, and it goes like this
\begin{python}
jobs = ('hello_world_job',)

def synthesis():
    data = jobs.hello_world_job.load() 
    print(data)
\end{python}
This method expects the \texttt{hello\_world\_job} input parameter to
be provided at execution time, and it is accomplished by the following
extended build script
\begin{python}
def main(urd):
    job1 = urd.build('hello_world')
    job2 = urd.build('print_result', hello_world_job=job1)
\end{python}
The \texttt{print\_result} method then loads the result from the
provided job and prints its contents to \texttt{stdout}.  Note that
this method does not return anything.

Figure~\ref{fig:execflow-print-result} illustrates the situation.
(Note the direction of the arrow: the second job, \texttt{test-1} has
\texttt{test-0} as input parameter, but \texttt{test-0} does not know
of any jobs run in the future.  Hence, arrows point to previous jobs.)

\begin{figure}[b]
  \begin{center}
    \input{figures/job0job1.pdftex_t}
    \caption{Job \texttt{test-0}, is used as input to the
      \texttt{print\_result} job.}
    \label{fig:execflow-print-result}
  \end{center}
\end{figure}

The example shows how a complex task may be split into several jobs,
each reading intermediate results from previous jobs.  The Accelerator
will keep track of all job dependencies, so there is no doubt which
jobs that are run when and on which data.  Furthermore, since the
Accelerator remembers if a job has been executed before, it will link
and re-use previous jobs when possible.  This may bring a significant
improvement in execution speed.  Furthermore, a re-used job is a proof
of that the code, input- and output data is unchanged and connected.


\section{Datasets: Storing Data}

The \texttt{dataset} is the Accelerator's default storage type for
small or large quantities of data, designed for parallel processing
and high performance.  Datasets are built on top of jobs, so
\emph{datasets are created by methods and stored in job directories,
  just like any job result.}

Internally, data in a dataset is stored in a row-column format, and is
typically \emph{sliced} into a fixed number of slices to allow
efficient parallel access, see figure~\ref{fig:dataset}. Columns are
accessed independently, so there is no overhead in reading a single or
a set of columns.  Only the data relevant for a task is read from
disk.


\begin{figure}[h!]
  \begin{center}
    \input{figures/dataset_files.pdftex_t}
    \caption{A dataset containing three columns, $A$, $B$, and $C$
      stored using two slices.  Each dotted box corresponds to a file,
      so there are two files for each column, allowing for parallel
      read of the data using two processes.}
    \label{fig:dataset}
  \end{center}
\end{figure}

Furthermore, datasets may be \textsl{hash partitioned}, so that
slice-membership is based on the hash value of a given column.
Partitioning base on, for example, a column containing some ID string
will puy all rows corresponding to any particular ID in a single slice
only.  In many practical applications, hash partitioning makes
parallel processes independent, minimising the need for complicated
merging operations.  This is explained further in
section~\ref{sec:slicing_and_hashing}.



\subsection{Importing Data}

A project typically starts with \textsl{importing} some data from a
file on disk.  The bundled method \texttt{csvimport} is designed to
parse a plethora of ``comma separated values''-file formats and store
the data as a dataset.  See figure~\ref{fig:dataset_csvimport}.
\begin{figure}[b]
  \begin{center}
    \input{figures/import_file1.pdftex_t}
    \caption{Importing \texttt{file0.txt}.}
    \label{fig:dataset_csvimport}
  \end{center}
\end{figure}
The method takes several input options in addition to the mandatory
filename to control the import process.  Here is an example
invocation
\begin{python}
def main(urd):
    jid = urd.build('csvimport', filename='file0.txt')
\end{python}
When executed, the created dataset will be stored in the resulting job
directory, and the name of the dataset will by default be the jobid
plus the string \texttt{default}.  For example, if the
\texttt{csvimport} jobid is \texttt{imp-0}, the dataset will be
referenced by \texttt{imp-0/default}.  In this case, and always when
there is no ambiguity, the jobid alone (\texttt{imp-0}) could be used
too, for simplicity.  In general, a job could contain any number of
datasets, but a single dataset is a common case.




\subsection{Linking Datasets, Chaining}

Just like jobs can be linked to each other, datasets can link to each
other too.  Since datasets are build on top of jobs, this is
straightforward.  Assume the file \texttt{file0.txt} is imported into
dataset \texttt{imp-0/default}, and that there is more data like it
stored in the file \texttt{file1.txt}.  The second file is imported
with a link to the first dataset, see
figure~\ref{fig:dataset_csvimport_chain}.
\begin{figure}[t]
  \begin{center}
    \input{figures/import_file0file1.pdftex_t}
    \caption{Chaining the import of \texttt{file1.txt} to the previous
      import of \texttt{file0.txt}.}
    \label{fig:dataset_csvimport_chain}
  \end{center}
\end{figure}
The \texttt{imp-1} (or \texttt{imp-1/default}) dataset reference can
now be used to access all data imported from \textsl{both} files!

Linking datasets containing related content is called \emph{chaining},
and this is particularly convenient when dealing with data that grows
over time.  A good example is any kind of \emph{log} data, such as
logs of transactions, user interactions, and similar.  Using chaining,
datasets can be with more rows just by linking, which is a lightweight
constant time operation.



\subsection{Adding New Columns to a Dataset}
In the previous section it was shown that datasets can be chained and
thereby grow in number of rows.  A dataset chain is created simply by
linking one dataset to the other, so the overhead is minimal.  In this
section it is shown that by the same principles, it is equally simple
to add new columns to existing datasets.  Adding columns is a common
operation and the Accelerator handles this situation efficiently using
links.

The idea is very simple.  Assume a ``source'' dataset to which one or
more new columns should be added.  A new dataset is created containing
\textsl{only} the new column(s), and while creating it, the
constructor is instructed to link all the source dataset's columns to
the new dataset such that the new dataset appears to contain all
columns from both datasets.  (Note that this linking is similar to but
different from chaining.)

Accessing the new dataset will transparently access all the columns in
both the new and the source dataset in parallel, making it
indistinguishable from a single dataset.  See
Figure~\ref{fig:dep_dataset_append_column}.

\begin{figure}[b]
  \begin{center}
    \input{figures/dataset_append_column.pdftex_t}
    \caption{Adding one new column to the source dataset.}
    \label{fig:dep_dataset_append_column}
  \end{center}
\end{figure}

A common case is to compute new columns based on existing ones.  In
this case, values are written to the new columns in the new dataset
while reading from the iterator iterating over the existing columns in
the source dataset.  This will be discussed in detail in
section~\ref{sec:appending_new_columns}



\subsection{Multiple Datasets in a Job}

Typically, a method creates a single dataset in the job directory, but
there is no limit to how many datasets that could be created and
stored in a single job directory.  This leads to some interesting
applications.

One application for keeping multiple datasets in a job is when data is
split into subsets based on some condition.  This could, for example,
be when a dataset is split into a training set and a test set.  One
way to achieve this using the Accelerator is by creating a Boolean
column that tells if the current row is train or test data, followed
by a job that splits the dataset in two based on the value on that
column.  See Figure~\ref{fig:dep_dataset_csvimport_chain}.

\begin{figure}[h!]
  \hspace{1cm}
  \input{figures/filter_dataset.pdftex_t}
  \caption{\texttt{job-1} separates the dataset
    \texttt{job-0/default} into two new datasets, named
    \texttt{job-1/train} and \texttt{job-1/test}.}
  \label{fig:dep_dataset_csvimport_chain}
\end{figure}

In the setup of figure~\ref{fig:dep_dataset_csvimport_chain} we have
full tracking from either \texttt{train} or \texttt{test} datasets.
If we want to know the source of one of these sets, we just follow the
links back to the previous jobs until we reach the source job.  In the
figure, \texttt{job-0} may for example be a \texttt{csvimport} job,
and will therefore contain the name of the input file in its
parameters.  Thus, it is straightforward to link any data to its
source.

Splitting a dataset into parts creates ``physical'' isolation while
still keeping all the data at the same place.  No data is lost in the
process, and this is good for transparency reasons.  For example, a
following method may iterate over \textsl{both} datasets in
\texttt{job-1} and by that read the complete dataset.



\subsection{Parallel Dataset Access and Hash Partitioning}
As shown earlier in this chapter, data in datasets is stored in
multiple files for two reasons.  The first reason is that we can read
only the columns that we need, without overhead, and the second reason
is that it allows fast parallel reads.  The parameter \texttt{slices}
determines how many slices that the dataset should be partitioned
into, and it also sets the number of parallel process that is used for
processing the dataset.  There is always one process for each slice of
the dataset, and each process operates on a unique part of the
dataset.

Datasets can be partitioned, or sliced, in different ways.  One
obvious way is to use round robin, where each consecutive data row is
written to the next slice, modulo the number of slices.  This leads to
``well balanced'' datasets with approximately equal number of rows per
slice.  Another alternative to slicing is to slice based on the hash
value of a particular column's values.  Using this method, all rows
with the same value in the hash column end up in the same slice.  This
is efficient for many parallel processing tasks, and will be described
in detail later on.




\subsection{Dataset Column Types}

There are large a number of useful types available for dataset
columns.  They include \textsl{floating} and \textsl{integer point
  numbers}, \textsl{Booleans}, \textsl{timestamps}, several
\textsl{string types} (handling all kinds of string encodings), and
\textsl{json} as well as \textsl{pickle} types for storing arbitrary
data collections.  Most of these types come with advanced parsers,
making importing data from text files straightforward with
deterministic handling of errors, overflows, and so on.



\subsection{Dataset Attributes}
The dataset has a number of attributes associated with it, such as
shape, number of rows, column names and types, and more.
An attribute is accessed like this
\begin{python}
datasets = ('source',)
def synthesis():
    print(datasets.source.shape)
    print(datasets.source.columns)
\end{python}
and so on.


%%%%%%%%%%%%%%%%%%%%%%%%%%%%%%%%%%%%%%%%%%%%%%%%%%%%%%%%%%%%%%%%%%%%%%%%%%%%%%%%

\section{Iterators: Working with Data}

Data in a dataset is typically accessed using an \emph{iterator} that
reads and streams one dataset slice at a time to a CPU core.  The
parallel processing capabilities of the Accelerator makes it possible
to dispatch a set of parallel iterators, one for each slice, in order
to have efficient parallel processing of the whole dataset.

This section shows how iterators are used for reading data, how to
take advantage of slicing to have parallel processing, and how to
efficiently create new datasets.


\subsection{Iterator Basics}

Basic iterator functionality is introduced here using an example.
Assume a dataset that has a column containing movie titles named
\texttt{movie}, and the problem is to extract the ten most frequently
occuring movies.  Here's a complete method solving this problem
\begin{python}
from collections import Counter
datasets = ('source',)

def synthesis():
    c = Counter(datasets.source.iterate(None, 'movie'))
    top10 = c.most_common(10)
    print(top10)
    return top10
\end{python}
This will print the ten most common movie titles and their
corresponding counts in the \texttt{source} dataset.  The code will
run on a single CPU core, because we use the single-process
\texttt{synthesis()} function, which is called and executed only once.
The \texttt{dataset.iterate()} (class-)method therefore has to read
through all slices, one at a time, in a serial fashion, and this is
reflected by the first argument to the iterator being \pyNone.  The
method also returns the variable \texttt{top10} so that it can be used
by other methods.



\subsection{Parallel Execution}
It is easy to write parallel programs using the Accelerator.  The fact
that data in a dataset is sliced into disjoint sets and files makes
parallel processing of data straightforward.  Here's a slightly
modified version of the program from the previous section that will
now execute in parallel
\begin{python}
def analysis(sliceno):
    return Counter(datasets.source.iterate(sliceno, 'movie'))

def synthesis(analysis_res)
    c = analysis_res.merge_auto()
    top10 = c.most_common(10)
    return top10
\end{python}
For larger datasets, this parallel version of the movie title counter
will run much faster.  Here, \texttt{.iterate()} is moved inside the
\texttt{analysis()} function.  This function is forked once for each
slice, and the argument \texttt{sliceno} will contain an integer
between zero and the number of slices minus one.  The returned value
from the analysis functions will be available as input to the
synthesis function in the \texttt{analysis\_res} Python iterable.  It
is possible to merge the results explicitly, but the this iterable
comes with a rather magic method \texttt{merge\_auto()}, that merges
the results from all slices into one based on the data type.  It can
for example merge \texttt{Counter}s, \texttt{set}s, and composed types
like \texttt{set}s of \texttt{Counter}s, and so on.


\subsection{Iterating over Several Columns}
Since each column is stored independently in a dataset, there is no
overhead from reading a subset of a dataset's columns.  In the
previous section we've seen how to iterate over a single column using
\texttt{.iterate()}.  Iterating over more columns is straightforward
by feeding a list of column names to \texttt{.iterate()}, like in this
example
\begin{python}
from collections import defaultdict
datasets = ('source',)

def analysis(sliceno):
    user2movieset = defaultdict(set)
    for user, movie in datasets.source.iterate(sliceno, ('user', 'movie')):
        user2movieset[user].add(movie)
    return user2movieset
\end{python}
This example creates a lookup dictionary from users to sets of movies.
Note that in this case, we would like to have the dataset hashed on
the \texttt{user} column, so that each user appears in exactly one slice.
This will make later merging (if necessary) much easier.

It is also possible to iterate over all columns by specifying an empty
list of columns or by using the value \pyNone.
\begin{python}
...
def analysis(sliceno):
    for columns in datasets.source.iterate(sliceno, None):
        ...
\end{python}
Here, \texttt{columns} will be a list of values, one for each column
in the dataset, in sorted column name order.


\subsection{Iterating over Dataset Chains}

The \texttt{iterate} function is used to iterate over a single
dataset.  There is a corresponding function, \texttt{iterate\_chain},
that is used for iterating over chains of datasets.  This function
takes a number of arguments, such as
\begin{itemize}
\item[] \texttt{length}, i.e.\ the number of datasets to iterate over.
  By default, it will iterate over all datasets in the chain.
\item[] \texttt{callbacks}, functions that can be called before and/or
  after each dataset in a chain.  Very useful for aggregating data
  between datasets.
\item[] \texttt{stop\_id} which stops iterating at a certain dataset.
  This dataset could be from \textsl{another} job's parameters, so we
  can for example iterate exactly over all new datasets not covered by
  a previous job.
\item[] \texttt{range}, which allows for iterating over a range of
  data.
\end{itemize}
The \texttt{range} options is based on the max/min values stored for
each column in the dataset.  Assuming that the chain is sorted, one
can for example set
\begin{python}
range={timestamp, ('2016-01-01', '2016-01-31')}
\end{python}
in order to get rows within the specified range only.  Using
\texttt{range=} is quite costly, since it requires each row in the
dataset chain with dates within the range to be checked against the
range criterion.  Therefore, there is a \texttt{sloppy} version that
iterates over complete datasets in the chain that contains at least
one row with a date within the range.  This runs at full speed, and is
useful, for example, to very quickly produce histograms or plots of
subsets of a huge dataset.



\subsection{Job Execution Flow and Result Passing}

Execution of code in a method is either parallel or serial depending
on which function is used to encapsulate it.  There are three
functions in a method that are called from the Accelerator when a
method is running, and they are \texttt{prepare()},
\texttt{analysis()}, and \texttt{synthesis()}.  All three may exist in
the same method, and at least one is required.  When the method
executes, they are called one after the other.
\begin{itemize}
\item[] \texttt{prepare()} is executed first.  The returned value is
  available in the variable \texttt{prepare\_res}.
\item[] \texttt{analysis()} is run in parallel processes, one for each
  slice.  It is called after completion of, and actually forked from
  \texttt{prepare()}.  Common input parameters are \texttt{sliceno},
  holding the number of the current process instance, and
  \texttt{prepare\_res}.  The return value for each process becomes
  available in the \texttt{analysis\_res} variable.
\item[] \texttt{synthesis()} is called after the last
  \texttt{analysis()}-process is completed.  It is typically used to
  aggregate parallel results created by \texttt{analysis()} and takes
  both \texttt{prepare\_res} and \texttt{analysis\_res} as optional
  parameters.  The latter is an iterator of the results from the
  parallel processes.
\end{itemize}
Figure~\ref{fig:prepanasyn} shows the execution order from top to
bottom, and the data passed between functions in coloured branches.
\texttt{prepare()} is executed first, and its return value is
available to both the \texttt{analysis()} and \texttt{synthesis()}
functions.  There are \texttt{slices} (a configurable parameter)
number of parallel \texttt{analysis()} processes, and their output is
available to the \texttt{synthesis()} function, which is executed
last.

Return values from any of the three functions may be stored in the
job's directory making them available to other jobs.


\begin{figure}[t]
  \begin{center}
    \input{figures/prepanasyn.pdftex_t}
    \caption{Execution flow and result propagation in a method.}
    \label{fig:prepanasyn}
  \end{center}
\end{figure}



\subsection{Job Parameters}
\label{sec:jobparams}
We've seen how completed jobs can be used as input to new
jobs.  Jobs are one of three kinds of input parameters that
a job can take.  Here the input parameters are summarised:
\begin{itemize}
\item[] \texttt{jobs}, a set, or list, of identifiers to previously executed jobs;
\item[] \texttt{options}, a dictionary of options; and
\item[] \texttt{datasets}, a set, or list, of input \textsl{datasets}.
\end{itemize}
See Figure~\ref{fig:execflow}.  Parameters are entered as global
variables early in the method's source.


\begin{figure}[b]
  \begin{center}
    \input{figures/execflow.pdftex_t}
    \caption{Execution flow of a method.  The method takes optionally
      three kinds of parameters: \texttt{options}, \texttt{jobs},
      and \texttt{datasets}.}
    \label{fig:execflow}
  \end{center}
\end{figure}





\section{A Class Based Programming Model}
The Accelerator is based on an class based paradigm.  Access the the
Accelerator's build in functions and parameters are typically done
through a few \textsl{objects} that are populated by the running
Accelerator.

    \caption{High level view of the Accelerator framework.  See text
      for details.}
    \label{fig:overview}
  \end{center}
\end{figure}

On the left side there is the \texttt{run} program.  To the right, there
are two servers, called \texttt{daemon} and \texttt{urd}.  The
\texttt{run} program runs what is called \texttt{build scripts}, that
execute jobs on the \texttt{daemon} server.  This server will load and
store information and results for all jobs executed using the
\textsl{workdirs} file system based database.

In parallel, all jobs covered by a build script may be stored by the
\texttt{urd} server into the \textsl{job logs} file system database.
\texttt{urd} is also responsible for finding collections, or lists, of
related previously executed jobs.  The \texttt{urd} server ensures
reproducibility and transparency, and it will be further discussed in
chapter~\ref{chap:urd}.



\section{Jobs}
The basic operation of the Accelerator is to execute small Python
programs called \textsl{methods}.  In a method, a few special
functions are used to execute code sequentially or in parallel and to
pass parameters and results.  A method that has completed execution is
called a \textsl{job}.

Jobs are stored in \textsl{job directories}.  A dedicated directory
will be created for each new job, and the directory will contain all
information regarding the job, such as its input parameters, stored
files, return values, profiling information, and more.

The Accelerator has a database that keeps track of all jobs that have
been run.  This is very useful for avoiding unnecessary re-computing
and instead rely on reusing previously computed results.  This does
not only speed up processing and encourage incremental design, but
also makes it transparent which code and which data was used for any
particular result, thus minimising uncertainty.


\subsection{A Very Simple Job:  ``Hello, World''}
The following example method is very simple.  It does not take any
input parameters and does almost nothing, it will just return the
string ``\texttt{hello world}'' and exit.
\begin{python}
def synthesis():
    return "hello world"
\end{python}
In order to get the method to execute, it is called from a
\textsl{build script} looking something like this
\begin{python}
def main(urd)
    job = urd.build('hello_world')
    print(job.load())
\end{python}
The \texttt{urd} object contains functions for job building and
organisation, and is described in chapter~\ref{chap:urd} and
section~\ref{sec:classes:urd}.  Remember that during the job build
process, a job directory is created that will contain everything
associated with the build.

When execution is completed, a job object, of type \texttt{Job}, is
returned to the user.  This object provides a convenient interface to
the data in the corresponding job directory, and contains member
functions such as \texttt{.load()}, that is used in the example to
read back the returned value from the job.


\subsection{Jobs Can Only be Run Once}
If the build script is executed again, the \texttt{hello\_world} job
will not be re-built, simply because the Accelerator remembers that
the job has been built in the past, and its associated information is
stored in a job directory.  Instead, the Accelerator immediately
returns a job object representing the previous run.  This means that
from a user's perspective, there is no difference between job running
and job result recalling!  In order to have the method executing
again, either the source code or input parameters need to change.  If
there are changes, the method will be re-executed, and a new job will
be created that reflects these changes.


\subsection{Back to the ``Hello, World'' example}
Figure~\ref{fig:execflow-hello-world} illustrates the dispatch of the
\texttt{hello\_world} method.  The created job gets the \textsl{jobid}
\texttt{test-0}, and parts of the corresponding job directory
information is shown in green.  (Jobids are job identifiers, that are
named by their corresponding \textsl{workdir} plus an integer counter
value.)  The job directory contains several files, of which the most
important are
\begin{itemize}
\item[] \texttt{setup.json}, containing job meta information;
\item[] \texttt{result.pickle}, containing the returned data; and
\item[] \texttt{method.tar.gz}, containing the method's source code.
\end{itemize}

\begin{figure}[h!]
  \begin{center}
    \input{figures/job0.pdftex_t}
    \caption{A simple hello world program, represented as graph and
      work directory.}
    \label{fig:execflow-hello-world}
  \end{center}
\end{figure}

The \texttt{Job} class provides a convenient way to access important
files in this directory.  For example, the job's return value can be
loaded into a variable using the \texttt{.load()} function, like this
\begin{python}
def main(urd)
    job = urd.build('hello_world')
    print(job.load())
\end{python}
Running this build script will print the string to the \texttt{run}
program's standard output.



\subsection{Workdirs and Sharing Jobs}

Workdirs are used to separate jobs into different physical locations.
The Accelerator can be set up to have any number of workdirs
associated, but only one is used for writing.

If the same workdir is entered into two or more different user's
configuration files, the workdir and its contents will be shared
between the users.  Each Accelerator server will update its knowledge
about the contents of all workdirs before executing a build script, to
make sure that the latest jobs are taken into account.  The Urd
database, as described in chapter~\ref{chap:urd}, is very useful for
sharing job information between users.


\subsection{Linking Jobs}

Using jobs, complex tasks can be split into several smaller
operations.  Jobs can be connected so that the next job will depend on
the result of a previous job or set of jobs, and so on.

To continue the simple example, assume for a second that the ``hello
world''-job is computationally expensive, and that it returns a result
that is to be used as input to further processing.  To keep things
simple, this further processing is represented by printing the result
to standard output.  A new method \texttt{print\_result} is created,
and it goes like this
\begin{python}
jobs = {'hello_world_job',}

def synthesis():
    print(jobs.hello_world_job.load())
\end{python}
This method expects the \texttt{hello\_world\_job} input parameter to
be provided at execution time, and this is accomplished by the
following build script
\begin{python}
def main(urd):
    job1 = urd.build('hello_world')
    job2 = urd.build('print_result', hello_world_job=job1)
\end{python}
The \texttt{print\_result} method then loads the result from the
provided job and prints its contents to \texttt{stdout}.  Note that
this method does not return anything.

Figure~\ref{fig:execflow-print-result} illustrates the situation.
(Note the direction of the arrow: the second job, \texttt{test-1} has
\texttt{test-0} as input parameter, but \texttt{test-0} does not know
of any jobs run in the future.  Hence, arrows point to previous jobs.)

\begin{figure}[h!]
  \begin{center}
    \input{figures/job0job1.pdftex_t}
    \caption{Job \texttt{test-0}, is used as input to the
      \texttt{print\_result} job.}
    \label{fig:execflow-print-result}
  \end{center}
\end{figure}

The example shows how a complex task may be split into several jobs,
each reading intermediate results from previous jobs.  The Accelerator
will keep track of all job dependencies, so there is no doubt which
jobs that are run when and on which data.  Furthermore, since the
Accelerator remembers if a job has been executed before, it will link
and ``recycle'' previous jobs.  This may bring a significant
improvement in execution speed.  Furthermore, a recycled job is a
proof of that the code, input- and output data is connected.


\section{Datasets: Storing Data}

The \texttt{dataset} is the Accelerator's default storage type for
small or large quantities of data, designed for parallel processing
and high performance.  Datasets are built on top of jobs, so
\emph{datasets are created by methods and stored in job directories,
  just like any job result.}

Internally, data in a dataset is stored in a row-column format, and is
typically \emph{sliced} into a fixed number of slices to allow
efficient parallel access, see figure~\ref{fig:dataset}. Columns are
accessed independently, so there is no overhead in reading a single or
a set of columns.


\begin{figure}[h!]
  \begin{center}
    \input{figures/dataset_files.pdftex_t}
    \caption{A dataset containing three columns, $A$, $B$, and $C$
      stored using two slices.  Each dotted box corresponds to a file,
      so there are two files for each column, allowing for parallel
      read of the data using two processes.}
    \label{fig:dataset}
  \end{center}
\end{figure}

Furthermore, datasets may be \textsl{hash partitioned}, so that
slicing is based on the hash value of a given column.  Slicing on, for
example, a column containing some ID string will partition all rows
such that rows corresponding to any particular ID is stored in a
single slice only.  In many practical applications, hash partitioning
makes parallel processes independent, minimising the need for
complicated merging operations.  This is explained further in
section~\ref{sec:slicing_and_hashing}.



\subsection{Importing Data}

A project typically starts with \textsl{importing} some data from a
file on disk.  The bundled method \texttt{csvimport} is designed to
parse a plethora of ``comma separated values''-file formats and store
the data as a dataset.  See figure~\ref{fig:dataset_csvimport}.
\begin{figure}[h!]
  \begin{center}
    \input{figures/import_file1.pdftex_t}
    \caption{Importing \texttt{file0.txt}.}
    \label{fig:dataset_csvimport}
  \end{center}
\end{figure}
The method takes several input options in addition to the mandatory
filename to control the import process.  Here is an example
(non-simplified) invocation
\begin{python}
def main(urd):
    jid = urd.build('csvimport', filename='file0.txt')
\end{python}
When executed, the created dataset will be stored in the resulting job
directory, and the name of the dataset will by default be the jobid
plus the string \texttt{default}.  For example, if the
\texttt{csvimport} jobid is \texttt{imp-0}, the dataset will be
referenced by \texttt{imp-0/default}.  In this case, and always when
there is no ambiguity, the jobid alone (\texttt{imp-0}) could be used
too.  In general, a job could contain any number of datasets, but a
single dataset is a common case.




\subsection{Linking Datasets, Chaining}

Just like jobs can be linked to each other, datasets can link to each
other too.  Since datasets are build on top of jobs, this is
straightforward.  Assume the file \texttt{file0.txt} is imported into
dataset \texttt{imp-0/default}, and that there is more data like it
stored in the file \texttt{file1.txt}.  The second file is imported
with a link to the first dataset, see
figure~\ref{fig:dataset_csvimport_chain}.
\begin{figure}[h!]
  \begin{center}
    \input{figures/import_file0file1.pdftex_t}
    \caption{Chaining the import of \texttt{file1.txt} to the previous
      import of \texttt{file0.txt}.}
    \label{fig:dataset_csvimport_chain}
  \end{center}
\end{figure}
The \texttt{imp-1} (or \texttt{imp-1/default}) dataset reference can
now be used to access all data imported from \textsl{both} files!

Linking datasets containing related content is called \emph{chaining},
and this is particularly convenient when dealing with data that grows
over time.  A good example is any kind of \emph{log} data, such as
logs of transactions, user interactions, and similar.  Using chaining,
datasets can be with more rows just by linking, which is a lightweight
constant time operation.



\subsection{Adding New Columns to a Dataset}
In the previous section it was shown that datasets can be chained and
thereby grow in number of rows.  A dataset chain is created simply by
linking one dataset to the other, so the overhead is minimal.  In this
section it is shown that it is equally simple to add new columns to
existing datasets.  Adding columns is a common operation and the
Accelerator handles this situation efficiently using links.

The idea is very simple.  Assume a ``source'' dataset to which one or
more new columns should be added.  A new dataset is created containing
\textsl{only} the new column(s), and while creating it, the
constructor is instructed to link all the source dataset's columns to
the new dataset such that the new dataset appears to contain all
columns from both datasets.  (Note that this linking is similar to but
different from chaining.)

Accessing the new dataset will transparently access all the columns in
both the new and the source dataset in parallel, making it
indistinguishable from a single dataset.  See
Figure~\ref{fig:dep_dataset_append_column}.

\begin{figure}[h!]
  \begin{center}
    \input{figures/dataset_append_column.pdftex_t}
    \caption{Adding one new column to the source dataset.}
    \label{fig:dep_dataset_append_column}
  \end{center}
\end{figure}

A common case is to compute new columns based on existing ones.  In
this case, values are written to the new columns in the new dataset
while reading from the iterator iterating over the existing columns in
the source dataset.  This will be discussed in detail in
section~\ref{sec:appending_new_columns}



\subsection{Multiple Datasets in a Job}

Typically, a method creates a single dataset in the job directory, but
there is no limit to how many datasets that could be created and
stored in a single job directory.  This leads to some interesting
applications.

One application for keeping multiple datasets in a job is when data is
split into subsets based on some condition.  This could, for example,
be when a dataset is split into a training set and a test set.  One
way to achieve this using the Accelerator is by creating a Boolean
column that tells if the current row is train or test data, followed
by a job that splits the dataset in two based on the value on that
column.  See Figure~\ref{fig:dep_dataset_csvimport_chain}.

\begin{figure}[h!]
  \hspace{1cm}
  \input{figures/filter_dataset.pdftex_t}
  \caption{\texttt{job-1} separates the dataset
    \texttt{job-0/default} into two new datasets, named
    \texttt{job-1/train} and \texttt{job-1/test}.}
  \label{fig:dep_dataset_csvimport_chain}
\end{figure}

%% The figure shows how \texttt{job-1} has created two datasets,
%% \texttt{job-1/train} and \texttt{job-1/test}, based on the input
%% dataset \texttt{job-0/default}.  A third job, \texttt{job-2} is then
%% accessing the \texttt{job-1/train} dataset.  (Note that \texttt{job-1}
%% does not have a \texttt{default} dataset.)

In the setup of figure~\ref{fig:dep_dataset_csvimport_chain} we have
full tracking from either \texttt{train} or \texttt{test} datasets.
If we want to know the source of one of these sets, we just follow the
links back to the previous jobs until we reach the source job.  In the
figure, \texttt{job-0} may for example be a \texttt{csvimport} job,
and will therefore contain the name of the input file in its
parameters.  Thus, it is straightforward to link any data to its
source.

Splitting a dataset into parts creates ``physical'' isolation while
still keeping all the data at the same place.  No data is lost in the
process, and this is good for transparency reasons.  For example, a
following method may iterate over \textsl{both} datasets in
\texttt{job-1} and by that read the complete dataset.



\subsection{Parallel Dataset Access and Hashing}
As shown earlier in this chapter, data in datasets is stored in
multiple files for two reasons.  One reason is that we can read only
the columns that we need, without overhead, and the other is to allow
fast parallel reads.  The parameter \texttt{slices} determines how
many slices that the dataset should be partitioned into, and it also
sets the number of parallel process that may be used for processing
the dataset.  There is always one process for each slice of the
dataset, and each process operates on a unique part of the dataset.

Datasets can be partitioned, sliced, in different ways.  One obvious
way is to use round robin, where each consecutive data row is written
to the next slice, modulo the number of slices.  This leads to ``well
balanced'' datasets with approximately equal number of rows per slice.
Another alternative to slicing is to slice based on the hash value of
a particular column's values.  Using this method, all rows with the
same value in the hash column end up in the same slice.  This is
efficient for many parallel processing tasks, and we'll talk more
about it later on.

Methods may be designed simpler and more efficient using hash
partitioning, since the partitioning ensures some kind of data
independence between slices and processes.  If, however, the same
method is used on data that is not partitioned in the expected way, it
will not process the data correctly.  To ensure that an assumption
about hash partitioning is correct, there is an optional
\texttt{hashlabel} parameter to the iterators that will cause a
failure if the supplied column name does not correspond to the
dataset's hashlabel.

On the other hand it is also possible to have the iterator re-hash
on-the-fly.  In general this is not recommended, since there is a
\texttt{dataset\_rehash} method that does the same and stores the
result for immediate re-use.  Using \texttt{dataset\_rehash} will be
much more efficient.




\subsection{Dataset Column Types}

There are a number of useful types available for dataset columns.
They include \textsl{floating} and \textsl{integer point numbers},
\textsl{Booleans}, \textsl{timestamps}, several \textsl{string types}
(handling all kinds of encodings), and \textsl{json} types for storing
arbitrary data collections.  Most of these types come with advanced
parsers, making importing data from text files straightforward with
deterministic handling of errors, overflows, and so on.



\subsection{Dataset Attributes}
The dataset has a number of attributes associated with it, such as
shape, number of rows, column names and types, and more.
An attribute is accessed like this
\begin{python}
datasets = ('source',)
def synthesis():
    print(datasets.source.shape)
    print(datasets.source.columns)
\end{python}
and so on.


%%%%%%%%%%%%%%%%%%%%%%%%%%%%%%%%%%%%%%%%%%%%%%%%%%%%%%%%%%%%%%%%%%%%%%%%%%%%%%%%

\section{Iterators: Working with Data}

Data in a dataset is typically accessed using an \emph{iterator} that
reads and streams one dataset slice at a time to a CPU core.  The
parallel processing capabilities of the Accelerator makes it possible
to dispatch a set of parallel iterators, one for each slice, in order
to have efficient parallel processing of the dataset.

This section shows how iterators are used for reading data, how to
take advantage of slicing to have parallel processing, and how to
efficiently create new datasets.

\subsection{Iterator Basics}

Assume a dataset that has a column containing movie titles named
\texttt{movie}, and the problem is to extract the ten most frequent
movies.  Consider the following complete example
\begin{python}
from collections import Counter
datasets = ('source',)

def synthesis():
    c = Counter(datasets.source.iterate(None, 'movie'))
    print(c.most_common(10))
\end{python}
This will print the ten most common movie titles and their
corresponding counts in the \texttt{source} dataset.  The code will
run on a single CPU core, because we use the single-process
\texttt{synthesis} function, which is called and executed only once.
The \texttt{.iterate} (class-)method therefore has to read through all
slices, one at a time, in a serial fashion, and this is reflected by
the first argument to the iterator being \texttt{None}.



\subsection{Parallel Execution}
The Accelerator is much about parallel processing, and since datasets
are sliced, the program can be  modified to execute in parallel by
doing the following modification
\begin{python}
def analysis(sliceno):
    return Counter(datasets.source.iterate(sliceno, 'movie'))

def synthesis(analysis_res)
    c = analysis_res.merge_auto()
    print(c.most_common(10))
\end{python}
Here, \texttt{.iterate} is run inside the \texttt{analysis()}
function.  This function is forked once for each slice, and the
argument \texttt{sliceno} will contain an integer between zero and the
number of slices minus one.  The returned value from the analysis
functions will be available as input to the synthesis function in the
\texttt{analysis\_res} Python iterable.  It is possible to merge the
results explicitly, but the iterator comes with a rather magic method
\texttt{merge\_auto()}, which merges the results from all slices into
one based on the data type.  It can for example merge
\texttt{Counter}s, \texttt{set}s, and composed types like
\texttt{set}s of \texttt{Counter}s, and so on.  For larger datasets,
this version will run much faster.



\subsection{Iterating over Several Columns}
Since each column is stored independently in a dataset, there is no
overhead from reading a subset of a dataset's columns.  In the
previous section we've seen how to iterate over a single column using
\texttt{iterate}.  Iterating over more columns is straightforward by
feeding a list of column names to \texttt{iterate}, like in this
example
\begin{python}
from collections import defaultdict
datasets = {'source',}

def analysis(sliceno):
    user2movieset = defaultdict(set)
    for user, movie in datasets.source.iterate(sliceno, ('user', 'movie')):
        user2movieset[user].add(movie)
    return user2movieset
\end{python}
This example creates a lookup dictionary from users to sets of movies.
Note that in this case, we would like to have the dataset hashed on
the \texttt{user} column, so that each user appears in exactly one slice.
This will make later merging (if necessary) much easier.

It is also possible to iterate over all columns by specifying an empty
list of columns or by using the value \texttt{None}.
\begin{python}
...
def analysis(sliceno):
    for columns in datasets.source.iterate(sliceno, None):
        ...
\end{python}
Here, \texttt{columns} will be a list of values, one for each column
in the dataset.


\subsection{Iterating over Dataset Chains}

The \texttt{iterate} function is used to iterate over a single
dataset.  There is a corresponding function, \texttt{iterate\_chain},
that is used for iterating over chains of datasets.  This function
takes a number of arguments, such as
\begin{itemize}
\item[] \texttt{length}, i.e.\ the number of datasets to iterate over.
  By default, it will iterate over all datasets in the chain.
\item[] \texttt{callbacks}, functions that can be called before and/or
  after each dataset in a chain.  Very useful for aggregating data
  between datasets.
\item[] \texttt{stop\_id} which stops iterating at a certain dataset.
  This dataset could be from \textsl{another} job's parameters, so we
  can for example iterate exactly over all new datasets not covered by
  a previous job.
\item[] \texttt{range}, which allows for iterating over a range of
  data.
\end{itemize}
The \texttt{range} options is based on the max/min values stored for
each column in the dataset.  Assuming that the chain is sorted, one
can for example set
\begin{python}
  range={timestamp, ('2016-01-01', '2016-01-31')}
\end{python}
in order to get rows within the specified range only.  Using
\texttt{range=} is quite costly, since it requires each row in the
dataset chain with dates within the range to be checked against the
range criterion.  Therefore, there is a \texttt{sloppy} version that
iterates over complete datasets in the chain that contains at least
one row with a date within the range.  This is useful, for example, to
very quickly produce histograms or plots of subsets of the data.



%% \subsection{Dataset Translators and Filters}

%% The iterator may perform data translation and filtering on-the-fly
%% using the \texttt{translators} and \texttt{filters} options.  Here is
%% an example of how a dictionary can be fed into the iterator to map a
%% column
%% \begin{python}
%% mapper = {2: "HUMANLIKE", 4: "LABRADOR", 5: "STARFISH",}
%% for animal in datasets.source.iterate_chain(sliceno, \
%%   "NUM_LEGS", translator={"NUM_LEGS": mapper,}):
%%     ...
%% \end{python}
%% This will iterate over the \texttt{NUM\_LEGS} column, and map numbers
%% to strings according to the \texttt{mapper} dict.

%% Filters work similarly.



\subsection{Job Execution Flow and Result Passing}

Execution of code in a method is either parallel or serial depending
on which function is used to encapsulate it.  There are three
functions in a method that are called from the Accelerator when a
method is running, and they are \texttt{prepare()},
\texttt{analysis()}, and \texttt{synthesis()}.  All three may exist in
the same method, and at least one is required.  When the method
executes, they are called one after the other.
\begin{itemize}
\item[] \texttt{prepare()} is executed first.  The returned value is
  available in the variable \texttt{prepare\_res}.
\item[] \texttt{analysis()} is run in parallel processes, one for each
  slice.  It is called after completion of \texttt{prepare()}.  Common
  input parameters are \texttt{sliceno}, holding the number of the
  current process instance, and \texttt{prepare\_res}.  The return
  value for each process becomes available in the
  \texttt{analysis\_res} variable.
\item[] \texttt{synthesis()} is called after the last
  \texttt{analysis()}-process is completed.  It is typically used to
  aggregate parallel results created by \texttt{analysis()} and takes
  both \texttt{prepare\_res} and \texttt{analysis\_res} as optional
  parameters.  The latter is an iterator of the results from the
  parallel processes.
\end{itemize}
Figure~\ref{fig:prepanasyn} shows the execution order from top to
bottom, and the data passed between functions in coloured branches.
\texttt{prepare()} is executed first, and its return value is
available to both the \texttt{analysis()} and \texttt{synthesis()}
functions.  There are \texttt{slices} (a configurable parameter)
number of parallel \texttt{analysis()} processes, and their output is
available to the \texttt{synthesis()} function, which is executed
last.

Return values from any of the three functions may be stored in the
job's directory making them available to other jobs.


\begin{figure}[h!]
  \begin{center}
    \input{figures/prepanasyn.pdftex_t}
    \caption{Execution flow and result propagation in a method.}
    \label{fig:prepanasyn}
  \end{center}
\end{figure}



\subsection{Job Parameters}
\label{sec:jobparams}
We've seen how completed jobs can be used as input to new
jobs.  Jobs are one of three kinds of input parameters that
a job can take.  Here the input parameters are summarised:
\begin{itemize}
\item[] \texttt{jobs}, a set of identifiers to previously executed jobs;
\item[] \texttt{options}, a dictionary of options; and
\item[] \texttt{datasets}, a set of input \textsl{datasets}.
\end{itemize}
See Figure~\ref{fig:execflow}.  Parameters are entered as global
variables early in the method's source.


\begin{figure}[h!]
  \begin{center}
    \input{figures/execflow.pdftex_t}
    \caption{Execution flow of a method.  The method takes optionally
      three kinds of parameters: \texttt{options}, \texttt{jobs},
      and \texttt{datasets}.}
    \label{fig:execflow}
  \end{center}
\end{figure}





\section{A Class Based Programming Model}
See figure~\ref{fig:classes}.
\begin{figure}[h!]
  \begin{center}
    \section{The \texttt{Job} and \texttt{CurrentJob} Classes}
The \texttt{Job} class is used to represent and operate on existing
jobs.  An object of this class is returned from job \texttt{build()}
calls as well as when retreiving jobs from Urd or a \texttt{JobList}
object.  The \texttt{CurrentJob} class is an extension that provides
mechanisms for operations performed while a job is executing, such as
saving files to the job's jobdir.  The classes are derived from
the \texttt{str} class, and objects of these classes decay to
(unicode) strings when pickled.

The following attributes are available on both the \texttt{Job}
and \texttt{CurrentJob} classes:
\starttabletwo
\texttt{method} & The job method.  This can be overriden by \texttt{name=} if job instance is output from Urd or a \texttt{build()} call.\\
\texttt{path} & The filesystem directory where the job is stored.\\
\texttt{workdir} & The workdir name (the part before \texttt{-number} in the jobid).\\
\texttt{number} & The job number as an \texttt{int}.\\
\texttt{filename()} & Return absolute path to a file in a job.\\
\texttt{withfile()} & A \texttt{JobWithFile} with this job.\\
\texttt{params} & Return a dict corresponding to the file \texttt{setup.json} for this job.\\
\texttt{post} & Return a dict corresponding to \texttt{post.json} for this job.\\
\texttt{open()} & Similar to standard \texttt{open}, use to open files\\
\texttt{load()} & Load a pickle file from the job's directory.\\
\texttt{json\_load()} & Load a json file from the job's directory.\\
\texttt{dataset()} & Return a named dataset from the job.\\
\texttt{datasets} & List of datasets in this job.\\
\texttt{output()} & Return what the job printed to \texttt{stdout} and \texttt{stderr}.\\
\stoptabletwo
\noindent In addition, the \texttt{CurrentJob} class has these
attributes too:
\starttable
\texttt{datasetwriter()} && to get a DatasetWriter object.  See documentation for \texttt{Dataset.DatasetWriter()}, section~\ref{}.\\
\texttt{open()}&& With extra temp argument.\\
\texttt{save()} && to store a pickle file\\
\texttt{json\_save()} && to store a json file\\
\stoptable
\noindent Detailed description of the functions, where neccessary, follows.



\subsection{\texttt{Job.filename()}}
%\begin{leftbar}
\starttable
\texttt{filename} & \textsl{Mandatory} & Name of file in job directory.\\
\texttt{sliceno}  & \pyNone & Set to current slice number if sliced, otherwise \pyNone.\\
\stoptable
Return the absolute (full path) filename to a file stored in the job.
If the file is sliced, a particular slice file can be retrieved using
the \texttt{sliceno} parameter.  Sliced files are described in section~\ref{sec:sliced_files}.
%\end{leftbar}



\subsection{\texttt{Job.open()}}
%\begin{leftbar}
\starttable
\texttt{filename} & \textsl{Mandatory} & Name of file.\\
\texttt{mode} & \texttt{r} & Open file in this mode, see Python's \texttt{open()}\\
\texttt{sliceno} & \pyNone & Read or write sliced files.\\
\texttt{encoding} & \pyNone & Same as Python's \texttt{open()}\\
\texttt{errors} & \pyNone & Same as Python's \texttt{open()}\\
\texttt{temp} & \pyNone & Control file persistence.  See text.\\
\stoptable
This is a wrapper around the standard \texttt{open} function with some
extra features.  Note that
\begin{itemize}
\item[--]  \texttt{Job.open()} can only read files, not write
them, and therefore ``\texttt{r}'' flag must be set.
\item[--]  \texttt{CurrentJob.open()} can both read and write.
\item[--]  \texttt{CurrentJob.open()} must be used as a context manager,
like this
\begin{python}
with job.open(...) as fh:
    ....
\end{python}
\item[--]  \texttt{CurrentJob.open()} can use the \texttt{temp} flag to
modify the persistence of written files.
\end{itemize}
The \texttt{temp} argument is used to control the persistence of files
written using \texttt{.open()}.  This is useful mainly for debug
purposes, and explained in section~\ref{sec:debugflag}.  Sliced files
are described in section~\ref{sec:slicedfiles}.
%\end{leftbar}


\subsection{\texttt{Job.withfile()}}
%\begin{leftbar}
\starttable
\texttt{filename} & \textsl{Mandatory} & Name of file.\\
\texttt{sliced} & \pyFalse & Boolean indicating if the file is sliced or not.\\
\texttt{extra} & \pyNone & Any additional information to the job to be built.\\
\stoptable
The \texttt{.withfile()} is used to highlight a specific file in a job
and feed it to another job \texttt{build()}.  The file could be
sliced.
%\end{leftbar}


\subsection{\texttt{Job.load()}}
%\begin{leftbar}
\starttable
\texttt{filename} & \texttt{result.pickle} & \hspace{2ex}Name of file.\\
\texttt{sliceno} & \pyNone & \\
\texttt{encoding} & \texttt{bytes} & \\
\stoptable
Load a file from a job in Python's pickle format.
%\end{leftbar}


\subsection{\texttt{Job.json\_load()}}
%\begin{leftbar}
\starttable
\texttt{filename} & \texttt{result.json} & Name of file.\\
\texttt{sliceno} & \pyNone & \\
%\texttt{unicode\_as\_utf8bytes} & \texttt{PY2} & \\
\stoptable
Load a file from a job, in JSON format.
%The \texttt{unicode\_as\_utf8bytes} flag is active in Python2 in order
%to let strings decode as bytes.  Override this for unicode.  In
%Python3, strings decode as unicode.
%\end{leftbar}


\subsection{\texttt{Job.dataset()}}
%\begin{leftbar}
\starttable
\texttt{name} & \texttt{default} & \\
\stoptable
Get a dataset instance from a job.
%\end{leftbar}


\subsection{\texttt{Job.output()}}
%\begin{leftbar}
\starttable
\texttt{what} & \pyNone & \\
\stoptable
Get everything a job has printed to \texttt{stdout}
and \texttt{stderr} in a string variable.
%\end{leftbar}


\subsection{\texttt{Job.save()}}
%\begin{leftbar}
\starttable
\texttt{obj} & \textsl{Mandatory} & \\
\texttt{filename} & \texttt{result.pickle} & \\
\texttt{sliceno} & \pyNone & \\
\texttt{temp} & \pyNone & \\
\stoptable
For \texttt{CurrentJob} instances only.  Save data into the current
job's directory in Python's pickle format.
The \texttt{temp} argument is used to control the persistence of files
written using \texttt{.save()}.  This is useful mainly for debug
purposes, and explained in section~\ref{sec:debugflag}.
%\end{leftbar}


\subsection{\texttt{Job.json\_save()}}
%\begin{leftbar}
\starttable
\texttt{obj} & \textsl{Mandatory} & \\
\texttt{filename} & \texttt{result.json} & \\
\texttt{sliceno} & \pyNone & \\
\texttt{sort\_keys} & \pyTrue & \\
\texttt{temp} & \pyNone & \\
\stoptable
For \texttt{CurrentJob} instances only.  Save data into the current
job's directory in JSON format.
The \texttt{temp} argument is used to control the persistence of files
written using \texttt{.json\_save()}.  This is useful mainly for debug
purposes, and explained in section~\ref{sec:debugflag}.
%\end{leftbar}


\subsection{Sliced Files}
\label{sec:slicedfiles}
%\begin{leftbar}
A \textsl{sliced} file is actually a set of files used to store data
independently in each \analysis process using a common name.  The
functions that operate on files, such as for example \texttt{.open()}
and \texttt{.load()}, can switch to sliced files using
the \texttt{sliceno} parameter.  From a user's perspective, they
always appear to work on single files.  For example
\begin{python}
def analysis(sliceno, job):
    data = ...
    job.date(data, "mydata", sliceno=sliceno, temp=False)
\end{python}
will create a set of files \texttt{mydata.\%d}, where \texttt{\%d} is
replaced by the slice number.  In this way, data can be passed ``in
parallel'' between different jobs.
%\end{leftbar}


\subsection{File Persistence}
\label{sec:debugflag}
The \texttt{temp} argument controls persistence of files stored
using \texttt{.open()}, \texttt{.save()}, or \texttt{.json\_save()}.
By default it is being set to \pyFalse, which implies that the stored
file is \textsl{not} temporary.  But setting it to \pyTrue, like in
the following
\begin{python}
job.save(data, filename, temp=True)
\end{python}
will cause the stored file to be deleted upon job completion.  The
functionality can be combined with the \textsl{debug} mode, see below.
\begin{snugshade}
\begin{center}
\begin{tabular*}{\textwidth}{l@{\extracolsep{\fill}}ll}
  \texttt{temp}      & ``normal'' mode     & debug mode  \\\hline
  \pyFalse           & stored              & stored\\
  \pyTrue            & stored and removed  & stored\\
\end{tabular*}
\end{center}
\end{snugshade}
\noindent Debug mode is active if the Accelerator \texttt{daemon} is
started with the \texttt{---debug} flag.


\clearpage
\section{The \texttt{JobWithFile} Class}
The \texttt{JobWithFile} class is used to create a job input parameter
from a file stored in a job.
\starttabletwo
\texttt{resolve()} & Return filename. \\
\texttt{load()} & load file contents. \\
\texttt{json\_load()} & load JSON file contents.\\
\stoptabletwo
All three functions take the argument \texttt{sliceno}, which default
is set to \pyNone, indicating that it is actually a single file on
disk.  If \texttt{sliceno} is set, it is assumed that the file is
sliced, see section~\ref{sec:slicedfiles}, and the function will look
up that slice of the file only.



\clearpage
\section{The \texttt{JobList} Class}
Objects of the \texttt{JobList} class are returned by member functions
to the \texttt{Urd} class.  They are used to group sessions of jobs
together.
\starttabletwo
\texttt{find()} & Return a new \texttt{JobList} with only jobs with that method or name in it.\\
\texttt{get()} & Return the latest \texttt{Job} with that method or name.\\
\texttt{[<method>]} & Same as \texttt{.get} but error if no job with that method or name is in the list.\\
\texttt{as\_tuples} &  The \texttt{JobList} represented as \texttt{(method, jid)} \texttt{tuple}s.\\
\texttt{pretty} & Return a prettified string version of the \texttt{JobList}.\\
\texttt{profile} & Total execution time and execution time per method.\\
\texttt{print\_profile()} & Print profiling information to \texttt{stdout}.\\
\stoptabletwo
\noindent Detailed description of the functions, where neccessary, follows.


\subsection{\texttt{JobList.find()}}
%\begin{leftbar}
\starttable
\texttt{method} & \textsl{Mandatory} & Method or name to find.\\
\stoptable
Return a new \texttt{Joblist} will all jobs in the
current \texttt{JobList} matching the \texttt{method} argument.  The
matching part is either the unique name of the method's source code,
or the name optionally given at build time using the \texttt{name=}
argument.
%\end{leftbar}


\subsection{\texttt{JobList.get()}}
%\begin{leftbar}
\starttable
\texttt{method} & \textsl{Mandatory} & Method or name to find.\\
\texttt{default} & \pyNone & Return the latest matching job.\\
\stoptable
Return the latest job that matches the \texttt{method} argument.  The
matching part is either the unique name of the method's source code,
or the name optionally given at build time using the \texttt{name=}
argument.  If no matches are found, it will return
the \texttt{default} argument.
%\end{leftbar}


\subsection{\texttt{JobList.print\_profile()}}
%\begin{leftbar}
\starttable
\texttt{verbose} & \pyTrue & In addition to total time, print execution time for each method in list.\\
\stoptable
Print total execution time for the \texttt{Joblist}, and,
conditionally, execution time for each job in the list,
to \texttt{stdout}.
%\end{leftbar}



\clearpage
\section{The \texttt{Dataset} Class}
The \texttt{Dataset} class is used to operate on small or large
datasets stored on disk.  It decays to a (unicode) string when
pickled.

\starttabletwo
\texttt{columns} & A \texttt{dict} from column to properties, such as type, min, and max values.\\
\texttt{previous} & The dataset's previous dataset, if it exists, \pyNone otherwise.\\
\texttt{parent} & The dataset's parent dataset, if it exists, \pyNone otherwise.\\
\texttt{filename} & The dataset's filename, if it exists.  (\texttt{csvimport} sets this.)\\
\texttt{hashlabel} & Column used for hash partitioning, or \pyNone.\\
\texttt{caption} & The dataset's caption.\\
\texttt{lines} & A \texttt{list} with number of lines per slice.\\
\texttt{shape} & A tuple containing number of columns and number of lines in dataset.\\
\texttt{link\_to\_here()} & Used to associate a subjob's dataset with the current job, see section~\ref{sec:linktohere}.\\
\texttt{merge()} & Merge this dataset with another dataset, see section~\ref{sec:datasetmerge}.\\
\texttt{chain()} & A \texttt{DatasetChain} object, see section~\ref{sec:dschain}\\
\texttt{iterate\_chain()} & Iterator over chains, see section~\ref{sec:iterator}.\\
\texttt{iterate()} & Iterator over dataset see section~\ref{sec:iterator}.\\
\texttt{iterate\_list()} & Iterator over a list of datasets, see section~\ref{sec:iterator}.\\
\stoptabletwo
\noindent Detailed description of the functions, where neccessary, follows.


\subsection{\texttt{Dataset.link\_to\_here()}}
%\begin{leftbar}
\starttable
\texttt{name} & \texttt{default} & The new name of the dataset.\\
\texttt{column\_filter} & \pyNone & Iterable of columns to include, or \pyNone to get all.\\
\texttt{override\_previous} & \texttt{\_no\_override} & Set this to the new previous.\\
\stoptable
Use this to expose a subjob as a dataset in your job, like in this
example:
\begin{python}
def synthesis():
    job = build('ex')
    job.dataset().link_to_here(name='new')
\end{python}
The current job will now appear to have a dataset named \texttt{new},
that is actually a link to the subjob's \texttt{default} dataset.  It
is possible to filter which columns should be visible in the link
using \texttt{column\_filter}.  For chaining purposes, it is possible
for the link to expose a parent dataset of choice, set using
the \texttt{override\_previous} parameter.
%\end{leftbar}


\subsection{\texttt{Dataset.merge()}}
%\begin{leftbar}
\starttable
\texttt{other} & \textsl{Mandatory} & Merge with this dataset.\\
\texttt{name} & ``\texttt{default}''& Name of new dataset\\
\texttt{previous} & \pyNone& The new dataset's \texttt{previous} dataset.\\
\texttt{allow\_unrelated} & \pyFalse& Set this if the datasets do not share a common ancestor.\\
\stoptable
Merge this and other dataset. Columns from the other dataset take
priority.  If datasets do not have a common ancestor you get an error
unless \texttt{allow\_unrelated} is set. The new dataset always has
the previous specified here (even if \pyNone).  Returns the new
dataset.
%\end{leftbar}


\subsection{\texttt{Dataset.chain()}}
%\begin{leftbar}
\starttable
\texttt{length} & \texttt{-1} & Number of datasets in chain.  The default value of \texttt{-1} will include all datasets in chain.\\
\texttt{reverse} & \pyFalse & Reverse order of chain.\\
\texttt{stop\_ds} & \pyNone & If set, chain will start at the dataset after \texttt{stop\_ds}.\\
\stoptable
This function will return a \texttt{DatasetChain} object, see
section~\ref{sec:datasetchain}.
%\end{leftbar}



\clearpage
\section{The \texttt{DatasetChain} Class}
These are lists of datasets returned from Dataset.chain.
They exist to provide some convenience methods on chains.
\starttabletwo
\texttt{min()} & Min value for a specified column over the whole chain.\\
\texttt{max()} & Max value for a specified column over the whole chain.\\
\texttt{lines()} & Number of rows in this chain, optionally for a specific slice. \\
\texttt{column\_counts()} & The number of datasets each column appears in.\\
\texttt{column\_count()} & Number of datasets in this chain that contain a specified column.\\
\texttt{with\_column()} & Return a new chain without any datasets that don't contain a specified column.\\
\stoptabletwo
\noindent Detailed description of the functions, where neccessary, follows.


\subsection{\texttt{DatasetChain.min()}, \texttt{DatasetChain.max()}}
%\begin{leftbar}
\starttable
\texttt{column} & \textsl{Mandatory} & Min/max value of column, see text.\\
\stoptable
Minimum or maximum value for column over the whole chain.  Will be
\pyNone if no dataset in the chain contains \texttt{column}, if all datasets are
empty or if \texttt{column} has a type without min/max tracking.
%\end{leftbar}


\subsection{\texttt{DatasetChain.lines()}}
%\begin{leftbar}
\starttable
\texttt{sliceno} & \pyNone & If set, return number of lines in speficied slice.\\
\stoptable
Number of rows in this chain, optionally for a specific slice.
%\end{leftbar}


\subsection{\texttt{DatasetChain.column\_counts()}}
%\begin{leftbar}
Return a Python \texttt{Counter},\mintinline{python}|{colname:
occurances}|, holding the number of datasets each column appears in.
Takes no options.
%\end{leftbar}


\subsection{\texttt{DatasetChain.column\_count()}}
%\begin{leftbar}
\starttable
\texttt{column} & \textsl{Mandatory} & A column name.\\
\stoptable
Number of datasets in this chain that contain a specified column.
%\end{leftbar}


\subsection{\texttt{DatasetChain.with\_column()}}
%\begin{leftbar}
\starttable
\texttt{column} & \textsl{Mandatory} & A column name.\\
\stoptable
Return a new \texttt{DatasetChain} with all datasets in this chain
containing a speficied column.
%\end{leftbar}



\clearpage
\section{The \texttt{DatasetWriter} Class}
The \texttt{DatasetWriter} class is used to create datasets.  Datasets
could be stand-alone, part of a chain, or an extension (new columns)
to an existing dataset.

The class has a number of member functions, described below, that may
be used for dataset creation.  Alternatively, the new dataset could be
set up using the \texttt{DatasetWriter} \textsl{constructor}.  The
constructor approach is currently only documented in the source code,
see \texttt{dataset.py}.

\starttabletwo
\texttt{add()} & Add a new column to the dataset under creation.\\
\texttt{hashcheck()} & Check if value belongs in current slice.\\
\texttt{set\_slice()} & Set which slice that will receive the next write.\\
\texttt{enable\_hash\_discard()} & Make the write functions silently discard data that does not hash to the current slice. \\
\texttt{get\_split\_write()} & Get a writer object, see section~\ref{}.\\
\texttt{get\_split\_write\_list()} & Get a writer object, see section~\ref{}.\\
\texttt{get\_split\_write\_dict()} & Get a writer object, see section~\ref{}.\\
\texttt{discard()} & Discard the dataset under creation.\\
\texttt{finish()} & Call this if dataset is to be used before creating job finishes, e.g.\ if the dataset under creation is input to a subjob.\\
\stoptabletwo
\noindent Detailed description of the functions, where neccessary, follows.


\subsection{\texttt{DatasetWriter.add()}}
\label{sec:datasetwriter_add}
%\begin{leftbar}
\starttable
\texttt{colname} & \textsl{Mandatory} & Name of new column.\\
\texttt{coltype} & \textsl{Mandatory} & Type of new column.\\
\texttt{none\_support} & \pyFalse & Set to \pyTrue to allow storing {\pyNone}s.\\
\stoptable

Add a new column to a dataset in creation.  This example will create
an \texttt{age} column of type \texttt{number}, where the values could
also be \pyNone.
\begin{python}
dw.add('age', 'number', none_support=True)
\end{python}
All dataset types are described in section~\ref{}.
%\end{leftbar}


\subsection{\texttt{DatasetWriter.hashcheck()}}
%\begin{leftbar}
\starttable
\texttt{value} & \textsl{Mandatory} & Some data/\\
\stoptable
Check if a value belongs to the current slice.
Return \pyTrue if \texttt{value} belongs to the current
slice, \pyFalse otherwise.
%\end{leftbar}


\subsection{\texttt{DatasetWriter.set\_slice()}}
%\begin{leftbar}
\starttable
\texttt{sliceno} & \textsl{Mandatory} & Slice number to use for writing.\\
\stoptable
Specify which slice that will receive the next write(s).  Use this if
writing data in \prepare or \synthesis.
%\end{leftbar}


\subsection{\texttt{DatasetWriter.enable\_hash\_discard()}}
%\begin{leftbar}
Takes no options.  Set this in each slice or after
each \texttt{set\_slice()} to make the writer discard values that do
not belong to the current slice.
%\end{leftbar}



\clearpage
\section{The \texttt{Urd} Class}
\starttabletwo
\texttt{get()} & Get an Urd item from a specified list and timetamp.\\
\texttt{latest()} & Get the latest Urd item for a specified list.\\
\texttt{first()} & Get the first Urd item for a specified list.\\
\texttt{peek()} & Get an Urd item from a specified list and timetamp without recording.\\
\texttt{peek\_latest()} & Get the latest Urd item for a specified list without recording.\\
\texttt{peek\_first()} & Get the first Urd item for a specified list without recording.\\
\texttt{since()} & Get all timestamps later than a specified timestamp for a specified list.\\
\texttt{list()} & List all Urd lists\\
\texttt{begin()} & Start a new Urd session.\\
\texttt{abort()} & Abort a running Urd session.\\
\texttt{finish()} & Finish a running Urd session and store its contents.\\
\texttt{truncate()} & Discard all Urd items later than a specified timestamp for a specified list.\\
\texttt{set\_workdir()} & Set the target workdir.\\
\texttt{build()} & Build a job.\\
\texttt{build\_chained()} & Build a job with chaining.\\
\texttt{warn()} & Add a warning message to be displayed at the end of the build.\\
\stoptabletwo
\noindent Detailed description of the functions, where neccessary, follows.


\subsection{\texttt{Urd.get()}}
\starttable
\texttt{path} & \textsl{Mandatory} & \\
\texttt{timestamp} & \textsl{Mandatory} & \\
\stoptable
Get an Urd item with specified list and timestamp.  The operation is
recorded in the current Urd session.
%\begin{leftbar}
%\end{leftbar}


\subsection{\texttt{Urd.latest()}}
\starttable
\texttt{path} & \textsl{Mandatory} & \\
\stoptable
Get the latest job in a specified Urd list.  The operation is recorded
in the current Urd session.
%\begin{leftbar}
%\end{leftbar}


\subsection{\texttt{Urd.first()}}
\starttable
\texttt{path} & \textsl{Mandatory} & \\
\stoptable
Get the first job in a specified Urd list.  The operation is recorded
in the current Urd session.
%\begin{leftbar}
%\end{leftbar}


\subsection{\texttt{Urd.peek()}}
\starttable
\texttt{path} & \textsl{Mandatory} & \\
\texttt{timestamp} & \textsl{Mandatory} & \\
\stoptable
Same as \texttt{.get()}, but without recording the dependency.
%\begin{leftbar}
%\end{leftbar}


\subsection{\texttt{Urd.peek\_latest()}}
\starttable
\texttt{path} & \textsl{Mandatory} & \\
\stoptable
Same as \texttt{.latest()}, but without recording the dependency.
%\begin{leftbar}
%\end{leftbar}


\subsection{\texttt{Urd.peek\_first()}}
\starttable
\texttt{path} & \textsl{Mandatory} & \\
\stoptable
Same as \texttt{.first()}, but without recording the dependency.
%\begin{leftbar}
%\end{leftbar}


\subsection{\texttt{Urd.since()}}
\starttable
\texttt{path} & \textsl{Mandatory} & \\
\texttt{timestamp} & \textsl{Mandatory} & \\
\stoptable
Return a list of all timestamps more recent than the
input \texttt{timestamp} for a specified Urd list.
%\begin{leftbar}
%\end{leftbar}


\subsection{\texttt{Urd.list()}}
Return a list of all available Urd lists.
%\begin{leftbar}
%\end{leftbar}


\subsection{\texttt{Urd.begin()}}
\starttable
\texttt{path} & \textsl{Mandatory} & \\
\texttt{timestamp} & \textsl{Mandatory} & \\
\texttt{caption} & \pyNone & \\
\texttt{update} & \pyFalse & \\
\stoptable
Start a new Urd session.
%\begin{leftbar}
%\end{leftbar}


\subsection{\texttt{Urd.abort()}}
Abort the current Urd session, discard its contents.
%\begin{leftbar}
%\end{leftbar}


\subsection{\texttt{Urd.finish()}}
\starttable
\texttt{path} & \textsl{Mandatory} & \\
\texttt{timestamp} & \textsl{Mandatory} & \\
\texttt{caption} & \pyNone & \\
\stoptable
Finish the current Urd session and store it in the Urd database.
%\begin{leftbar}
%\end{leftbar}


\subsection{\texttt{Urd.truncate()}}
\starttable
\texttt{path} & \textsl{Mandatory} & \\
\texttt{timestamp} & \textsl{Mandatory} & \\
\stoptable
Discard everything later than \texttt{timestamp} for the specified Urd
list.
%\begin{leftbar}
%\end{leftbar}


\subsection{\texttt{Urd.set\_workdir()}}
\starttable
\texttt{workdir} & \textsl{Mandatory} & \\
\stoptable
Set target workdir.  It can be set to any workdir present in the
Accelerator's configuration file.
%\begin{leftbar}
%\end{leftbar}


\subsection{\texttt{Urd.build()}}
%\begin{leftbar}
\starttable
\texttt{method} & \textsl{Mandatory} & Method to build.\\
\texttt{options} & \texttt{\{\}} & Input options.\\
\texttt{datasets} & \texttt{\{\}} & Input datasets.\\
\texttt{jobs} & \texttt{\{\}} & Input jobs.\\
\texttt{name} & \pyNone & Record job using this name instead of method name.\\
\texttt{caption} & \pyNone & Optional caption\\
%\texttt{why\_build} & \pyFalse & \\
\texttt{workdir} & \pyNone & Store job in this workdir.\\
\stoptable
Build a job.  If an Urd session is running, the job and its
dependencies will be recorded.
%\end{leftbar}


\subsection{\texttt{Urd.build\_chained()}}
Build a chained job. @@@@@@@@@@@@@@@@  TESTA DETTA!
%\begin{leftbar}
Same options as \texttt{build}
%\end{leftbar}


\subsection{\texttt{Urd.warn()}}
Print a string to \texttt{stdout} when the build script ends with no
errors.
\starttable
\texttt{line} & \textsl{Mandatory} & Some string\\
\stoptable

    \caption{Most important relations between classes.}
    \label{fig:classes}
  \end{center}
\end{figure}



\section{Accelerator Exceptions}
There are a number of custom defined \texttt{Exception}s in the
Accelerator code in order to simplify debugging.



%%%%%%%%%%%%%%%%%%%%%%%%%%%%%%%%%%%%%%%%%%%%%%%%%%%%%%%%%%%%%%%%%%%%%%%%%%%%%%%%

