%%%%%%%%%%%%%%%%%%%%%%%%%%%%%%%%%%%%%%%%%%%%%%%%%%%%%%%%%%%%%%%%%%%%%%%%%%%%
%                                                                          %
% Copyright (c) 2018 eBay Inc.                                             %
%                                                                          %
% Licensed under the Apache License, Version 2.0 (the "License");          %
% you may not use this file except in compliance with the License.         %
% You may obtain a copy of the License at                                  %
%                                                                          %
%  http://www.apache.org/licenses/LICENSE-2.0                              %
%                                                                          %
% Unless required by applicable law or agreed to in writing, software      %
% distributed under the License is distributed on an "AS IS" BASIS,        %
% WITHOUT WARRANTIES OR CONDITIONS OF ANY KIND, either express or implied. %
% See the License for the specific language governing permissions and      %
% limitations under the License.                                           %
%                                                                          %
%%%%%%%%%%%%%%%%%%%%%%%%%%%%%%%%%%%%%%%%%%%%%%%%%%%%%%%%%%%%%%%%%%%%%%%%%%%%

\label{chap:urd}

\section{Introduction to Urd}

Urd is the processing flow controller in the framework.  It is the
primary job dispatcher as well as the bookkeeper of all jobs executed.
Events in urd are quantified into what is called \textsl{sessions}.
The core of Urd is a transaction log database storing these sessions
together with meta information.  The result is a server providing
lookups for all jobs executed together with their context.

The Urd database is partitioned into what is called \textsl{lists}.
Lists are where information about executed jobs are stored, and how
they relate to eachother.  Lists are globally readable, but writing
requires authentication, so that, for example, only the production
user may publish a model to go live.

With the exception of experimental work, all work initiated by Urd is
run in closed sessions, with well defined starting and ending points.
The input dependencies to these sessions are recorded, together with
the resulting output.



\section{Urd Sessions}

A minimal example of creating an Urd session is as follows
\begin{python}
def main(urd):
    urd.begin('test')
    ...
    urd.finish('test', '2016-10-25')
\end{python}
Every job that is dispatched between \texttt{begin} and
\texttt{finish} will be appended to the \texttt{test} Urd list with
timestamp \texttt{20161025}.  In addition, all lookups of jobs done by
Urd will be appended to the Urd list as dependencies.

The Urd list will be updated only when the \texttt{finish} function is
called, since it is responsible for updating the urd transaction log.
Before \texttt{finish}, nothing is stored, and it is perfectly okay to
omit \texttt{finish} during development work.

There are a number of options associated with a session, as shown
here,
\begin{python}
urd.begin(path, timestamp, caption=None, update=False)
urd.finish(path, timestamp, caption=None)
\end{python}
and the following applies
\begin{itemize}
\item [] \texttt{path} is the name of the Urd list, and the same
  \texttt{path} must be specified in both \texttt{begin} and
  \texttt{finish}.

\item [] \texttt{timestamp} is mandatory, but could be set in either
  \texttt{begin}, \texttt{finish}, or both.  \texttt{finish}
  overrides \texttt{begin}.

\item [] \texttt{caption} is optional, and can be set in either
  \texttt{begin} or \texttt{finish}.  \texttt{finish} overrides
  \texttt{begin}.
\end{itemize}
There is also an \texttt{update} option that will be discussed in
section~\ref{sec:trunc-update}.



\subsection{Timestamp resolution}

Timestamps may be specified in various resolution depending on the
application.  The time format is
\begin{verbatim}
"%Y-%m-%dT%H:%M:%S"
\end{verbatim}
and it can be truncated as shown in the following examples covering all possible cases.
\begin{python}
'2016-10-25'               day resolution
'2016-10-25T15'            hour resolution
'2016-10-25T15:25'         minute resolution
'2016-10-25T15:25:00'      second resolution
\end{python}



\subsection{Aborting an Urd Session}

When an Urd session is initiated, a new session cannot be started
until the current session has finished.  A session may be aborted,
however, like this
\begin{python}
urd.begin('test')
urd.abort()
\end{python}
Aborted sessions are not stored in the Urd transaction log.




\section{Building Jobs}

Jobs are dispatched in Urd sessions using the \texttt{build} function.
The syntax is just the same as in the previous section that did not
use sessions.
\begin{python}
jobid = urd.build('method1', options={}, datasets={}, jobids={}, ...)
\end{python}
Here, \texttt{options}, \texttt{datasets}, and \texttt{jobids} are
optional, depending on the method to be dispatched.  In addition, a
name and a caption may be specified too
\begin{python}
jobid = urd.build('method1', name='myjob', caption='looking for something')
\end{python}
The name will override the default name, which is the name of the
method, in the Urd list.  In this case, this job will now be referred
to as \texttt{myjob} (instead of default \texttt{method1}).  This is
useful to separate jobs if the same method is used multiple times in
the same list.  A jobid to the finished job is returned upon
successful completion.



\subsection{Handling Consecutive Jobs}

Using the output jobid from the \texttt{build} function, it is
straightforward to connect jobs in series.  For example
\begin{python}
jid_filter = urd.build('filter_data', datasets=dict(source=<some input>))
jid_reduce = urd.build('reduce', datasets=dict(source=jid_filter))
\end{python}
In the example above, the first method, \texttt{filter\_data}, creates
a new dataset from its input.  This is then forwarded to the second
method, \texttt{reduce}, using the jobid reference
\texttt{jid\_filter}.

If the first method or its input data is changed, the first job will
be run again.  This will cause the jobid \texttt{jid\_filter} to
change too, which will, in turn, trig execution of the \texttt{reduce}
job.  The Urd list will contain jobids to both jobs, and all input
parameters as well, so it is clear which \texttt{filter\_data} job
that was used as input for the \texttt{reduce} job.



\subsection{Building Chained Jobs}

Using the \texttt{build\_chained} function, it is possible to build
chained jobs implicitly, like this
\begin{python}
jobid = urd.build_chained('method1', name='myjob')
\end{python}
This function takes the same options as the standard \texttt{build}
method, with the exception that name is mandatory, since it is used to
find the previous job of matching type.






\section{Urd Sessions with Dependencies}


\comment{forst en import-automata har, och sedan anvand den...}

A job may have dependencies, such as other jobs or datasets.  These
dependencies are input to the job using the corresponding arguments to
the \texttt{build} function.  Locating these jobs or datasets,
however, is exactly a design goal of Urd.  Urd implements a
\texttt{get} function that looks up jobids and dependencies from a key
that is composed of an Urd list name plus a timestamp.  There are
also, for convenience, \texttt{first} and \texttt{latest} functions to
get the first and latest job in an Urd list.

Here is an example.  Assume that we have a build script that imports
data files.  Information about the import jobs is stored in the
\texttt{import} Urd list.
\begin{python}
def main(urd):
    now = '20180403'
    urd.begin('import', now)
    jid_prev = urd.latest('import')
    urd.build('csvimport',
        options=dict(filename='log' + now + '.txt',),
        datasets=dict(previous=jid_prev)
    )
    urd.finish('import')
\end{python}
Note how the \texttt{previous} dataset is assigned from the output of
the \texttt{urd.latest} call, making the import jobs \textsl{chained}.
The call to \texttt{latest} will be recorded in the \texttt{import}
Urd list as well.

Now, assume that a method \texttt{computesomething}, uses these
imported datasets.  When dispatching \texttt{computesomething}, it
should be using the latest available \texttt{import}.  This example
shows again how the function \texttt{latest} is used for this purpose
\begin{python}
def main(urd):
    urd.begin('test')
    latest_import = urd.latest('import').joblist.jobid
    urd.build('computesomething', datasets=dict(source=latest_import))
    urd.finish('test', '20180403')
\end{python}
Two things have happened here.  First, urd has provided a jobid link
to the latest available \texttt{import} dataset.  Second, the
dependency of exactly this version of \texttt{import} to
\texttt{computesomething} is recorded in the urd list \texttt{test}
for timestamp \texttt{20180403}.  So, if there is a question in the
future which version of the \texttt{import} database that was used on
that date for the \texttt{computesomething} function, it is
immediately available from urd.

The more general form is \texttt{get}, which is shown below together
with its derived convenience-functions
\begin{python}
  urd.get('test', '20161001')
  urd.latest('test')
  urd.first('test')
\end{python}
And here is an example of running \texttt{computesomething} on \texttt{import} data
from previous month
\begin{python}
def main(urd):
  urd.begin('test')
  import = urd.get('import', '20160925').joblist.jobid
  urd.build('computesomething', datasets=dict(import=import))
  urd.finish('test', '20161025')
\end{python}



\section{Avoiding Recording Dependency}
Dependency-recording will be activated on use of the \texttt{get},
\texttt{latest}, and \texttt{first} functions.  If, for some reason,
the point is to just have a look at the database to see what is in
there, it can be done using the peek functions, \texttt{peek} and
\texttt{peek\_latest}, like this:
\begin{python}
urd.peek('test', '20161025')
urd.peek_latest('test')
urd.peek_first('test')
\end{python}
Note that this is in general not recommended.  These functions will
look up Urd lists with jobids that may be used to build new jobs, but
these dependencies will not be stored in the current Urd session,
causing a loss of continuity and visibility.



\section{More on Finding Items in Urd}
There is a \texttt{list} function that returns what lists are recorded
in the database:
\begin{pythonBEG}
  print(urd.list())
  # ['ab/test', 'ab/live']
\end{pythonBEG}
And there is also a \texttt{since} function that returns a list of all
timestamps after the input argument
\begin{pythonMID}
  print(urd.since('20161005'))
  # ["20161006", "20161007", "20161008", "20161009"]
\end{pythonMID}
The \texttt{since} is rather relaxed with respect to the resolution of
the input.  The input timestamp may be truncated from the right down
to only one digits.  An input of zero is also valid.
\begin{pythonEND}
  print(urd.since('0'))
  # ["20160101", "20161004", "20161005", "20161006", "20161007", "20161008"]
  print(urd.since('2016'))
  # ["20160101", "20161004", "20161005", "20161006", "20161007", "20161008"]
  print(urd.since('20161'))
  # ["20161004", "20161005", "20161006", "20161007", "20161008"]

  print(urd.since('2016105'))
  # ["20161006", "20161007", "20161008"]
  ...
  print(urd.since('2016105 000000'))
  # ["20161006", "20161007", "20161008"]
\end{pythonEND}



\section{Truncating and Updating}
\label{sec:trunc-update}
Since the Urd database is designed using log files, it will always
keep a consistent history of all events taken place.  It is not
possible to erase or modify old entries, but it is okay to update the
latest or set a marker in the log indicating that the list is starting
over from a certain date and everything before this marker should not
be considered anymore.

To update a list, use the \texttt{update} argument
\begin{python}
urd.begin('test', '20161025', update=True)
\end{python}
If update is True, the entry in the test list at '20161025' will be
updated, unless there has been no change.  and in order to set a
marker in the database indicating that everything before a certain
date in time should be discarded, do like this
\begin{python}
urd.truncate('test', '20160930')
\end{python}
This will rollback everything that has happened in the \texttt{test}
list back to '20160930'.  Remember, internally Urd stores the complete
history in a log file in plain text.


\section{More on Joblist and Jobtuple}

Urd is using the type joblist to keep track of successfully executed
jobs.  Each item in the joblist is of type jobtuple.  This section
will start by describing jobtuple first and then joblist.

\subsubsection{Jobtuple}

The \jobtuple type is used to group method names and corresponding
jobids.  It is basically a tuple with some extra properties, such as a
conversion of a jobtuple to \texttt{str}, which happens for example
when printing it, returns the jobid as a string.

\begin{pythonBEG}
>>> jt = JobTuple('imprt', 'jid-0')

>>> jt
('imprt', 'jid-0')
\end{pythonBEG}
as expected, and

\begin{pythonMID}
>>> jt.method
'imprt'

>>> jt.jobid
'jid-0'
\end{pythonMID}
but note that

\begin{pythonEND}
>>> print(jt)  # str and encode return jobid only
jid-0
\end{pythonEND}



\subsubsection{JobList}

\label{sec:joblist}
The \joblist is a list with add-ons for bookkeeping and finding jobs.
It stores instances of \jobtuple.  Here is an example.  First, define
a \jobtuple

\begin{pythonBEG}
>>> jt = JobTuple('imprt', 'jid-0')
\end{pythonBEG}
then define a joblist initiated with the same tuple.  Then append some
more jobs directly using the \texttt{append} method.

\begin{pythonMID}
>>> jl = JobList(jt)
>>> jl.append('learn', 'lrn-0')
>>> jl.append('imprt', 'imp-1')
\end{pythonMID}
Let's see how and what is stored in the \joblist.  The \texttt{pretty}
method is quite useful, but note that just printing the object will
show the last jobid only.

\begin{pythonMID}
>>> print(jl.pretty)
JobList(
   [  0]  imprt : jid-0
   [  1]  learn : lrn-0
   [  2]  imprt : imp-1
)

>>>  print(jl)  # jobid of latest appended JobTuple
imp-1
\end{pythonMID}

It is easy to retrieve the last job with a particular \texttt{method}
name, either by lookup or by using \texttt{find}.

\begin{pythonMID}
>>> jl['imprt']         # latest jobid with name 'imprt'
('imprt', 'imp-1')

>>> print(jl['imprt'])   # jl['imprt'] is JobTuple
imp-1
\end{pythonMID}
The Find method returns a \joblist.  Slicing also returns {\joblist}s

\begin{pythonMID}
>>> jl.find('imprt')
JobList([('imprt', 'jid-0'), ('imprt', 'imp-1')])

>>> jl[:2]
JobList([('imprt', 'jid-0'), ('learn', 'lrn-0')])
\end{pythonMID}
Looking up by index returns \jobtuple.

\begin{pythonMID}
>>> jl[0]
('imprt', 'jid-0')
\end{pythonMID}
These conveniences are also supported

\begin{pythonEND}
>>> jl.all              # list of all jobids
'jid-0,lrn-0,imp-1'

>>> jl.method           # last method
'import'

>>> jl.jobid            # last jobid
'imp-1'
\end{pythonEND}



\newpage
\section{Talking directly to Urd:  The Urd HTTP-API}

In some situations it is convenient to make calls to urd directly
without using the framework.  Urd will react to HTTP requests, so a
tool like \texttt{curl} suffice.

\noindent Show all stored lists like this
\begin{shell}
% curl http://localhost:8833/list
["ab/test"]
\end{shell}

Looking up the latest stored job in the test list
\begin{shell}
% curl http://localhost:8833/ab/test/latest
{"caption": "", "automata": "test", "user": "ab", "deps": {},
  "timestamp": "20161025", "joblist": [["method1", "test-56"],
  ["method2", "test-59"], ["method3", "test-60"]]}
\end{shell}
And see the first stored job in the test list
\begin{shell}
% curl http://localhost:8833/ab/test/first
{"caption": "", "automata": "test", "user": "ab", "deps": {},
  "timestamp": "20161025", "joblist": [["method1", "test-56"],
  ["method2", "test-59"], ["method3", "test-60"]]}
\end{shell}
See what is inside the test list stored at \texttt{20161025}
\begin{shell}
% curl http://localhost:8833/ab/test/20161025
{"caption": "", "automata": "test", "user": "ab", "deps": {},
  "timestamp": "20161025", "joblist": [["method1", "test-56"],
  ["method2", "test-59"], ["method3", "test-60"]]}
\end{shell}
And what is avaible in the test list that is more recent than \texttt{20161024}
\begin{shell}
% curl http://localhost:8833/ab/test/since/20161024
["20161025"]
\end{shell}
\begin{shell}
% curl http://localhost:8833/ab/test/since/20161026
[]
\end{shell}
