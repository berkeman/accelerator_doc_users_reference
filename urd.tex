%%%%%%%%%%%%%%%%%%%%%%%%%%%%%%%%%%%%%%%%%%%%%%%%%%%%%%%%%%%%%%%%%%%%%%%%%%%%
%                                                                          %
% Copyright (c) 2018 eBay Inc.                                             %
%                                                                          %
% Licensed under the Apache License, Version 2.0 (the "License");          %
% you may not use this file except in compliance with the License.         %
% You may obtain a copy of the License at                                  %
%                                                                          %
%  http://www.apache.org/licenses/LICENSE-2.0                              %
%                                                                          %
% Unless required by applicable law or agreed to in writing, software      %
% distributed under the License is distributed on an "AS IS" BASIS,        %
% WITHOUT WARRANTIES OR CONDITIONS OF ANY KIND, either express or implied. %
% See the License for the specific language governing permissions and      %
% limitations under the License.                                           %
%                                                                          %
%%%%%%%%%%%%%%%%%%%%%%%%%%%%%%%%%%%%%%%%%%%%%%%%%%%%%%%%%%%%%%%%%%%%%%%%%%%%

\label{chap:urd}

This chapter is the continuation of chapter~\ref{chap:urd_basic},
``Basic Build Scripting''.  Please read about build script and
joblists before proceeding.


\section{Introduction to Urd}

Urd comes into play when simple build scripting is not enough.  A wide
variety of advanced tasks can be handled without it, but the
capabilities added by Urd in terms of job organisation, storage, and
retrieval makes it possible to handle much larger and more advanced
projects while maintaining in full control.

Using Urd, a project could be separated into functionally independent
parts, and all dependencies between jobs inside as well as between
these parts is tracked by a \textsl{transaction log}.  It is possible
to re-construct the state of any project part the way it was at any
instance in time.

More formally, Urd provides two things:
\begin{itemize}
\item[1.] Separation between build scripts,
        and a way to share information about built jobs between
        different scripts (or the same script at different points in
        time).
\item[2.] A searchable transaction log database of all jobs built
together with their dependencies.
\end{itemize}
The interesting transaction log database and many other aspects of Urd
will be explained in the rest of this chapter.



\section{A Simple Use case}
Assume a project where, say, movie recommendation data is to be
analysed.  Every hour, recommendations generated during the last hour
will appear in the shape of a new log file.  The project is using two
build scripts:
\begin{itemize}
\item[] The first build script is used to look for new files, and
  import and chain them as they appear.  For each new file imported,
  the build script will tell the Urd server the timestamp of the file
  as well as a list of all created jobs associated with that file.

\item[] The second build script is used for the data analysis work,
  and is perhaps run less regularly.  This script needs to know the
  job for the latest imported file, and this is a straightforward
  thing to ask Urd.  All analysis jobs are also stored in Urd together
  with a corresponding timestamp.
\end{itemize}
Here, Urd is used to forward information about executed jobs from
first build script to the second.  In this sense, Urd
provides \textsl{isolation} by message passing.

Urd can also be used to tell which input data that was used for a
particular data analysis job.  When querying Urd about a data analysis
job, it will respond with information about those jobs \textsl{as well
as} information about all Urd queries that was necessary for the jobs
to run.  This information is stored automatically in the build script
and it is there to ensure transparency and reproducibility.



\section{Urd Sessions and Lists}
A simple file import script will be used as example in this section:
\begin{python}
def main(urd):
    urd.build('csvimport', options=dict(filename='txn1.txt'))
\end{python}
In order to use this import job in a future context,
a \textsl{session} is created by wrapping the code by
the \texttt{urd.begin()} and \texttt{urd.finish()} functions, like this
\begin{python}
def main(urd):
    urd.begin('import/txn', '2018-05-03')
    urd.build('csvimport', options=dict(filename='txn1.txt'))
    urd.finish('import/txn')
\end{python}
Everything that happens between \texttt{begin()} and \texttt{finish()}
makes up the session.  The \texttt{finish()} function makes sure that
the session is stored permanently to disk for future reference.  In
this case, the session can be retrieved knowing its \textsl{list} name
\begin{verbatim}
    import/txn
\end{verbatim}
and its \textsl{timestamp}
\begin{verbatim}
    2018-05-03
\end{verbatim}
The list identifier is composed of two parts, \texttt{<user>/<list>}, where
\texttt{<user>} is for authorisation purposes.  Each user can
have any number of lists, but only the correct user may write to them,
as will be explained later.

The next sections will explain how to search the Urd database for
matching sessions in various ways.



\section{A First Urd Query}
The list created in the previous section can now be used by other
build scripts.  For example, here is a build script that does some
processing on the previously imported file
\begin{python}
def main(urd):
    urd.begin('process/test')

    import_session = urd.latest('import/txn')

    import_timestamp = import_session.timestamp
    import_job       = import_session.joblist['dataset_type']

    urd.build('process', datasets=dict(source=import_job))

    urd.finish('process/test', import_timestamp)
\end{python}
The first thing that happens is that all processing is covered in a
session named \texttt{process/text}.  At \texttt{begin()}, the
timestamp is still unknown, it is to be set to the same timestamp as
the import job has.

The script is then retrieving the recently created urd session stored
in list \texttt{import/txn}.  Two things are extracted from this data,
the \textsl{timestamp} and the \textsl{joblist}.  The timestamp will
be used for this session as well, to indicate that processing is based
on data with that particular timestamp.  A reference to
the \texttt{csvimport} job is then extracted from the joblist and fed
to the \texttt{process} job as an input dataset parameter. (For
information about the \texttt{joblist} class, see
section~\ref{sec:clumsy_dot_jobid}.)



\section{The Contents of the Stored Session}
\label{sec:urd_item}

Calling \texttt{urd.finish()} will update the Urd database with the
contents of the current \textsl{session}.  Each session is addressable
using a \textsl{list} name (in the format \texttt{<user>/<list>}) and
a \textsl{timestamp}.  Session data is stored internally in
the JSON format, and in build scripts it appears as
Python \texttt{dict}s.  The example presented earlier in this chapter
may have been recorded similarly to this
\begin{json}
{
    "user": "processing",
    "automata": "test",
    "timestamp": "2018-05-03",
    "caption": "",
    "joblist": [
        [
            "process",
            "TEST-37"
        ]
    ],
    "deps": {
        "import/txn": {
            "timestamp": "2018-05-03",
            "caption": "",
            "joblist": [
                [
                    "csvimport",
                    "TEST-34"
                ]
            ],
        }
    },
}
\end{json}

\noindent (This example states that at timestamp \texttt{2018-05-03} in list \texttt{processing/test},
there exists a \texttt{process} job \texttt{TEST-34} that
used a \texttt{csvimport} job \texttt{TEST-34}.  This job
also exists in the urd list \texttt{import/txn} at
timestamp \texttt{2018-05-03}.)

The most important keys are
\starttabletwo
\RPtwo \texttt{timestamp} & Timestamp of session\\[1ex]
\RPtwo \texttt{caption} & A caption\\[1ex]
\RPtwo \texttt{user/automata} & Name of Urd list\\[1ex]
\RPtwo \texttt{joblist} & An object of type \texttt{joblist}, containing all jobs built in the session.  For more information, see~\ref{sec:joblist}.\\[1ex]
\RPtwo \texttt{deps} & A dictionary of dependencies from \texttt{user/automata} to urd sessions: \mintinline{python}@{'user/automata': session}@.\\[1ex]
\stoptabletwo




\section{Urd Sessions:  \texttt{begin()} and \texttt{finish()}}

There are a number of options associated with a session, as shown
here,
\begin{python}
urd.begin(urdlist, timestamp, caption=None, update=False)
urd.finish(urdlist, timestamp, caption=None)
\end{python}
and the following applies
\starttabletwo
\RPtwo \texttt{urdlist} & is the name of the Urd list, and the same
  \texttt{urdlist} must be specified in both \texttt{begin()}
  and \texttt{finish()}.

  The \texttt{urdlist} is specified as \texttt{<user>/<list>}, where
  the \texttt{<user/>} part is optional.  The \texttt{user} string
  is also for authentication, and must correspond to the
  current \texttt{URD\_AUTH} settings, see section~\ref{sec:urd_setup}.\\[2ex]

\RPtwo \texttt{timestamp} & is \textsl{mandatory}, but could be set in either
  \texttt{begin()}, \texttt{finish()}, or both.  \texttt{finish()}
  will override \texttt{begin()}.\\[2ex]

\RPtwo \texttt{caption} & is \textsl{optional}, and can be set in either
  \texttt{begin()} or \texttt{finish()}.  \texttt{finish()}
  will override \texttt{begin()}.\\[2ex]

\RPtwo \texttt{update} &  If set to \pyTrue, the last item in the list may be updated.
  This option will be discussed in section~\ref{sec:trunc-update}.\\[2ex]
\stoptabletwo

The Urd transaction database will be written to only when
the \texttt{finish()} function is called.  Before
calling \texttt{finish()}, nothing is stored, and it is perfectly okay
to omit \texttt{finish()} to avoid storage or during development work.




\subsection{What if a Build Script is Run Again?}
Running a build script for the first time will cause jobs to be built.
The second time the same script is run, the Accelerator will look up
already built jobs and immediately return job references instead of building
anything.  As long as there are no changes, re-defining Urd sessions
that already exists is not a problem - they will be silently ignored.
But if there are any discrepancies, such as a job being rebuilt, Urd
will complain and refuse to store the differing session.

Normally, a build script can be written in such a way that re-running
it will be consistent will the Urd database and everything is fine.  A
mismatch with Urd is then an indication of some error.  But there are
cases when re-writing Urd history is the desired option, and this will
be discussed in section~\ref{sec:trunc-update}.







\section{Timestamp Resolution}

Timestamps may be specified in various resolution depending on the
application.  The full time format is (See Python's \texttt{datetime}
module for explanation.)
\begin{shell}
'%Y-%m-%dT%H:%M:%S'
\end{shell}
A specific timestamp could be shorter than the above specification in
order to cover wider time ranges.  The following examples cover all
possible cases.
\begin{python}
'2016-10-25'               # day resolution
'2016-10-25T15'            # hour resolution
'2016-10-25T15:25'         # minute resolution
'2016-10-25T15:25:00'      # second resolution
\end{python}
%The timestamp resolution is based in experience from a number of data
%science projects, where most data had some relation to time.  Also,
%machine learning models generated by the Accelerator could be
%timestamped with the creation time.



\section{Finding Items in Urd}

There are several ways to find stored sessions.  This section will
describe the \texttt{get()}, \texttt{first()}, and \texttt{latest()}
function calls.  For any of these calls to work, they have to be
issued from \textsl{within} a session, i.e.\ after a \texttt{begin()}
call.  Otherwise Urd would not be able to record all session
dependencies.

More ways to find sessions is described in
section~\ref{sec:more_urd_search}.



\subsection{Finding an Exact or Closest Match:  \texttt{get()}}
\label{sec:urd_get}

The \texttt{get()} function will return the single session, if
available, corresponding to a specified list and timestamp, like this
\begin{python}
urd.begin('ab/anotherlist')
urd.get("ab/test", "2018-01-01T23")
\end{python}
The timestamp must match exactly for an item to be returned.  This
strict behaviour can be relaxed by prefixing the timestamp with one of
\begin{itemize}
\item[] ``\texttt{<}'', ``\texttt{<=}'', ``\texttt{>}'', or ``\texttt{>=}''.
\end{itemize}
For example
\begin{python}
urd.get("ab/test", ">2018-01-01T01")
\end{python}
may return an item recorded as \texttt{2018-01-01T02}.  Relaxed
comparison is performed ``from left to right'', meaning that
\begin{python}
urd.get("ab/test", ">20")
\end{python}
will match the first recorded session in a year starting with
``\texttt{20}'', while
\begin{python}
urd.get("ab/test", "<=2018-05")
\end{python}
will match the latest timestamp starting with ``\texttt{2018-05}'' or less,
such as ``\texttt{2018-04-01}'' or ``\texttt{2018-05-31T23:59:59}''.

If there is no matching item, the \texttt{get()}-call will return
an \textsl{empty session}, i.e.\ something like this
\begin{python}
{'deps': {}, 'joblist': JobList([]), 'caption': '', 'timestamp': '0'}
\end{python}




\subsection{Finding the Latest Session:  \texttt{latest()}}
The \texttt{latest()} call will return the session with most recent
timestamp.  Example
\begin{python}
urd.begin('ab/anothertestlist')
urd.latest('ab/test')
\end{python}
will return a complete item like this
\begin{shell}
{'automata': 'test', 'caption': '', 'user': 'ab', 'deps': {}, \
 'joblist': JobList([('example', 'TEST-34')]), 'timestamp': '2018-05-06'}
\end{shell}
If the list is non-existing or empty, an \textsl{empty session} will be returned.



\subsection{Finding the first item:  \texttt{first()}}
The \texttt{first()} function works similarly to \texttt{latest}, but
will instead return the session with the \textsl{oldest} timestamp.





\section{Aborting an Urd Session:  \texttt{abort()}}

When an Urd session is initiated, a new session cannot be started
until the current session has finished.  A session may therefore be
aborted, and the \texttt{abort()} function is used for this, like so
\begin{python}
urd.begin('test')
urd.abort()
\end{python}
Similar to unfinished sessions, aborted sessions will not be stored in
the Urd transaction log.



\section{Truncating and Updating}
\label{sec:trunc-update}
Since the Urd database is designed using log files, it will always
keep a consistent history of all events taken place.  It is not
possible to erase or modify old entries, but it is okay to update the
latest item, or set a marker in the log indicating that the list is
starting over from a certain date and everything before this marker
should not be considered anymore.  This makes it possible to both
keeping the full history \textsl{and} being able to rewrite it.  There
is full transparency and reproducibility -- all sessions before an
update or restart marker are always kept in the Urd log file.



\subsection{Updating the last item}
To update the last item in a list, set the \texttt{update} argument
to \pyTrue
\begin{python}
urd.begin('test', '2016-10-25', update=True)
\end{python}
If update is \pyTrue, the entry in the test list at '2016-10-25' will
be updated, unless the new information is equivalent.
The \texttt{update()} call will simply add a new line to the Urd log
database, and if the timstamp is the same as the previous entry, the
new entry will be selected.



\subsection{Truncating a list}
In order to insert a marker in the database indicating that everything
before a certain timestamp should be discarded, use
the \texttt{truncate()} function like this
\begin{python}
urd.truncate(ab/'test', '2016-09-30')
\end{python}
This will rollback everything that has happened in
the \texttt{ab/test} list back to '2016-09-30'.  There is also a
special case,
\begin{python}
urd.truncate('ab/test', 0)
\end{python}
that will erase all items from memory and cause the list to start over
again.  Remember, internally Urd stores the complete history in a log
file in plain text.  Files can only be appended to, nothing is ever
removed.  It is always possible to recover any old result or
processing state.


\subsection{Truncation Consequences:  Ghosts}
When a list is truncated, all items after a specified timestamp are
made invisible.  Assuming that another list has stored a dependency of
an item that is truncated, the jobs in this list are now without
dependencies that can be looked up.  We call them ``ghosts''.  Ghosts
cannot be looked up in Urd, but they are still in the database, marked
as ghosts.



\section{Avoiding Recording Dependency}
Dependency-recording will be activated on use of the \texttt{get()},
\texttt{latest()}, and \texttt(){first} functions.  If, for some reason,
the point is to just have a look at the database to see what is in
there, it can be done using the \textsl{peek}
functions, \texttt{peek()}, \texttt{peek\_first()}, and
\texttt{peek\_latest()}, like this:
\begin{python}
urd.peek('test', '2016-10-25')
urd.peek_latest('test')
urd.peek_first('test')
\end{python}
Note that this is in general not recommended.  These functions will
look up Urd lists containing jobs that may be used to build new jobs, but
these dependencies will not be stored in the current Urd session,
causing a loss of continuity and visibility.



\section{More Search Functions}
\label{sec:more_urd_search}

There are two more functions for finding information in the Urd
database: \texttt{list} and \texttt{since}.


\subsection{Listing all urd lists:  \texttt{list()}}
The \texttt{list()} function will return a list of all lists recorded in
the database:
\begin{python}
print(urd.list())
\end{python}
may show something like
\begin{shell}
['ab/test', 'ab/live']
\end{shell}


\subsection{Listing all Items After a Specific Timestamp:  \texttt{since()}}
The \texttt{since()} function is used to extract lists of \textsl{timestamps}
corresponding to recorded session.  In its most basic form, it is called
with a timestamp like this
\begin{python}
urd.since('ab/test', '2016-10-05')
\end{python}
which returns a list with all existing timestamps more recent than the
one provided
\begin{shell}
['2016-10-06', '2016-10-07', '2016-10-08', '2016-10-09', '2016-10-09T20']
\end{shell}
The \texttt{since} is rather relaxed with respect to the resolution of
the input.  The input timestamp may be truncated from the right down
to only one digits.  An input of zero is also valid.  For example,
these are all valid
\begin{python}
urd.since('ab/test', '0')
urd.since('ab/test', '2016')
urd.since('ab/test', '2016-1')
urd.since('ab/test', '2016-10-05')
urd.since('ab/test', '2016-10-05T20')
urd.since('ab/test', '2016-10-05T20:00:00')
\end{python}



\section{Building Jobs: \texttt{build()}}
\label{sec:urd_build}

Jobs are dispatched in Urd sessions using the \texttt{build} function.
Here is the complete call with all possible parameters.
\begin{python}
job = urd.build(
    method,
    options={},    datasets={},    jobs={},
    name='',       caption='',
    workdir=None
)
\end{python}
Explanation of \texttt{build} parameters:
\starttabletwo
\RPtwo \texttt{method} & Name of method to build.  Enter \texttt{test}
    here if the method filename is \texttt{a\_test.py}.\\
    
\RPtwo \texttt{options=\{\}} & a \texttt{dict} of options to the method.
    This overrides options defined in the method itself, but adding
    options not prototyped in the method is \textsl{not} allowed.\\

\RPtwo \texttt{jobs=\{\}} & a \texttt{dict} of jobs to the method.
    It is possible to specify a list of jobs like this
\begin{python}
jobs=dict{alljobs=[job1, job2,...]}
\end{python}
\\

\RPtwo \texttt{datasets=\{\}} & a \texttt{dict} of datasets to the method.
    Datasets may be lists too, just like \texttt{jobs} above.\\

\RPtwo \texttt{workdir=None} & If specified, the job will be built in
    this workdir, assuming the workdir is specified in the
    configuration file as either source or target.\\

\RPtwo \texttt{name} & A string associated with the job.  Use it
    to distinguish several jobs created from the same method.\\
    
    \texttt{caption} & A caption string.  For decorative purposes
    only, this has no practical use.\\
\stoptabletwo

The \texttt{build()} function will only build a job when it has to,
otherwise it will just return a job reference to an existing matching
job.  In order to match, an existing job must have
\begin{itemize}
\item[-] exactly the same source code, i.e.\ the \textsl{hash} of the source code must match,
\item[-] exactly the same options, datasets, and jobs.
\end{itemize}
If the source code is changed, a job rebuild can be prevented using
the \texttt{equivalent\_hashes} variable as explained in
section~\ref{sec:equivalent_hashes}.



\section{Changing workdir:  \texttt{set\_workdir()}}
The target workdir specified in the configuration file is the only
workdir that is written to by default.  Any other workdir is read
only.  This behaviour can be overridden, either
\begin{itemize}
\item[] per job, using the \texttt{workdir=...} option to \texttt{urd.build} as shown in section~\ref{sec:urd_build}, or
\item[] using \texttt{urd.set\_workdir()}.
\end{itemize}
The latter,
\begin{python}
def main(urd):
    urd.set_workdir(<workdir>)}
\end{python}
will set the workdir for all coming \texttt{build} calls in the
current build script.  It can still be overridden using
the \texttt{workdir=} option to \texttt{urd.build}.



\section{Profiling a Build Script:  \texttt{print\_profile()}}
There is a helper function in the urd object that can be used to print
profiling information.  The following example is self-explanatory
\begin{python}
def main(urd):
    ...
    urd.print_profile()
\end{python}
This will print execution times for all jobs in the session
to \texttt{stdout}.  It may for example look like this
\begin{verbatim}
Time per method:
   color2        23.7 seconds  (25%)
   csvexport     17.5 seconds  (18%)
   lowpass2      15.6 seconds  (17%)
   newcol        14.4 seconds  (15%)
   black          5.7 seconds  (6%)
   colimage       5.4 seconds  (6%)
   sync           4.7 seconds  (5%)
   clamp          3.8 seconds  (4%)
   dataset_type   2.7 seconds  (3%)
   csvimport      1.4 seconds  (1%)
Total time 94.8 seconds
\end{verbatim}
The methods are sorted by execution time, top to bottom.



\section{Passing Flags from the Command Line}
It is possible to add a comma separated list of flags to the run command
like this
\begin{python}
ax run [script] --flags=verbose,skiptest
\end{python}
The flags will appear in the \texttt{urd}-object like this
\begin{python}
def main(urd):
    if 'verbose' in urd.flags:
       print('verbosity')
\end{python}



\section{profile\_jobs}
Returns a float corresponding to the total execution time (in seconds)
of an input list of jobs.
\begin{python}
from accelerator.automata_common import profile_jobs
...
print(profile_jobs(urd.joblist))
\end{python}



\section{The Urd HTTP-API}
Urd can be accessed directly without using the Accelerator by calling
its HTTP API.  These calls are easy to use and adds transparency since
the database contents can be peeked without writing any programs.
Here is a list of all API endpoints
\begin{verbatim}
  list
  <user>/<list>/first
  <user>/<list>/latest
  <user>/<list><timestamp>
  <user>/<list>/since/<timestamp>
\end{verbatim}
All calls return data in the JSON format.  The Accelerator comes with
a built in ``\texttt{curl}'' command that is designed for these calls,
and it is used like this
\begin{shell}
% ax curl <endpoint>
\end{shell}
But any standard tool such as the famous \texttt{curl} program works
too, for example like this
\begin{shell}
% curl http://localhost:8123/list
\end{shell}


\subsection{The \texttt{list} endpoint}
To show all stored lists issue
\begin{shell}
% ax curl list
["ab/test"]
\end{shell}
All available lists are returned in a typed JSON list.


\subsection{The \texttt{since} endpoint}
The \texttt{since} endpoint is used to get a list of all entries more
recent than a timestamp.  For example, to see information added
after \texttt{2016-10-24}, do
\begin{shell}
% ax curl ab/test/since/2016-10-24
["2016-10-25"]
\end{shell}
\begin{shell}
% ax curl ab/test/since/2016-10-26
[]
\end{shell}
Results are in JSON list format.


\subsection{The \texttt{first} and \texttt{latest} endpoints}
Looking up the latest stored job in the \texttt{ab/test} list
\begin{shell}
% ax curl ab/test/latest
{"caption": "", "automata": "test", "user": "ab", "deps": {},
  "timestamp": "2016-10-25", "joblist": [["method1", "test-56"],
  ["method2", "test-59"], ["method3", "test-60"]]}
\end{shell}
And see the first stored job in the test list
\begin{shell}
% ax curl ab/test/first
\end{shell}
works similarly.  The returned data is an Urd item, described in
section~\ref{sec:urd_item}, in JSON format.


\subsection{The \texttt{get} endpoint}
Actually, there is no explicit \texttt{get} endpoint.  Instead, the
API is just called by the name of the list and a timestamp.  For
example, to see what is inside the test list stored
at \texttt{2016-10-25}
\begin{shell}
% ax curl ab/test/2016-10-25
{"caption": "", "automata": "test", "user": "ab", "deps": {},
  "timestamp": "2016-10-25", "joblist": [["method1", "test-56"],
  ["method2", "test-59"], ["method3", "test-60"]]}
\end{shell}
The timestamp may be truncated to the right, and prefixed
by \texttt{>}, \texttt{>=}, \texttt{<}, and \texttt{<=}, just as
described in section~\ref{sec:urd_get}.  \textsl{Make sure to quote
the request if these characters are used in a call from the shell.}



\section{Urd Internals}
Urd can be accessed by a large number of clients.  Each client may add
to or truncate any list at any time.  In order to avoid race
conditions and make the database deterministic, all \texttt{add}-
and \texttt{truncate}-requests appears in a sequential manner to the
Urd server.  Each request is assigned with an unique timestamp, and
stored in the requested list.

When Urd is restarted, it reads all the database files, and sorts all
rows in order of the receive timestamp.  Thereafter, each row is
applied in increasing time order to the internal, RAM-based database.
Due to the unique timestamping, the result is a deterministic replica
of the previous run.
