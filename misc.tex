%%%%%%%%%%%%%%%%%%%%%%%%%%%%%%%%%%%%%%%%%%%%%%%%%%%%%%%%%%%%%%%%%%%%%%%%%%%%
%                                                                          %
% Copyright (c) 2019-2021 Anders Berkeman                                  %
%                                                                          %
% Licensed under the Apache License, Version 2.0 (the "License");          %
% you may not use this file except in compliance with the License.         %
% You may obtain a copy of the License at                                  %
%                                                                          %
%  http://www.apache.org/licenses/LICENSE-2.0                              %
%                                                                          %
% Unless required by applicable law or agreed to in writing, software      %
% distributed under the License is distributed on an "AS IS" BASIS,        %
% WITHOUT WARRANTIES OR CONDITIONS OF ANY KIND, either express or implied. %
% See the License for the specific language governing permissions and      %
% limitations under the License.                                           %
%                                                                          %
%%%%%%%%%%%%%%%%%%%%%%%%%%%%%%%%%%%%%%%%%%%%%%%%%%%%%%%%%%%%%%%%%%%%%%%%%%%%

\section{Workdirs}
Jobs are stored in \textsl{workdirs}.  Workdirs are defined in the
Accelerator's configuration file, where at least one workdir must be
specified.

By default, the only workdir that is written to is the target workdir,
while all other defined workdirs are for reading.  It is possible to
override this, however, by setting the \mintinline{python}|workdir=|
option in the \texttt{urd.build()} call, see
section~\ref{sec:urd_build}.

Jobdirs are stored in the workdir by the server, and jobdirs will
inherit the workdir name and add a suffix that is an incremental job
counter.  Here is an example of a workdir named \texttt{test}, that
contains three jobdirs.
\begin{shell}
test/
    .slices.conf
    test-0/
    test-1/
    test-2/
    test-LATEST -> test-2
\end{shell}
The \texttt{.slices.conf} file contains the number of slices used for
the workdir.  The link \texttt{<workdir>-LATEST} is always pointing to
the last jobdir created.  This is useful for example when iteratively
testing a method and accessing its data for example for plotting
purposes.  Each new build of the (modified) method will create a new
job, and the link will always point to the most recent
version.


\subsection{Creating a Workdir}
If a workdir defined in the configuration file does not exist on disk
at the stated location, the server will exit and print an error
stating that a directory is missing.  The first time the server
encounters a new directory it will initialise it in accordance with
the configuration file.  So, new workdirs are created by adding them
to the configuration file \textsl{and} creating the corresponding
directories.  The Accelerator will then initiate these directories on
the next startup.

The initiation process creates a file named \texttt{.slices.conf} that
indicates that the directory is now a workdir.  This file contains the
number of slices that is used for the workdir.



\section{Status (Progress) Reporting}
During job building, it is possible to press \texttt{C-t},
i.e.\ \texttt{Ctrl} + \texttt{t} simultaneously, in the \texttt{run}
shell to get status information.  The built in status will report the
processing state, if it is in \prepare, \analysis, or \synthesis.
Iterators report which dataset (perhaps in a chain) that is currently
being iterated, and the \texttt{blob} functions report status of file
pickling.

The \texttt{status} module makes it possible to insert
status reporting into any method.  For more information about status
reporting, see section~\ref{sec:statusreporting}.



\section{Generating Progress Messages:  the \texttt{status} Module}
\label{sec:statusreporting}
The status module is used by the Accelerator to report processing
state.  It is also used by various functions to report iterator and
file access progress.  Status messages are presented in
the \texttt{run} shell by pressing \texttt{C-t}, i.e.\
the \texttt{Ctrl} and \texttt{t} keys simultaneously.

The status module can be used to write progress and status messages
for any function.  Here is an example of how to use the status module
\begin{python}
from accelerator import status
...
def analysis(sliceno):
    msg = "reached line %d already!"
    with status(msg % (0,) as update:
        for ix, data in enumerate(datasets.source.iterate(sliceno, 'data')):
            if ix % 1000000 == 0:
                update(msg % (ix,))
\end{python}
In the example above, the status message will be updated once every
million iteration.  By pressing \texttt{C-t} during its execution, the
user will get a message telling how many lines the iterator has
reached.



\section{Working with Relative Paths}
In some situations, like importing data from files, it is convenient
to store the absolute path of the files as a configuration parameter
and then work only with relative paths in the source code.  This has
two advantages.
\begin{itemize}
\item[] First, it makes it possible to move input files around without
forcing a re-build of the import jobs, and
\item[] second, absolute paths will not be stored in the source code.
\end{itemize}
In order to make use of relative paths, store the ``system dependent''
left part of the path in the Accelerator's configuration file.  There
are two variables in the configuration file that can be used for this,
and they have different purposes.  The \texttt{input\_directory}
variable is intended for reading input files, and
the \texttt{result\_directory} is intended for writing output.  See
the following subsections for details.


\subsection{The \texttt{input\_directory}}
\label{sec:INPUT_DIR}
Files stored in the \texttt{input\_directory} can be accessed using
the \texttt{CurrentJob} object either like this
\begin{python}
def synthesis(job)
    fname = job.input_filename('afile.csv')
\end{python}
or like this
\begin{python}
def synthesis(job)
    with job.open_input('afile.csv', 'rt') as fh:
        for line in fh:
            ...
\end{python}
Since the \texttt{input\_directory} is defined in the Accelerator's
configuration file, the input directory can be moved and re-defined
without having to re-execute any built jobs.



\subsection{The \texttt{result\_directory}}
\label{sec:RESULT_DIR}
Storing files in job directories is great for transparency, but in
some cases it is convenient to keep a reference to result files in a
common place.  This is the purpose of the \texttt{result\_directory}.
However, storing files in this directory directly would void the
connection to the job that created it.  A better way is to keep the
file in the job directory and create a symbolic (soft) link to it in
the result directory.  This functionality is implemented in the
\texttt{job.link\_result()} function that is available to
\textsl{build scripts}, and used like this
\begin{python}
def main(urd):
    job = urd.build('somejob')
    job.link_result()  # links the job's "result.pickle" to result_directory
    job.link_result('result.txt')
    job.link_result('result.txt', linkname='result_from_somejob.txt')
\end{python}
Note that this only works in build scripts.  The
\texttt{job.link\_result(filename)} call will create a symbolic link
named \texttt{filename} in the \texttt{result\_directory} defined in
the Accelerator's configuration file.  The link will point to the
original file in the job directory, and will be silently replaced if
it already exists.



\section{Concurrency}
\label{sec:concurrency}
By default, a job containing the \texttt{analysis()} function will be
forked into \texttt{slices} parallel processes, where \texttt{slices}
is specified in the Accelerator's configuration file.  A standard
method like \texttt{dataset\_sort} will sort in parallel in all slices
for maximum performance, but for large datasets and systems with
little RAM, this could lead to running out of memory.  A solution to
this problem is to limit the number of allowed parallel processes.

This could be done either in the \texttt{build} call, using the
\texttt{concurrency=} parameter, or on the command line as an option
to the \texttt{ax run} command.  In both cases, the limit could be set
to all methods, or it could be set to a specific method only.

If \texttt{concurrency} is set to a number, like this
\begin{python}
  concurrency=3
\end{python}
the number of parallel processes is limited to this number for all
methods.  Alternatively, concurrency can be specified for a single
method like this
\begin{python}
  concurrency="dataset_sort=3"
\end{python}

