\section{The \texttt{Job} and \texttt{CurrentJob} Classes}
\label{sec:classes:job}
The \texttt{Job} and \texttt{CurrentJob} classes are similar, but used
in different contexts:
\begin{itemize}
\item[] The \texttt{Job} class is used to represent and operate on
  \textsl{existing} jobs.  An object of this class is returned from
  job \texttt{build()} calls as well as when retreiving jobs from Urd
  or a \texttt{JobList} object.
\item[] The \texttt{CurrentJob} class is an extension that provides
  mechanisms for operations performed while a job is
  \textsl{executing}, such as saving files to the job's jobdir.
\end{itemize}
The classes are derived from the \texttt{str} class, and objects of
these classes decay to (unicode) strings when pickled.  The following
attributes are available on both the \texttt{Job} and
\texttt{CurrentJob} classes:
\starttabletwo
\texttt{chain()} & Job chain\\
\texttt{dataset()} & Return a named dataset from the job.\\
\texttt{datasets} & List of datasets in job.\\
\texttt{filename()} & Return absolute path to a file in a job.\\
\texttt{files()} & Return list of files created by job.\\
\texttt{json\_load()} & Load a json file from the job's directory.\\
\texttt{link\_result()} & Create a soft link from a file in a job's directory to \texttt{result\_directory}.\\
\texttt{load()} & Load a pickle file from the job's directory.\\
\texttt{method} & The job's method.  This can be overriden by \texttt{name=} if job instance is output from Urd or a \texttt{build()} call.\\
\texttt{number} & The job number as an \texttt{int}.\\
\texttt{open()} & Similar to standard \texttt{open}, use to open files.\\
\texttt{output()} & Return what the job printed to \texttt{stdout} and \texttt{stderr}.\\
\texttt{params} & Return a dict corresponding to the file \texttt{setup.json} for this job.\\
\texttt{path} & The filesystem directory where the job is stored.\\
\texttt{post} & Return a dict corresponding to the job's \texttt{post.json}.\\
\texttt{withfile()} & A \texttt{JobWithFile} with this job.\\
\texttt{workdir} & The workdir name (the part before \texttt{-number} in the jobid).\\
\stoptabletwo

\noindent In addition, the \texttt{CurrentJob} class has these unique attributes:
\starttabletwo
\texttt{datasetwriter()} & Returns a DatasetWriter object.  See documentation for \texttt{Dataset.DatasetWriter()}, section~\ref{sec:classes:datasetwriter}.\\
\texttt{input\_filename()} & Full filename of file in \texttt{input\_directory}.\\
\texttt{json\_save()} & Store a json file in the current job directory.\\
\texttt{open()} & Added extra \texttt{temp} argument.\\
\texttt{open\_input()} & Open a file in \texttt{input\_directory}.\\
\texttt{register\_file()} & Record a file produced by this job.\\
\texttt{save()} & Store a pickle file in the current job directory.\\
\stoptabletwo
\noindent Detailed description of the functions, where neccessary, follows.


\subsection{\texttt{Job.chain()}}
This works like \texttt{dataset.chain()}, but for chains of \texttt{job}s.
\starttable
\texttt{length} & \texttt{-1} & Length of chain.\\
\texttt{reverse} & \pyFalse & Reverse direction of chain.\\
\texttt{stop\_job} & \pyNone & Chain to here.\\
\stoptable


\subsection{\texttt{Job.dataset()}}
%\begin{leftbar}
\starttable
\texttt{name} & \texttt{default} & \\
\stoptable
Get a dataset instance from a job.
%\end{leftbar}


\subsection{\texttt{Job.files()}}
\starttable
\texttt{pattern} & \texttt{''} & Return only files matching \texttt{pattern}\\
\stoptable

This method returns a list of all filenames corresponding to files
created by the job using the functions \texttt{CurrentJob.open()},
\texttt{CurrentJob.save()}, or \texttt{CurrentJob.json\_save()}.  The
list can be filtered using the \texttt{pattern} option.  Filtering is
based on Python's \texttt{fnmatch}-functionality.


\subsection{\texttt{Job.filename()}}
%\begin{leftbar}
\starttable
\texttt{filename} & \textsl{Mandatory} & Name of file in job directory.\\
\texttt{sliceno}  & \pyNone & Set to current slice number if sliced, otherwise \pyNone.\\
\stoptable
Return the absolute (full path) filename to a file stored in the job.
If the file is sliced, a particular slice file can be retrieved using
the \texttt{sliceno} parameter.  Sliced files are described in section~\ref{sec:sliced_files}.
%\end{leftbar}


\subsection{\texttt{Job.input\_filename()}}
Return the full filename of a file stored in the \texttt{input\_directory}.
\starttable
\texttt{filename} & \textsl{Mandatory} & Name of file.\\
\stoptable


\subsection{\texttt{Job.open\_input()}}
Open a file in the \texttt{input\_directory}.
\starttable
\texttt{filename} & \textsl{Mandatory} & Name of file.\\
\texttt{mode} & \texttt{r} & Open file in this mode, see Python's \texttt{open()}\\
\texttt{encoding} & \pyNone & Same as Python's \texttt{open()}\\
\texttt{errors} & \pyNone & Same as Python's \texttt{open()}\\
\stoptable


\subsection{\texttt{Job.json\_load()}}
%\begin{leftbar}
\starttable
\texttt{filename} & \texttt{result.json} & Name of file.\\
\texttt{sliceno} & \pyNone & \\
%\texttt{unicode\_as\_utf8bytes} & \texttt{PY2} & \\
\stoptable
Load a file from a job, in JSON format.
%The \texttt{unicode\_as\_utf8bytes} flag is active in Python2 in order
%to let strings decode as bytes.  Override this for unicode.  In
%Python3, strings decode as unicode.
%\end{leftbar}


\subsection{\texttt{Job.load()}}
%\begin{leftbar}
\starttable
\texttt{filename} & \texttt{result.pickle} & \hspace{2ex}Name of file.\\
\texttt{sliceno} & \pyNone & \\
\texttt{encoding} & \texttt{bytes} & \\
\stoptable
Load a file from a job in Python's pickle format.
%\end{leftbar}


\subsection{\texttt{Job.open()}}
%\begin{leftbar}
\starttable
\texttt{filename} & \textsl{Mandatory} & Name of file.\\
\texttt{mode} & \texttt{r} & Open file in this mode, see Python's \texttt{open()}\\
\texttt{sliceno} & \pyNone & Read or write sliced files.\\
\texttt{encoding} & \pyNone & Same as Python's \texttt{open()}\\
\texttt{errors} & \pyNone & Same as Python's \texttt{open()}\\
\texttt{temp} & \pyNone & Control file persistence.  See text.\\
\stoptable
This is a wrapper around the standard \texttt{open} function with some
extra features.  Note that
\begin{itemize}
\item[--]  \texttt{Job.open()} can only read files, not write
them, and therefore ``\texttt{r}'' flag must be set.
\item[--]  \texttt{CurrentJob.open()} can both read and write.
\item[--]  \texttt{CurrentJob.open()} must be used as a context manager,
like this
\begin{python}
with job.open(...) as fh:
    ....
\end{python}
\item[--]  \texttt{CurrentJob.open()} can use the \texttt{temp} flag to
modify the persistence of written files.
\item[--] \texttt{CurrentJob.open()} will register the file to the
  current \texttt{job}.
\end{itemize}
The \texttt{temp} argument is used to control the persistence of files
written using \texttt{.open()}.  This is useful mainly for debug
purposes, and explained in section~\ref{sec:debugflag}.  Sliced files
are described in section~\ref{sec:slicedfiles}.



\subsection{\texttt{Job.output()}}
\starttable
\texttt{what} & \pyNone & Which functions to return output from.\\
\stoptable
Get everything a job has printed to \texttt{stdout}
and \texttt{stderr} in a string variable.  The parameter \texttt{what} can be set to
\begin{itemize}
\item[] \pyNone, which returns everything,
\item[] \texttt{prepare}, which returns everything from \texttt{prepare},
\item[] \texttt{analysis}, which returns everything from \texttt{analysis},
\item[] \texttt{synthesis}, which returns everything from \texttt{synthesis}, or
\item[] a number, which returns output from the corresponding \texttt{analysis} slice.
\end{itemize}
  


\subsection{\texttt{Job.register\_file()}}
Register a file to the running \texttt{job}.  This happens
automatically when using \texttt{Job.open()}, but if the file was
produced in a way where that is not practical this can be used to
manually register the file.
\starttable
\texttt{filename} & \textsl{Mandatory} & Name of file.\\
\stoptable


\subsection{\texttt{Job.withfile()}}
%\begin{leftbar}
\starttable
\texttt{filename} & \textsl{Mandatory} & Name of file.\\
\texttt{sliced} & \pyFalse & Boolean indicating if the file is sliced or not.\\
\texttt{extra} & \pyNone & Any additional information to the job to be built.\\
\stoptable
The \texttt{.withfile()} is used to highlight a specific file in a job
and feed it to another job \texttt{build()}.  The file could be
sliced.
%\end{leftbar}


\subsection{\texttt{Currentjob.link\_result()}}
\starttable
\texttt{filename} & \textsl{Mandatory} & Name of file in job directory.\\
\texttt{linkname}  & \pyNone & Name of link if set.\\
\stoptable
Use to create a soft link from a file in a job directory to the
\texttt{result\_directory}.

\textbf{NOTE}: This only works for \texttt{Job} instances, and not
\texttt{CurrentJob} instances.  This is for reproducibility reasons.
Links in \texttt{result\_directory} cannot be recreated if created in
a job, since jobs can only be executed once.


\subsection{\texttt{CurrentJob.json\_save()}}
%\begin{leftbar}
\starttable
\texttt{obj} & \textsl{Mandatory} & \\
\texttt{filename} & \texttt{result.json} & \\
\texttt{sliceno} & \pyNone & \\
\texttt{sort\_keys} & \pyTrue & \\
\texttt{temp} & \pyNone & \\
\stoptable
For \texttt{CurrentJob} instances only.  Save data into the current
job's directory in JSON format.
The \texttt{temp} argument is used to control the persistence of files
written using \texttt{.json\_save()}.  This is useful mainly for debug
purposes, and explained in section~\ref{sec:debugflag}.
%\end{leftbar}


\subsection{\texttt{CurrentJob.save()}}
%\begin{leftbar}
\starttable
\texttt{obj} & \textsl{Mandatory} & \\
\texttt{filename} & \texttt{result.pickle} & \\
\texttt{sliceno} & \pyNone & \\
\texttt{temp} & \pyNone & \\
\stoptable
For \texttt{CurrentJob} instances only.  Save data into the current
job's directory in Python's pickle format.
The \texttt{temp} argument is used to control the persistence of files
written using \texttt{.save()}.  This is useful mainly for debug
purposes, and explained in section~\ref{sec:debugflag}.
%\end{leftbar}


\subsection{Sliced Files}
\label{sec:slicedfiles}
%\begin{leftbar}
A \textsl{sliced} file is actually a set of files used to store data
independently in each \analysis process using a common name.  The
functions that operate on files, such as for example \texttt{.open()}
and \texttt{.load()}, can switch to sliced files using
the \texttt{sliceno} parameter.  From a user's perspective, they
always appear to work on single files.  For example
\begin{python}
def analysis(sliceno, job):
    data = ...
    job.date(data, "mydata", sliceno=sliceno, temp=False)
\end{python}
will create a set of files \texttt{mydata.\%d}, where \texttt{\%d} is
replaced by the slice number.  In this way, data can be passed ``in
parallel'' between different jobs.
%\end{leftbar}


\subsection{File Persistence}
\label{sec:debugflag}
The \texttt{temp} argument controls persistence of files stored
using \texttt{.open()}, \texttt{.save()}, or \texttt{.json\_save()}.
By default it is being set to \pyFalse, which implies that the stored
file is \textsl{not} temporary.  But setting it to \pyTrue, like in
the following
\begin{python}
job.save(data, filename, temp=True)
\end{python}
will cause the stored file to be deleted upon job completion.  The
functionality can be combined with the \textsl{debug} mode, see below.
\begin{snugshade}
\begin{center}
\begin{tabular*}{\textwidth}{l@{\extracolsep{\fill}}ll}
  \texttt{temp}      & ``normal'' mode     & debug mode  \\\hline
  \pyFalse           & stored              & stored\\
  \pyTrue            & stored and removed  & stored\\
\end{tabular*}
\end{center}
\end{snugshade}
\noindent Debug mode is active if the Accelerator server is started
with the \texttt{---debug} flag.


\clearpage
\section{The \texttt{JobWithFile} Class}
The \texttt{JobWithFile} class is used to create a job input parameter
from a file stored in a job.
\starttabletwo
\texttt{filename()} & Return filename.\\
\texttt{load()} & Load file contents. Takes an \texttt{encoding=}-parameter.\\
\texttt{json\_load()} & Load JSON file contents.\\
\texttt{open()} & Returns a filehandle to the open file.  Takes \texttt{encoding=}-  and \texttt{errors=}-parameters.\\
\stoptabletwo
All above functions take the argument \texttt{sliceno}, which default
is set to \pyNone, indicating that it is actually a single file on
disk.  If \texttt{sliceno} is set, it is assumed that the file is
sliced, see section~\ref{sec:slicedfiles}, and the function will look
up that slice of the file only.


\clearpage
\section{The \texttt{JobList} Class}
\label{sec:classes:joblist}
Objects of the \texttt{JobList} class are returned by member functions
to the \texttt{Urd} class.  They are used to group sessions of jobs
together.
\starttabletwo
\texttt{find()} & Return a new \texttt{JobList} with only jobs with that method or name in it.\\
\texttt{get()} & Return the latest \texttt{Job} with that method or name.\\
\texttt{[<method>]} & Same as \texttt{.get} but error if no job with that method or name is in the list.\\
\texttt{as\_tuples} &  The \texttt{JobList} represented as \texttt{(method, jid)} \texttt{tuple}s.\\
\texttt{pretty} & Return a prettified string version of the \texttt{JobList}.\\
\texttt{exectime} & Execution times in total as well as per method.\\
\texttt{print\_exectimes()} & Print execution time information to \texttt{stdout}.\\
\stoptabletwo
\noindent Detailed description of the functions, where neccessary, follows.


\subsection{\texttt{JobList.find()}}
%\begin{leftbar}
\starttable
\texttt{method} & \textsl{Mandatory} & Method or name to find.\\
\stoptable
Return a new \texttt{Joblist} will all jobs in the
current \texttt{JobList} matching the \texttt{method} argument.  The
matching part is either the unique name of the method's source code,
or the name optionally given at build time using the \texttt{name=}
argument.
%\end{leftbar}


\subsection{\texttt{JobList.get()}}
%\begin{leftbar}
\starttable
\texttt{method} & \textsl{Mandatory} & Method or name to find.\\
\texttt{default} & \pyNone & Return the latest matching job.\\
\stoptable
Return the latest job that matches the \texttt{method} argument.  The
matching part is either the unique name of the method's source code,
or the name optionally given at build time using the \texttt{name=}
argument.  If no matches are found, it will return
the \texttt{default} argument.
%\end{leftbar}


\subsection{\texttt{JobList.print\_exectimes()}}
%\begin{leftbar}
\starttable
\texttt{verbose} & \pyTrue & In addition to total time, print execution time for each method in list.\\
\stoptable
Print total execution time for the \texttt{Joblist}, and,
conditionally, execution time for each job in the list,
to \texttt{stdout}.
%\end{leftbar}



\clearpage
\section{The \texttt{Dataset} Class}
The \texttt{Dataset} class is used to operate on small or large
datasets stored on disk.  It decays to a (unicode) string when
pickled.

\starttabletwo
\texttt{columns} & A \texttt{dict} from column to properties, such as type, min, and max values.\\
\texttt{previous} & The dataset's previous dataset, if it exists, \pyNone otherwise.\\
\texttt{parent} & The dataset's parent dataset, if it exists, \pyNone otherwise.\\
\texttt{filename} & The dataset's filename, if it exists.  (\texttt{csvimport} sets this.)\\
\texttt{hashlabel} & Column used for hash partitioning, or \pyNone.\\
\texttt{caption} & The dataset's caption.\\
\texttt{lines} & A \texttt{list} with number of lines per slice.\\
\texttt{shape} & A tuple containing number of columns and number of lines in dataset.\\
\texttt{link\_to\_here()} & Used to associate a subjob's dataset with the current job, see section~\ref{sec:subjobs}.\\
\texttt{merge()} & Merge this dataset with another dataset, see section~\ref{sec:classes:dataset_merge}.\\
\texttt{chain()} & A \texttt{DatasetChain} object, see section~\ref{sec:classes:datasetchain}\\
\texttt{iterate\_chain()} & Iterator over chains, see chapter~\ref{chap:iterators}.\\
\texttt{iterate()} & Iterator over dataset see chapter~\ref{chap:iterators}.\\
\texttt{iterate\_list()} & Iterator over a list of datasets, see chapter~\ref{chap:iterators}.\\
\stoptabletwo
\noindent Detailed description of the functions, where neccessary, follows.


\subsection{\texttt{Dataset.link\_to\_here()}}
%\begin{leftbar}
\starttable
\texttt{name} & \texttt{default} & The new name of the dataset.\\
\texttt{column\_filter} & \pyNone & Iterable of columns to include, or \pyNone to get all.\\
\texttt{override\_previous} & \texttt{\_no\_override} & Set this to the new previous.\\
\stoptable
Use this to expose a subjob as a dataset in your job, like in this
example:
\begin{python}
def synthesis():
    job = build('ex')
    job.dataset().link_to_here(name='new')
\end{python}
The current job will now appear to have a dataset named \texttt{new},
that is actually a link to the subjob's \texttt{default} dataset.  It
is possible to filter which columns should be visible in the link
using \texttt{column\_filter}.  For chaining purposes, it is possible
for the link to expose a parent dataset of choice, set using
the \texttt{override\_previous} parameter.
%\end{leftbar}


\subsection{\texttt{Dataset.merge()}}
\label{sec:classes:dataset_merge}
%\begin{leftbar}
\starttable
\texttt{other} & \textsl{Mandatory} & Merge with this dataset.\\
\texttt{name} & ``\texttt{default}''& Name of new dataset\\
\texttt{previous} & \pyNone& The new dataset's \texttt{previous} dataset.\\
\texttt{allow\_unrelated} & \pyFalse& Set this if the datasets do not share a common ancestor.\\
\stoptable
Merge this and other dataset. Columns from the other dataset take
priority.  If datasets do not have a common ancestor you get an error
unless \texttt{allow\_unrelated} is set. The new dataset always has
the previous specified here (even if \pyNone).  Returns the new
dataset.
%\end{leftbar}


\subsection{\texttt{Dataset.chain()}}
%\begin{leftbar}
\starttable
\texttt{length} & \texttt{-1} & Number of datasets in chain.  The default value of \texttt{-1} will include all datasets in chain.\\
\texttt{reverse} & \pyFalse & Reverse order of chain.\\
\texttt{stop\_ds} & \pyNone & If set, chain will start at the dataset after \texttt{stop\_ds}.\\
\stoptable
This function will return a \texttt{DatasetChain} object, see
section~\ref{sec:classes:datasetchain}.
%\end{leftbar}



\clearpage
\section{The \texttt{DatasetChain} Class}
\label{sec:classes:datasetchain}
These are lists of datasets returned from Dataset.chain.
They exist to provide some convenience methods on chains.
\starttabletwo
\texttt{min()} & Min value for a specified column over the whole chain.\\
\texttt{max()} & Max value for a specified column over the whole chain.\\
\texttt{lines()} & Number of rows in this chain, optionally for a specific slice. \\
\texttt{column\_counts()} & The number of datasets each column appears in.\\
\texttt{column\_count()} & Number of datasets in this chain that contain a specified column.\\
\texttt{with\_column()} & Return a new chain without any datasets that don't contain a specified column.\\
\texttt{none\_support()} & Return \pyTrue if any dataset in the chain has None-support for this column.\\
\mintinline{python}|iterate(...)| & Same arguments as \texttt{Dataset.iterate()}.  Will iterate over the whole chain.\\
\stoptabletwo
\noindent Detailed description of the functions, where neccessary, follows.


\subsection{\texttt{DatasetChain.min()}, \texttt{DatasetChain.max()}}
%\begin{leftbar}
\starttable
\texttt{column} & \textsl{Mandatory} & Min/max value of column, see text.\\
\stoptable
Minimum or maximum value for column over the whole chain.  Will be
\pyNone if no dataset in the chain contains \texttt{column}, if all datasets are
empty or if \texttt{column} has a type without min/max tracking.
%\end{leftbar}


\subsection{\texttt{DatasetChain.lines()}}
%\begin{leftbar}
\starttable
\texttt{sliceno} & \pyNone & If set, return number of lines in speficied slice.\\
\stoptable
Number of rows in this chain, optionally for a specific slice.
%\end{leftbar}


\subsection{\texttt{DatasetChain.column\_counts()}}
%\begin{leftbar}
Return a Python \texttt{Counter},\mintinline{python}|{colname:
occurances}|, holding the number of datasets each column appears in.
Takes no options.
%\end{leftbar}


\subsection{\texttt{DatasetChain.column\_count()}}
%\begin{leftbar}
\starttable
\texttt{column} & \textsl{Mandatory} & A column name.\\
\stoptable
Number of datasets in this chain that contain a specified column.
%\end{leftbar}


\subsection{\texttt{DatasetChain.with\_column()}}
%\begin{leftbar}
\starttable
\texttt{column} & \textsl{Mandatory} & A column name.\\
\stoptable
Return a new \texttt{DatasetChain} with all datasets in this chain
containing a speficied column.
%\end{leftbar}



\clearpage
\section{The \texttt{DatasetWriter} Class}
\label{sec:classes:datasetwriter}
The \texttt{DatasetWriter} class is used to create datasets.  Datasets
could be stand-alone, part of a chain, or an extension (new columns)
to an existing dataset.

The class has a number of member functions, described below, that may
be used for dataset creation.  Alternatively, the new dataset could be
set up using the \texttt{DatasetWriter} \textsl{constructor}.  The
constructor approach is currently only documented in the source code,
see \texttt{dataset.py}.

\starttabletwo
\texttt{add()} & Add a new column to the dataset under creation.\\
\texttt{hashcheck()} & Check if value belongs in current slice.\\
\texttt{set\_slice()} & Set which slice that will receive the next write.\\
\texttt{enable\_hash\_discard()} & Make the write functions silently discard data that does not hash to the current slice. \\
\texttt{get\_split\_write()} & Get a writer object, see section~\ref{sec:create_in_synthesis}.\\
\texttt{get\_split\_write\_list()} & Get a writer object, see section~\ref{sec:create_in_synthesis}.\\
\texttt{get\_split\_write\_dict()} & Get a writer object, see section~\ref{sec:create_in_synthesis}.\\
\texttt{discard()} & Discard the dataset under creation.\\
\texttt{finish()} & Call this if dataset is to be used before creating job finishes, e.g.\ if the dataset under creation is input to a subjob.\\
\stoptabletwo
\noindent Detailed description of the functions, where neccessary, follows.


\subsection{\texttt{DatasetWriter.add()}}
\label{sec:datasetwriter_add}
%\begin{leftbar}
\starttable
\texttt{colname} & \textsl{Mandatory} & Name of new column.\\
\texttt{coltype} & \textsl{Mandatory} & Type of new column.\\
\texttt{none\_support} & \pyFalse & Set to \pyTrue to allow storing {\pyNone}s.\\
\stoptable

Add a new column to a dataset in creation.  This example will create
an \texttt{age} column of type \texttt{number}, where the values could
also be \pyNone.
\begin{python}
dw.add('age', 'number', none_support=True)
\end{python}
All dataset types are described in chapter~\ref{chap:datasets}.
%\end{leftbar}


\subsection{\texttt{DatasetWriter.hashcheck()}}
%\begin{leftbar}
\starttable
\texttt{value} & \textsl{Mandatory} & Some data/\\
\stoptable
Check if a value belongs to the current slice.
Return \pyTrue if \texttt{value} belongs to the current
slice, \pyFalse otherwise.
%\end{leftbar}


\subsection{\texttt{DatasetWriter.set\_slice()}}
%\begin{leftbar}
\starttable
\texttt{sliceno} & \textsl{Mandatory} & Slice number to use for writing.\\
\stoptable
Specify which slice that will receive the next write(s).  Use this if
writing data in \prepare or \synthesis.
%\end{leftbar}


\subsection{\texttt{DatasetWriter.enable\_hash\_discard()}}
%\begin{leftbar}
Takes no options.  Set this in each slice or after
each \texttt{set\_slice()} to make the writer discard values that do
not belong to the current slice.
%\end{leftbar}



\clearpage
\section{The \texttt{Urd} Class}
\label{sec:classes:urd}
\starttabletwo
\texttt{get()} & Get an Urd item from a specified list and timetamp.\\
\texttt{latest()} & Get the latest Urd item for a specified list.\\
\texttt{first()} & Get the first Urd item for a specified list.\\
\texttt{peek()} & Get an Urd item from a specified list and timetamp without recording.\\
\texttt{peek\_latest()} & Get the latest Urd item for a specified list without recording.\\
\texttt{peek\_first()} & Get the first Urd item for a specified list without recording.\\
\texttt{since()} & Get all timestamps later than a specified timestamp for a specified list.\\
\texttt{list()} & List all Urd lists\\
\texttt{begin()} & Start a new Urd session.\\
\texttt{abort()} & Abort a running Urd session.\\
\texttt{finish()} & Finish a running Urd session and store its contents.\\
\texttt{truncate()} & Discard all Urd items later than a specified timestamp for a specified list.\\
\texttt{set\_workdir()} & Set the target workdir.\\
\texttt{build()} & Build a job.\\
\texttt{build\_chained()} & Build a job with chaining.\\
\texttt{warn()} & Add a warning message to be displayed at the end of the build.\\
\stoptabletwo
\noindent Detailed description of the functions, where neccessary, follows.


\subsection{\texttt{Urd.get()}}
\starttable
\texttt{path} & \textsl{Mandatory} & \\
\texttt{timestamp} & \textsl{Mandatory} & \\
\stoptable
Get an Urd item with specified list and timestamp.  The operation is
recorded in the current Urd session.
%\begin{leftbar}
%\end{leftbar}


\subsection{\texttt{Urd.latest()}}
\starttable
\texttt{path} & \textsl{Mandatory} & \\
\stoptable
Get the latest job in a specified Urd list.  The operation is recorded
in the current Urd session.
%\begin{leftbar}
%\end{leftbar}


\subsection{\texttt{Urd.first()}}
\starttable
\texttt{path} & \textsl{Mandatory} & \\
\stoptable
Get the first job in a specified Urd list.  The operation is recorded
in the current Urd session.
%\begin{leftbar}
%\end{leftbar}


\subsection{\texttt{Urd.peek()}}
\starttable
\texttt{path} & \textsl{Mandatory} & \\
\texttt{timestamp} & \textsl{Mandatory} & \\
\stoptable
Same as \texttt{.get()}, but without recording the dependency.
%\begin{leftbar}
%\end{leftbar}


\subsection{\texttt{Urd.peek\_latest()}}
\starttable
\texttt{path} & \textsl{Mandatory} & \\
\stoptable
Same as \texttt{.latest()}, but without recording the dependency.
%\begin{leftbar}
%\end{leftbar}


\subsection{\texttt{Urd.peek\_first()}}
\starttable
\texttt{path} & \textsl{Mandatory} & \\
\stoptable
Same as \texttt{.first()}, but without recording the dependency.
%\begin{leftbar}
%\end{leftbar}


\subsection{\texttt{Urd.since()}}
\starttable
\texttt{path} & \textsl{Mandatory} & \\
\texttt{timestamp} & \textsl{Mandatory} & \\
\stoptable
Return a list of all timestamps more recent than the
input \texttt{timestamp} for a specified Urd list.
%\begin{leftbar}
%\end{leftbar}


\subsection{\texttt{Urd.list()}}
Return a list of all available Urd lists.
%\begin{leftbar}
%\end{leftbar}


\subsection{\texttt{Urd.begin()}}
\starttable
\texttt{path} & \textsl{Mandatory} & \\
\texttt{timestamp} & \textsl{Mandatory} & \\
\texttt{caption} & \pyNone & \\
\texttt{update} & \pyFalse & \\
\stoptable
Start a new Urd session.
%\begin{leftbar}
%\end{leftbar}


\subsection{\texttt{Urd.abort()}}
Abort the current Urd session, discard its contents.
%\begin{leftbar}
%\end{leftbar}


\subsection{\texttt{Urd.finish()}}
\starttable
\texttt{path} & \textsl{Mandatory} & \\
\texttt{timestamp} & \textsl{Mandatory} & \\
\texttt{caption} & \pyNone & \\
\stoptable
Finish the current Urd session and store it in the Urd database.
%\begin{leftbar}
%\end{leftbar}


\subsection{\texttt{Urd.truncate()}}
\starttable
\texttt{path} & \textsl{Mandatory} & \\
\texttt{timestamp} & \textsl{Mandatory} & \\
\stoptable
Discard everything later than \texttt{timestamp} for the specified Urd
list.
%\begin{leftbar}
%\end{leftbar}


\subsection{\texttt{Urd.set\_workdir()}}
\starttable
\texttt{workdir} & \textsl{Mandatory} & \\
\stoptable
Set target workdir.  It can be set to any workdir present in the
Accelerator's configuration file.
%\begin{leftbar}
%\end{leftbar}


\subsection{\texttt{Urd.build()}}
%\begin{leftbar}
\starttable
\texttt{method} & \textsl{Mandatory} & Method to build.\\
\texttt{options} & \texttt{\{\}} & Input options.\\
\texttt{datasets} & \texttt{\{\}} & Input datasets.\\
\texttt{jobs} & \texttt{\{\}} & Input jobs.\\
\texttt{name} & \pyNone & Record job using this name instead of method name.\\
\texttt{caption} & \pyNone & Optional caption\\
%\texttt{why\_build} & \pyFalse & \\
\texttt{workdir} & \pyNone & Store job in this workdir.\\
\stoptable
Build a job.  If an Urd session is running, the job and its
dependencies will be recorded.
%\end{leftbar}


\subsection{\texttt{Urd.build\_chained()}}
Build a chained job.  Same options as \texttt{.build()}. See
chapter~\ref{chap:datasets} for more information.


\subsection{\texttt{Urd.warn()}}
Print a string to \texttt{stdout} when the build script ends with no
errors.
\starttable
\texttt{line} & \textsl{Mandatory} & Some string\\
\stoptable
