%%%%%%%%%%%%%%%%%%%%%%%%%%%%%%%%%%%%%%%%%%%%%%%%%%%%%%%%%%%%%%%%%%%%%%%%%%%%
%                                                                          %
% Copyright (c) 2018 eBay Inc.                                             %
% Modifications copyright (c) 2019-2021 Anders Berkeman                    %
%                                                                          %
% Licensed under the Apache License, Version 2.0 (the "License");          %
% you may not use this file except in compliance with the License.         %
% You may obtain a copy of the License at                                  %
%                                                                          %
%  http://www.apache.org/licenses/LICENSE-2.0                              %
%                                                                          %
% Unless required by applicable law or agreed to in writing, software      %
% distributed under the License is distributed on an "AS IS" BASIS,        %
% WITHOUT WARRANTIES OR CONDITIONS OF ANY KIND, either express or implied. %
% See the License for the specific language governing permissions and      %
% limitations under the License.                                           %
%                                                                          %
%%%%%%%%%%%%%%%%%%%%%%%%%%%%%%%%%%%%%%%%%%%%%%%%%%%%%%%%%%%%%%%%%%%%%%%%%%%%

\newcommand{\cmd}{\texttt{ax}\xspace}





The Accelerator is controlled using the \cmd shell command.  In order
to run any command, \cmd needs to have access to a configuration file
(see section~\ref{sec:configfile}).  The \cmd command will look for this file first
in the current directory, and then recursively in directories above
it.

It is assumed that the Accelerator server and build script
\texttt{run} commands are executed from the same directory.  This will
work out of the box.  But if set up correctly, they could be run from
different directories or even from different computers if necessary.

Asking for help is always an option.  To begin,
\begin{shell}
ax --help
\end{shell}
will print something like
\begin{snugshade}
\begin{verbatim}
usage: ax [--config CONFIG_FILE] command [args]

optional arguments:
  -h, --help            show this help message and exit
  --config CONFIG_FILE  configuration file
  --version             alias for the version command

commands:

         abort  abort running job(s)
  board-server  runs a webserver for displaying results
          curl  http request (with curl) to urd or the server
            ds  display information about datasets
          grep  search for a pattern in one or more datasets
          init  create a project directory
         intro  show introduction text
           job  information about a job
        method  information about methods
           run  run a build script
        script  information about build scripts
        server  run the main server
           urd  inspect urd contents
    urd-server  run the urd server
       version  show installed accelerator version
       workdir  information about workdirs

aliases:
  cat = grep -e ""

use "ax <command> --help" for <command> usage
try "ax intro" for an introduction
\end{verbatim}
\end{snugshade}
\noindent All commands accept the option \texttt{--help} to display
more detailed information about its useage.

Next, a quick summary will be presented before going into
the details of each command.


\section{Quick Summary of Commands}
To start with, a new project is initiated using the \texttt{init}
command,
\begin{shell}
ax init
\end{shell}
This command creates everything required to starting a project,
including directories for input data and workdir storage.  It even
sets up an empty \texttt{git} repository for the project's source
files.  All this can be overridden or specified in detail using the
command's switches.

Next, an Accelerator server is started using
\begin{shell}
ax server
\end{shell}
The server is configured using the configuration file that is
typically set up by the \texttt{init} command.

To see a list of available build scripts and their description, type
\begin{shell}
ax script
\end{shell}

A build script is run using
\begin{shell}
ax run
\end{shell}
The idea is that the \textsl{main} project build script is run just
like that, but it is possible to specify specific scripts to run as
arguments.  To run a script named \texttt{build\_name.py}, type
\texttt{ax run name}.  For systems with limited memory, and
interesting idea is the possibility to set the \textsl{concurrency}
for all or some methods using a switch.  This will save memory
resources, since only a subset of all parallel processes will be run
at once.

At any point in time, it is possible to abort the server processing
using
\begin{shell}
ax abort
\end{shell}

When coding, it is useful to have access to the list of available
methods and their interfaces, this is provided using the
\begin{shell}
ax method
\end{shell}
command.  It can either provide a list of methods per package,
together with an optional description, or given a specific method name
it will print its documentation + interface.

When a script has been run, any job can be inspected using
\begin{shell}
ax job <jobid/job-spec>
\end{shell}
that will print execution times, input parameters, output file names
etc for jobs.  See section~\ref{sec:jobspec} for alternative ways to
specify a job.

Similarly,
\begin{shell}
ax ds <dataset/dataset-spec>
\end{shell}
holds the key for a lot of information about datasets.  See
section~\ref{sec:jobspec} for alternative ways to specify a dataset.
Everything from column names, max/min-values, and types to
distribution of elements per slice is available.  If the input is a
job, it will look for the job's default dataset, but it can also show
the name of all dataset in a job, and more.

To look at the data in a dataset, use
\begin{shell}
ax grep <pattern> <dataset>
\end{shell}
or just
\begin{shell}
ax cat <dataset>
\end{shell}
to stream the data to \texttt{stdout}.  Using switches data can be
output per slice or ``in parallel'', it can print header labels and
more.

To learn about available workdirs, and which jobs that exists in workdirs,
\begin{shell}
ax workdir
\end{shell}
is used to either list all available workdirs, or list all jobs in a
workdir.

Finally,
\begin{shell}
ax version
\end{shell}
will print the current version of the Accelerator.



\section{Job and Dataset References for Command Line Tools}
For commands such as \texttt{ax job} and \texttt{ax ds}, there are
several convenient ways to specify the input jobs or datasets.

\begin{itemize}
\item[] by \textbf{jobid}, like this
  \begin{shell}
    ax job dev-12
  \end{shell}
\item[] by \textbf{method name}, like this
  \begin{shell}
    ax job csvimport
  \end{shell}
  This will look up the \textsl{latest} and \textsl{current} job
  created by the \texttt{csvimport} method, i.e.\ the latest finished
  job with current (unmodified) source code.
\item[] by \textbf{urdlist} key/timestamp, for example like this
  \begin{shell}
    ax job :import/2021-09-09:
  \end{shell}
  Here are examples of to use this:
  \begin{itemize}
    \item[] \texttt{:import:} latest job in joblist with latest timestamp in \texttt{import}.
    \item[] \texttt{:import/2021-09-09:} latest job in joblist with given timestamp.
    \item[] \texttt{:import/2021-09-09:csvimport} latest \texttt{csvimport} job in joblist with timestamp.
    \item[] \texttt{:import/2021-09-09:-2} second to last job in joblist with timestamp.
  \end{itemize}

  In addition, the \texttt{\textasciitilde} and \texttt{\texttt{\^{}}}
  can be used to ``follow'' the history of jobs in a similar fashion
  as is provided by the \texttt{git} version control tool.  Here are
  some examples of how to use it
  \begin{itemize}
  \item[] \texttt{:import/2021-09-09:\textasciitilde\textasciitilde} corresponds to the
    latest job in \texttt{import} that is two timestamps back.
  \item[] \texttt{:import/2021-09-09:\textasciitilde 2} is the same thing
  \item[] \texttt{:import/2021-09-09:\^{}} looks up the
    \texttt{.previous} to latest job in joblist with timestamp.
  \end{itemize}
\end{itemize}

Note that for specifying \textsl{datasets}, the
\texttt{\textasciitilde} and \texttt{\^{}} must be suffixed to a
dataset speficer, and not a job specifier.  In practice, this means
that a dataset name has to be given explicitly, i.e.\ write
\texttt{job-42/default\textasciitilde} and \textsl{not} \texttt{job-42\textasciitilde}.



\section{Initialisation: \texttt{ax init}}
\label{sec:initialisation}
In order to start a new project, a few things need to be set up, in
particular
\begin{itemize}
\item[] identify existing / create new \textsl{workdirs},
\item[] identify existing / create new \textsl{method packages}, and
\item[] write a \textsl{configuration file}.
\end{itemize}
This can be done manually, but a simple way to start from scratch is
to use the \texttt{init} command
\begin{shell}
ax init
\end{shell}
with the following options
\begin{shell}
ax init --help
\end{shell}
\begin{snugshade}
\begin{verbatim}
usage: ax init [-h] [--slices SLICES] [--name NAME] [--input INPUT] [--force]
               [--tcp [HOST/PORT]] [--no-git] [--examples] [DIR]

creates an accelerator project directory.
defaults to the current directory.
creates accelerator.conf, a method dir, a workdir and result dir.
both the method directory and workdir will be named <NAME>,
"dev" by default.

positional arguments:
  DIR                project directory to create. default "."

optional arguments:
  -h, --help         show this help message and exit
  --slices SLICES    override slice count detection
  --name NAME        name of method dir and workdir, default "dev"
  --input INPUT      input directory
  --force            go ahead even though directory is not empty, or workdir
                     exists with incompatible slice count
  --tcp [HOST/PORT]  listen on TCP instead of unix sockets.
                     specify HOST (can be IP) to listen on that host
                     specify PORT to use range(PORT, PORT + 3)
                     specify both as HOST:PORT
  --no-git           don't create git repository
  --examples         copy examples to project directory
\end{verbatim}
\end{snugshade}



\section{Accelerator Server: \texttt{ax server}}
\label{sec:axserver}
The Accelerator server, or daemon, needs to be running in order to
execute any commands.
\begin{shell}
ax server
\end{shell}
will start the Accelerator server, assuming that a configuration file
that makes sense is in place.
\begin{shell}
ax server --help
\end{shell}
\begin{snugshade}
\begin{verbatim}
usage: ax server [-h] [--keep-temp-files] [--debuggable]

optional arguments:
  -h, --help         show this help message and exit
  --keep-temp-files  keep temporary files in jobs
  --debuggable       make breakpoint() work in methods. note that this makes
                     a failing method kill the whole server.
\end{verbatim}%
\end{snugshade}
Communication with the Accelerator server will take place over an UNIX
socket by default.  There is no need for any additional configuration
to make that happen.  It is possible, however, to communicate over a
TCP port instead if specified in the Accelerator's configuration file.

The \texttt{debuggable}-flag will make methods execute in the same
session as the server, and use that terminal as \texttt{stdin}.  This
way methods can be debugged using the Python Debugger.  Note that a
failing method will kill the server if this flag is used.



\section{Running Build Scripts: \texttt{ax run}}
\label{sec:exec_runner}
Build scripts are executed using
\begin{python}
ax run <script>
\end{python}

\begin{python}
ax run --help
\end{python}

\begin{snugshade}
\begin{verbatim}
usage: ax run [options] [script]

positional arguments:
  script                build script to run. default "build".
                        searches under all method directories in alphabetical
                        order if it does not contain a dot.
                        prefixes build_ to last element unless specified.
                        package name suffixes are ok.
                        so for example "test_methods.tests" expands to
                        "accelerator.test_methods.build_tests".

optional arguments:
  -h, --help            show this help message and exit
  -f FLAGS, --flags FLAGS
                        comma separated list of flags
  -q, --quick           skip method updates and checking workdirs for new jobs
  -c SPEC, --concurrency SPEC
                        set max concurrency for methods, either method=N
                        or just N to set for all other methods
  -w WORKDIR, --workdir WORKDIR
                        build in this workdir
                        set_workdir() and workdir= override this.
  -W, --just_wait       just wait for running job, don't run any build script
  -p, --full-path       print full path to jobdirs
  --verbose VERBOSE     verbosity style {no, status, dots, log}
  --quiet               same as --verbose=no
  --horizon HORIZON     time horizon - dates after this are not visible in
                        urd.latest
\end{verbatim}
\end{snugshade}
When the \texttt{run} command starts, it will instruct the Accelerator
to scan all method directories to see if there are any new or changed
methods.  Thereafter, the Accelerator will proceed and scan all source
workdirs to see if any new jobs have been created (by another
Accelerator server).  Thereafter, it will execute the build script.




\section{Dataset Information: \texttt{ax ds}}
The \texttt{ds} command gives a compact, but easy to read,
overview of either
\begin{itemize}
\item[] a dataset,
\item[] a chain of datasets, or
\item[] available datasets in a job directory.
\end{itemize}
it provides information about column names and types, max and min
values, number of rows, and balance of rows between slices.

\begin{snugshade}
\begin{verbatim}
usage: ax ds [options] ds [ds [...]]

positional arguments:
  dataset               the job part of the dataset name can be specified in
                        the same ways as for "ax job". you can use ds~ or ds~N
                        to follow the chain N steps backwards, or ^ to follow
                        .parent. this requires specifying the ds-name, so
                        wd-1~ will not do this, but wd-1/default~ will.

optional arguments:
  -h, --help            show this help message and exit
  -c, --chain           list all datasets in a chain
  -C, --non-empty-chain
                        list all non-empty datasets in a chain
  -l, --list            list all datasets in a job with number of rows
  -L, --chainedlist     list all datasets in a job with number of chained rows
  -m, --suppress-minmax
                        do not print min/max column values
  -n, --suppress-columns
                        do not print columns
  -q, --suppress-errors
                        silently ignores bad input datasets/jobids
  -s, --slices          list relative number of lines per slice in sorted
                        order
  -S, --chainedslices   same as -s but for full chain
  -w, --location        show where (ds/filename) each column is stored
\end{verbatim}
\end{snugshade}

The \texttt{dataset} option is either a \textsl{dataset}, when used
with the \texttt{-s}, \texttt{-S}, and \texttt{-c} options, or a
\textsl{jobid} when used with \texttt{-l} option.  Datasets or jobids
could be either names or absolute paths.  Examples of valid datasets
are \texttt{test-2}, \texttt{test-2/default}, and
\texttt{/home/wdirs/test/test2/dsx}.  Of these, only \texttt{test-2}
is a valid jobid.
Here are all options
\begin{snugshade}
\begin{tabular}{p{4cm}p{9cm}}
  \texttt{-h}\hspace{3cm}\texttt{---help} & show help message and exit.\\[4ex]
  \texttt{-q}\hspace{3cm}\texttt{---quiet} & Silently ignore any error.\\
\end{tabular}
\end{snugshade}
When \texttt{ds} is fed with dataset(s)
\begin{snugshade}
\begin{tabular}{p{4cm}p{9cm}}
  \texttt{-c}\hspace{3cm}\texttt{---chain} & Print name and number of
  lines for all datasets in the chain.\\[4ex]
  \texttt{-C}\hspace{3cm}\texttt{---non-empty-chain} & Same as \texttt{-c},
  but elide empty datasets.\\[4ex]
  \texttt{-m}\hspace{3cm}\texttt{---suppress-minmax} & Don't print the
  [min, max] for each column.\\[4ex]
  \texttt{-n}\hspace{3cm}\texttt{---suppress-columns} & Don't print columns
  at all.\\[4ex]
  \texttt{-q}\hspace{3cm}\texttt{---suppress-errors} & Bad datasets and
  jobids are silently ignored.\\[4ex]
  \texttt{-s}\hspace{3cm}\texttt{---slices} & Print absolute and
  relative number of lines per slice for the input dataset.\\[2ex]
  \texttt{-S}\hspace{3cm}\texttt{---chain} & Same as \texttt{-s}, but
  data is for the whole chain of datasets.\\[4ex]
  \texttt{-w}\hspace{3cm}\texttt{---location} & Print location where data
  for each column is stored and how it is compressed.\\
\end{tabular}
\end{snugshade}
When \texttt{ds} is fed with jobid(s)
\begin{snugshade}
\begin{tabular}{p{4cm}p{9cm}}
  \texttt{-l}\hspace{3cm}\texttt{---list} & Print the name and number
  of lines of all datasets in the input jobid.\\[2ex]
  \texttt{-L}\hspace{3cm}\texttt{---chainedlist} & Same as \texttt{-l},
  but print number of lines for the whole chain from each dataset.\\
\end{tabular}
\end{snugshade}

\subsubsection*{Example invocation 1}
\begin{shell}
ax ds test-20 -S
\end{shell}
or
\begin{shell}
ax ds test-20/badlines
\end{shell}

The argument can be one or more jobids or dataset ids.  If the
argument is a jobid, it is assumed that the dataset name is
\texttt{default}.  If there are more than one dataset in the job, a
list of dataset names will be returned.

\subsubsection*{Example invocation 2}
Combining \texttt{ds} will shell features can be an elegant way to
extract information from a workdir.  For example
\begin{shell}
ax ds -l -q test-{0..99}
\end{shell}
will scan for datasets in the 100 first jobs of \texttt{test}.
Adding the \texttt{-q} option will make \texttt{ds} suppress the
warning messages for those jobs that do not contain any datasets.

\subsubsection*{Example invocation 3}
Find all datasets in \texttt{test-20}
\begin{shell}
ax ds -l test-20
\end{shell}

\subsubsection*{Example Output}
\begin{snugshade}
\begin{verbatim}
import-2340/default
    Previous: import-2245/default
    Hashlabel: serial_number
    Columns:
          capacity_bytes        int64   [        -1, 14000519643136]
          date                  date    [2019-04-01, 2019-06-30]
          failure               bool    [     False,       True]
          model                 ascii
        * serial_number         ascii
    5 columns
    9,831,138 lines
    Chain length 17, from import-269 to import-2340
    Balance, lines per slice, full chain:
          1:   4.44% (6,486,330)   20:   4.35% (6,365,896)    3:   4.32% (6,323,701)
          0:   4.43% (6,472,949)   17:   4.35% (6,363,206)   19:   4.31% (6,309,295)
         11:   4.39% (6,423,397)   21:   4.35% (6,360,759)   12:   4.31% (6,306,181)
          4:   4.39% (6,421,043)   15:   4.35% (6,358,757)    8:   4.31% (6,304,311)
          6:   4.39% (6,418,617)    7:   4.34% (6,352,750)   13:   4.31% (6,303,637)
         14:   4.39% (6,417,911)   18:   4.34% (6,348,128)    5:   4.28% (6,252,735)
          2:   4.36% (6,376,139)   10:   4.34% (6,347,694)   16:   4.26% (6,230,543)
         22:   4.35% (6,366,040)    9:   4.33% (6,325,241)
    Max to average ratio: 1.020
    146,235,260 total lines in chain
\end{verbatim}
\end{snugshade}
The \texttt{max to average ratio} shows the ratio between the slice
with most rows and the average number of rows.  This can be
interpreted as the execution time overhead for a dataset where all
slices are not of the same size.  When the ratio is close to one, the
dataset is perfectly balanced.


\section{Look at Specific Data in a Dataset: \texttt{ax grep}}
The \texttt{grep} command is a parallel grep for datasets that is
used to look at data stored in datasets or dataset chains.
\begin{snugshade}
\begin{verbatim}
usage: ax grep [options] [-e] pattern [...] [-d] ds [...] [[-n] column [...]]

positional arguments:
  pattern               (-e, --regexp)
  dataset               (-d, --dataset) can be specified as for "ax ds"
  columns               (-n, --column)

optional arguments:
  -h, --help            show this help message and exit
  -c, --chain           follow dataset chains
  --chain-length LENGTH, --cl LENGTH
                        follow chains at most this many datasets
  --stop-ds DATASET     follow chains at most to this dataset
                        these options work like in ds.chain() and you can
                        specify them several times (for several datasets)
  --colour [WHEN], --color [WHEN]
                        colour matched text. can be auto, never or always
  -i, --ignore-case     case insensitive pattern
  -v, --invert-match    select non-matching lines
  -o, --only-matching   only print matching part (or columns with -l)
  -l, --list-matching   only print matching datasets (or slices with -S)
                        when used with -o, only print matching columns
  -H, --headers         print column names before output (and on each change)
  -O, --ordered         output in order (one slice at a time)
  -M, --allow-missing-columns
                        datasets are allowed to not have (some) columns
  -g COLUMN, --grep COLUMN
                        grep this column only, can be specified multiple times
  -s SLICE, --slice SLICE
                        grep this slice only, can be specified multiple times
  -D, --show-dataset    show dataset on matching lines
  -S, --show-sliceno    show sliceno on matching lines
  -L, --show-lineno     show lineno (per slice) on matching lines
  -f FORMAT, --format FORMAT
                        output format, csv (default) / raw / json
  -t SEPARATOR, --separator SEPARATOR
                        field separator, default tab / tab-like spaces
  -T LENGTH, --tab-length LENGTH
                        field alignment, always uses spaces as separator
                        specify as many as you like of TABLEN/FIELDLEN/MINLEN
                        or as NAME=VALUE (e.g. "min=3/field=24")
                        TABLEN works like normal tabs
                        FIELDLEN sets a longer minimum between fields
                        MINLEN sets a minimum len for all separators
                        use "-T/" to just activate it (sets 8/16/2)
  -B NUM, --before-context NUM
                        print NUM lines of leading context
  -A NUM, --after-context NUM
                        print NUM lines of trailing context
  -C NUM, --context NUM
                        print NUM lines of context
                        context is only taken from the same slice of the same
                        dataset, and may intermix with output from other
                        slices. Use -O to avoid that, or -S -L to see it.
\end{verbatim}
\end{snugshade}
\noindent The \texttt{pattern} is a regular expression and \texttt{ds} are
datasets.  Strings and columns with special characters have to be quoted.
For example
\begin{shell}
ax grep Alice test-0 test-1/special "first name"
\end{shell}
Will look for the string \texttt{Alice} in the \texttt{first name} column of
the two datasets \texttt{test-0} and \texttt{test-1/special}.


\section{Information about Methods: \texttt{ax method}}
Use \texttt{ax method} to print a list of available methods together
with an optional description.
\begin{shell}
ax method
\end{shell}
The method description is shown by specifying a method name
\begin{shell}
ax method csvimport
\end{shell}
Descriptions are added using the \py{description = "some string"}
statement in a method.


\section{List all build scripts: \texttt{ax script}}
List all available build scripts using
\begin{shell}
ax script
\end{shell}
Descriptions are added using the \py{description = "some string"}
statement in a build script.


\section{Information about Jobs: \texttt{ax job}}
Get information about a job by typing
\begin{shell}
ax job <jobname>
\end{shell}


\section{Starting the Board Web Server: \texttt{ax board-server}}

Normally you don't need to start this separately as \texttt{ax server}
also runs the board server by default.


\section{Aliases}
It is possible to create aliases for common tasks.  Aliases are
inserted in the file \texttt{config} in the
\texttt{\$XDG\_CONFIG\_HOME/accelerator} directory.  By default, this
corresponds to the file \texttt{\$HOME/.config/accelerator/config}.
In this file, aliases are entered in the \texttt{[alias]} section like this
\begin{verbatim}
[alias]
    path = job -P
\end{verbatim}
This will create the alias \texttt{ax path} that is equivalent to
\texttt{ax job -P}.



\section{Show Dataset Contents: \texttt{ax cat}}
The data in a dataset may be printed to \texttt{stdout} using the
\texttt{cat} alias, for example like this
\begin{shell}
ax cat test-0 | less
\end{shell}
This is an alias for \texttt{ax grep ""}, which greps a dataset for the
empty regexp, which matches every line.
A common usage is to do

\begin{shell}
ax cat -HO test-0 column1 column2
\end{shell}
to list a set of columns (or all, if omitted) row by row in order with
a header showing label names.

If you find yourself missing the smart tab mode and colour when piping to
less you may wish to create an alias like
\begin{verbatim}
[alias]
    catc = cat -T/ --colour=always
\end{verbatim}
or similar.


\section{The Urd Job Database Server: \texttt{ax urd-server}}
By default, a local Urd server is running when the Accelerator server
is running.  Read more about this in section~\ref{chap:urd}.  A
stand-alone Urd server is started by
\begin{shell}
ax urd-server
\end{shell}
These are the options
\begin{shell}
ax urd-server --help
\end{shell}
\begin{snugshade}
\begin{verbatim}
usage: ax urd-server [-h] [--path PATH] [--allow-passwordless] [--quiet]

options:
  -h, --help            show this help message and exit
  --path PATH           database directory (can be relative to project
                        directory) (default: urd.db)
  --allow-passwordless  accept any pass for users not in passwd.
  --quiet               less chatty.
\end{verbatim}
\end{snugshade}
This will listen on the port/socket specified in the Accelerator's
configuration file. You will probably want to configure this as
\texttt{remote} so the server does not try to run another urd on the
same port/socket.


\subsection{Authorization to Urd}
Authorisation to Urd could be set in the \texttt{URD\_AUTH}
environment variable.  A common way to invoke the run command with Urd
authorisation is like this
\begin{shell}
URD_AUTH=user:passwd ax run [script]
\end{shell}
Note that the purpose of the authentication is
actually \textsl{identification}.  It is used to get write access to
certain Urd lists.  Nothing more.


\section{Aborting a Running Job: \texttt{ax abort}}
The command \texttt{ax abort} will abort any running job.  To abort an
entire build script, first type \texttt{Ctrl-C} followed by \texttt{ax
  abort}.
\begin{snugshade}
\begin{verbatim}
usage: ax abort [-h] [-q]

optional arguments:
  -h, --help   show this help message and exit
  -q, --quiet  no output
\end{verbatim}
\end{snugshade}


\section{Current Version: \texttt{ax version}}
Use \texttt{ax version} to find the currently running version.  The
Accelerator will complain if the \texttt{ax} command version does not
match the currently running server.


\section{Workdir Information: \texttt{ax workdir}}
By itself, \texttt{ax workdir} will print a list of all current
workdirs.  By specifying a workdir, all jobs in that workdir, together
with execution time and information if it is \textsl{old} or
\textsl{current} will be listed.  All jobs in all workdirs will be
listed using the \texttt{-a} option.
\begin{snugshade}
\begin{verbatim}
usage: ax workdir [-p] [-a | [workdir [workdir [...]]]

positional arguments:
  workdirs

optional arguments:
  -h, --help       show this help message and exit
  -a, --all        list all workdirs
  -p, --full-path  show full path
\end{verbatim}
\end{snugshade}


\section{The Board Web Server: \texttt{ax board-server}}
The Board server is a web server that exposes information about jobs
and results through the use of a standard web browser.  This is
started by default by the \texttt{ax server} command, but it can be
started separately using \texttt{ax board-server}.
\begin{snugshade}
\begin{verbatim}
usage: ax board-server [listen_on]
runs a web server on listen_on (default localhost:8520, can be socket path)
for displaying results (result_directory)
\end{verbatim}
\end{snugshade}
Note that unlike \texttt{urd-server} this does not use the port/socket
specified in the Accelerator's configuration file.


\section{Displaying Urd Database Content: \texttt{ax urd}}
\begin{snugshade}
\begin{verbatim}
usage: ax urd path [path [...]]

path is an optionally shortened path to an urd list.
one element is a list name, two elements are list and timestamp.
(and three elements are the whole thing, user/list/timestamp.)

use "path/since/ts" or just "path/" to list timestamps
use "/" (or nothing) to list all lists

you can also use :urdlist:[entry] job specifiers. urdlist follows the same
path rules as above, entry is an optional argument to joblist.get() printing
just the resultant jobid.

in job specifiers you can use ~ to move to forwards (later, down) along a
list of timestamps and ^ to move backwards (earlier, up).

examples:
  "ax urd example" is "ax urd eaenbrd/example/latest"
  "ax urd :example:" is also "ax urd eaenbrd/example/latest"
  "ax urd example/2021-04-14" is "ax urd eaenbrd/example/2021-04-14"
  "ax urd :foo/bar/first:" is "ax urd foo/bar/first"
  "ax urd :bar/first~:" is the second timestamp in eaenbrd/bar
  "ax urd example/" is "ax urd eaenbrd/example/since/0"
\end{verbatim}
\end{snugshade}
