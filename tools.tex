%%%%%%%%%%%%%%%%%%%%%%%%%%%%%%%%%%%%%%%%%%%%%%%%%%%%%%%%%%%%%%%%%%%%%%%%%%%%
%                                                                          %
% Copyright (c) 2018 eBay Inc.                                             %
%                                                                          %
% Licensed under the Apache License, Version 2.0 (the "License");          %
% you may not use this file except in compliance with the License.         %
% You may obtain a copy of the License at                                  %
%                                                                          %
%  http://www.apache.org/licenses/LICENSE-2.0                              %
%                                                                          %
% Unless required by applicable law or agreed to in writing, software      %
% distributed under the License is distributed on an "AS IS" BASIS,        %
% WITHOUT WARRANTIES OR CONDITIONS OF ANY KIND, either express or implied. %
% See the License for the specific language governing permissions and      %
% limitations under the License.                                           %
%                                                                          %
%%%%%%%%%%%%%%%%%%%%%%%%%%%%%%%%%%%%%%%%%%%%%%%%%%%%%%%%%%%%%%%%%%%%%%%%%%%%

\newcommand{\cmd}{\texttt{ax}\xspace}





The Accelerator is controlled using the \cmd shell command.  In order
to run any command, \cmd needs to have access to a configuration file
(see section~\ref{}).  The \cmd command will look for this file first
in the current directory, and then recursively in directories above
it.

It is assumed that the Accelerator \texttt{daemon} and build script
\texttt{run} commands are executed from the same directory.  This will
work out of the box.  But if set up correctly, they could be run from
different directories or even from different computers if necessary.

Asking for help is always an option.  To begin,
\begin{shell}
  ax --help
\end{shell}
will print something like
\begin{snugshade}
\begin{verbatim}
  usage: ax [--config CONFIG_FILE] command

  positional arguments:
  command

  optional arguments:
  --config CONFIG_FILE  Configuration file

  commands:

  daemon  Run the main daemon
  dsgrep  Search for a pattern in one or more datasets
  dsinfo  Display information about datasets
  init  Create a project directory
  run  Run an automata script
  urd  Run the urd daemon

  Use ax <command> --help for <command> usage.
\end{verbatim}
\end{snugshade}
each command will be introduced next.





\section{Initialisation}
In order to start a new project, a few things need to be set up, in
particular
\begin{itemize}
\item[] identify existing / create new \textsl{workdirs},
\item[] identify existing / create new \textsl{method packages}, and
\item[] write a \textsl{configuration file}.
\end{itemize}
This can be done manually, but a simple way to start from scratch is
to use the \texttt{init} command
\begin{shell}
  ax init
\end{shell}
with the following options
\begin{shell}
  ax init --help
\end{shell}
\begin{snugshade}
\begin{verbatim}
usage: bd init [-h] [--slices SLICES] [--name NAME] [--prefix PREFIX]
               [--input INPUT] [--force]
               [DIR]

Creates an accelerator project directory. Defaults to the current directory.
Creates accelerator.conf and a method directory. Also creates workdirs and
result dir (in ~/accelerator by default). Both the method directory and
workdir will be named <NAME>, "dev" by default.

positional arguments:
  DIR              Project directory to create, default "."

optional arguments:
  -h, --help       show this help message and exit
  --slices SLICES  Override slice count detection
  --name NAME      Name of method dir and workdir, default "dev"
  --prefix PREFIX  Put workdirs and daemon.log here, default
                   "${HOME}/accelerator"
  --input INPUT    Input directory
  --force          Go ahead even though directory is not empty, or workdir
                   exists with incompatible slice count
\end{verbatim}
\end{snugshade}
The command help listed above says most of it.  See
section~\ref{section_on_init} on how to set up a new project from
scratch.




\section{Accelerator Server/Daemon}

The Accelerator server, or daemon needs to be running in order to
execute any commands.

\subsection{Invocation}
\begin{shell}
% ax daemon
\end{shell}
will start the Accelerator daemon, assuming that a configuration file
that makes sense is in place.
\begin{shell}
% ax daemon --help
\end{shell}
\begin{snugshade}
\begin{verbatim}
usage: daemon [-h] [--debug] [--port PORT | --socket SOCKET]

optional arguments:
  -h, --help       show this help message and exit
  --debug
  --port PORT      Listen on tcp port (default: None)
  --socket SOCKET  Listen on unix socket (default: socket.dir/default)
\end{verbatim}%
\end{snugshade}
Communication with the Accelerator daemon will take place over an UNIX
socket by default.  There is no need for any additional configuration
to make that happen.  It is possible, however, to communicate over a
TCP port instead if specified.





\section{Running Build Scripts}
\label{sec:exec_runner}

\subsection{Invocation}
Build scripts are executed using
\begin{python}
ax run <script>
\end{python}

\begin{python}
ax run --help
\end{python}

\begin{snugshade}
\begin{verbatim}
  usage: run [options] [script]

positional arguments:
  script                automata script to run. default "automata".
                        searches under all method directories in alphabetical
                        order if it does not contain a dot.
                        prefixes automata_ to last element unless specified.
                        package name suffixes are ok.
                        so for example "test_methods.tests" expands to
                        "accelerator.test_methods.automata_tests".

optional arguments:
  -h, --help            show this help message and exit
  -p PORT, --port PORT  framework listening port
  -H HOSTNAME, --hostname HOSTNAME
                        framework hostname
  -S SOCKET, --socket SOCKET
                        framework unix socket (default ./socket.dir/default)
  -f FLAGS, --flags FLAGS
                        comma separated list of flags
  -A, --abort           abort (fail) currently running job(s)
  -q, --quick           skip method updates and checking workdirs for new jobs
  -w, --just_wait       just wait for running job, don't run any automata
  -F, --fullpath        print full path to jobdirs
  --verbose VERBOSE     verbosity style {no, status, dots, log}
  --quiet               same as --verbose=no
  --horizon HORIZON     time horizon - dates after this are not visible in
                        urd.latest
\end{verbatim}
\end{snugshade}
When the \texttt{run} command starts, it will instruct the Accelerator
to scan all method directories to see if there are any new or changed
methods.  Thereafter, the Accelerator will proceed and scan all source
workdirs to see if any new jobs have been created (by another
Accelerator daemon).  Thereafter, it will execute the build script.

Is for the \texttt{daemon} command, there is no need to specify any
socket or port as long as the commands are issued from the same
directory.




\section{Dataset Information}
The \texttt{dsinfo} command gives a compact, but easy to read,
overview of either
\begin{itemize}
\item[] a dataset,
\item[] a chain of datasets, or
\item[] available datasets in a job directory.
\end{itemize}
it provides information about column names and types, max and min
values, number of rows, and balance of rows between slices.

\subsection{Invocation}
\begin{shell}
ax dsinfo [options] [dataset [dataset [...]]]
\end{shell}

The \texttt{dataset} option is either a \textsl{dataset}, when used
with the \texttt{-s}, \texttt{-S}, and \texttt{-c} options, or a
\textsl{jobid} when used with \texttt{-l} option.  Datasets or jobids
could be either names or absolute paths.  Examples of valid datasets
are \texttt{test-2}, \texttt{test-2/default}, and
\texttt{/home/wdirs/test/test2/dsx}.  Of these, only \texttt{test-2}
is a valid jobid.
Here are all options
\begin{snugshade}
\begin{tabular}{p{4cm}p{9cm}}
  \texttt{-h}\hspace{3cm}\texttt{---help} & show help message and exit.\\[4ex]
  \texttt{-q}\hspace{3cm}\texttt{---quiet} & Silently ignore any error.\\
\end{tabular}
\end{snugshade}
When \texttt{dsinfo} is fed with dataset(s)
\begin{snugshade}
\begin{tabular}{p{4cm}p{9cm}}
  \texttt{-c}\hspace{3cm}\texttt{---chain} & Print name and number of
  lines for all datasets in the chain.\\[4ex]
  \texttt{-s}\hspace{3cm}\texttt{---slices} & Print absolute and
  relative number of lines per slice for the input dataset.\\[2ex]
  \texttt{-S}\hspace{3cm}\texttt{---chain} & Same as \texttt{-s}, but
  data is for the whole chain of datasets.\\
\end{tabular}
\end{snugshade}
When \texttt{dsinfo} is fed with jobid(s)
\begin{snugshade}
\begin{tabular}{p{4cm}p{9cm}}
  \texttt{-l}\hspace{3cm}\texttt{---list} & Print the name and number
  of lines of all datasets in the input jobid.\\
\end{tabular}
\end{snugshade}

\subsubsection*{Example invocation 1}
\begin{shell}
ax dsinfo test-20 -S
\end{shell}
or
\begin{shell}
ax dsinfo test-20/badlines
\end{shell}

The argument can be one or more jobids or dataset ids.  If the
argument is a jobid, it is assumed that the dataset name is
\texttt{default}.  If there are more than one dataset in the job, a
list of dataset names will be returned.

\subsubsection*{Example invocation 2}
Combining \texttt{dsinfo} will shell features can be an elegant way to
extract information from a workdir.  For example
\begin{shell}
ax dsinfo -l -q test-{0..99}
\end{shell}
will scan for datasets in the 100 first jobs of \texttt{test}.
Adding the \texttt{-q} option will make \texttt{dsinfo} suppress the
warning messages for those jobs that do not contain any datasets.

\subsubsection*{Example invocation 3}
Find all datasets in \texttt{test-20}
\begin{shell}
ax dsinfo -l test-20
\end{shell}


\subsubsection*{Example Output}
\begin{snugshade}
\begin{verbatim}
import-2340/default
    Previous: import-2245/default
    Hashlabel: serial_number
    Columns:
          capacity_bytes        int64   [        -1, 14000519643136]
          date                  date    [2019-04-01, 2019-06-30]
          failure               bool    [     False,       True]
          model                 ascii
        * serial_number         ascii
    5 columns
    9,831,138 lines
    Chain length 17, from import-269 to import-2340
    Balance, lines per slice, full chain:
          1:   4.44% (6,486,330)   20:   4.35% (6,365,896)    3:   4.32% (6,323,701)
          0:   4.43% (6,472,949)   17:   4.35% (6,363,206)   19:   4.31% (6,309,295)
         11:   4.39% (6,423,397)   21:   4.35% (6,360,759)   12:   4.31% (6,306,181)
          4:   4.39% (6,421,043)   15:   4.35% (6,358,757)    8:   4.31% (6,304,311)
          6:   4.39% (6,418,617)    7:   4.34% (6,352,750)   13:   4.31% (6,303,637)
         14:   4.39% (6,417,911)   18:   4.34% (6,348,128)    5:   4.28% (6,252,735)
          2:   4.36% (6,376,139)   10:   4.34% (6,347,694)   16:   4.26% (6,230,543)
         22:   4.35% (6,366,040)    9:   4.33% (6,325,241)
    Max to average ratio: 1.020
    146,235,260 total lines in chain
\end{verbatim}
\end{snugshade}
The \texttt{max to average ratio} shows the ratio between the slice
with most rows and the average number of rows.  This can be
interpreted as the execution time overhead for a dataset where all
slices are not of the same size.







\section{Find Data in Dataset}
The \texttt{dsgrep} command is a parallel grep for datasets that is
used to look at data stored in datasets or dataset chains.
\subsection{Invocation}
\begin{shell}
ax dsgrep [options] pattern ds [ds [...]] [column [column [...]]
\end{shell}

The \texttt{pattern} is a regular expression and \texttt{ds} are
datasets.  For example
\begin{shell}
ax dsgrep Alice test-0 test-1/special name
\end{shell}
Will look for the string \texttt{Alice} in the \texttt{name} column of
the two datasets \texttt{text-0} and \texttt{test-1/special}.
Optional arguments are
\begin{snugshade}
  \begin{tabular}{p{4cm}p{9cm}}
      \texttt{-h}\hspace{3cm}\texttt{---help} & show help message and exit\\[4ex]
      \texttt{-c}\hspace{3cm}\texttt{---chain} & follow dataset chains\\[4ex]
      \texttt{-i}\hspace{3cm}\texttt{---ignore-case} & Case insensitive pattern\\[4ex]
      \texttt{-s N}\hspace{3cm}\texttt{---slice N} & Grep in slice \texttt{N} only\\
  \end{tabular}
\end{snugshade}
Strings and columns with special characters have to be quoted.



\subsection{Abuse dsgrep to show datasets}
The data in a dataset may be printed to \texttt{stdout} by
\texttt{grep}ing using a regexp that always matches, like this
\begin{shell}
ax dsgrep . test-0 | less
\end{shell}
(The regexp \texttt{.} will match any string that is at least one character long.)




\section{The Urd Job Database Server}
Urd is run as a stand-alone server.
\subsection{Invocation}
Start Urd by
\begin{shell}
ax urd
\end{shell}
These are the options
\begin{shell}
ax urd --help
\end{shell}
\begin{snugshade}
\begin{verbatim}
usage: urd [-h] [--port PORT] [--path PATH]

optional arguments:
-h, --help   show this help message and exit
--port PORT  server port (default: 8080)
--path PATH  database directory (can be relative to project directory)
(default: ./urd.db)
\end{verbatim}
\end{snugshade}
Read about running and configuring Urd on page~\ref{urd_howto_run}



\subsection{Authorization to Urd}
Authorisation to Urd could be set in the \texttt{URD\_AUTH}
environment variable.  A common way to invoke the run command with Urd
authorisation is like this
\begin{shell}
% URD_AUTH=user:passwd ax run [script]
\end{shell}
Note that the purpose of the authentication is
actually \textsl{identification}.  It is used to get write access to
certain Urd lists.  Nothing more.
