%%%%%%%%%%%%%%%%%%%%%%%%%%%%%%%%%%%%%%%%%%%%%%%%%%%%%%%%%%%%%%%%%%%%%%%%%%%%
%                                                                          %
% Copyright (c) 2018 eBay Inc.                                             %
%                                                                          %
% Licensed under the Apache License, Version 2.0 (the "License");          %
% you may not use this file except in compliance with the License.         %
% You may obtain a copy of the License at                                  %
%                                                                          %
%  http://www.apache.org/licenses/LICENSE-2.0                              %
%                                                                          %
% Unless required by applicable law or agreed to in writing, software      %
% distributed under the License is distributed on an "AS IS" BASIS,        %
% WITHOUT WARRANTIES OR CONDITIONS OF ANY KIND, either express or implied. %
% See the License for the specific language governing permissions and      %
% limitations under the License.                                           %
%                                                                          %
%%%%%%%%%%%%%%%%%%%%%%%%%%%%%%%%%%%%%%%%%%%%%%%%%%%%%%%%%%%%%%%%%%%%%%%%%%%%

\newcommand{\cmd}{\texttt{ax}\xspace}





The Accelerator is controlled using the \cmd shell command.  Asking
for help is always an option.  To begin,
\begin{shell}
  ax --help
\end{shell}
will print something like
\begin{snugshade}
\begin{verbatim}
  usage: ax [--config CONFIG_FILE] command

  positional arguments:
  command

  optional arguments:
  --config CONFIG_FILE  Configuration file

  commands:

  daemon  Run the main daemon
  dsgrep  Search for a pattern in one or more datasets
  dsinfo  Display information about datasets
  init  Create a project directory
  run  Run an automata script
  urd  Run the urd daemon

  Use ax <command> --help for <command> usage.
\end{verbatim}
\end{snugshade}
each command will be introduced next.





\section{Initialisation}
In order to start a new project, a few things needs to be set up.  A
simple way to start from scratch is to use the \texttt{init} command
\begin{shell}
  ax init
\end{shell}
with the following options
\begin{shell}
  ax init --help
\end{shell}
\begin{snugshade}
\begin{verbatim}
  usage: init [-h] [--slices SLICES] [--name NAME] [--directory DIR]
  [--prefix PREFIX] [--source SOURCE] [--force]

  Creates an accelerator project directory. Defaults to the current directory.
  Creates accelerator.conf and a method directory. Also creates workdirs and
  result dir (in ~/accelerator by default). Both the method directory and
  workdir will be named <NAME>, "dev" by default.

  optional arguments:
  -h, --help       show this help message and exit
  --slices SLICES  Override slice count detection
  --name NAME      Name of method dir and workdir, default ``dev''
  --directory DIR  project directory to create. default ``.''
  --prefix PREFIX  Put workdirs and daemon.log here, default
  ``${HOME}/accelerator''
  --source SOURCE  source directory
  --force          Go ahead even though directory is not empty, or workdir
  exists with incompatible slice count
\end{verbatim}
\end{snugshade}




\section{Accelerator Server/Daemon}

The Accelerator server, or daemon needs to be running in order to
execute any commands.

\subsection{Invocation}
\begin{shell}
% ax daemon
\end{shell}
will start the Accelerator daemon, assuming that a configuration file
that makes sense is in place.
\begin{shell}
% ax daemon --help
\end{shell}
\begin{snugshade}
\begin{verbatim}
usage: daemon [-h] [--debug] [--port PORT | --socket SOCKET]

optional arguments:
  -h, --help       show this help message and exit
  --debug
  --port PORT      Listen on tcp port (default: None)
  --socket SOCKET  Listen on unix socket (default: socket.dir/default)
\end{verbatim}%
\end{snugshade}

Optional arguments
\begin{snugshade}
\begin{tabular}{p{4cm}p{9cm}}
  \texttt{-h}\hspace{3cm}\texttt{---help} & show help message and
  exit.\\[4ex]

  \texttt{---debug} & Start in debug mode.  See section~\ref{sec:debugflag}\\[2ex]
  \texttt{---config CONFIG\_FILE} & configuration file, default
  \texttt{../conf/framework.conf}\\[4ex]

  \texttt{---port PORT} & listen on TCP port (default \pyNone)\\[4ex]

  \texttt{---socket SOCKET} & listen on unix socket, default
  \texttt{socket.dir/default}\\[4ex]
\end{tabular}
\end{snugshade}
Communication with the Accelerator daemon will take place over an UNIX
socket.  There is no need for any additional configuration to make
that happen.  It is possible, however, to communicate over a TCP port
instead if specified.





\section{Running Build Scripts}
\label{sec:exec_runner}

\subsection{Invocation}
Build scripts are executed using
\begin{python}
ax run [script]
\end{python}

\begin{python}
ax run --help
\end{python}

\begin{snugshade}
\begin{verbatim}
  usage: run [options] [script]

positional arguments:
  script                automata script to run. default "automata".
                        searches under all method directories in alphabetical
                        order if it does not contain a dot.
                        prefixes automata_ to last element unless specified.
                        package name suffixes are ok.
                        so for example "test_methods.tests" expands to
                        "accelerator.test_methods.automata_tests".

optional arguments:
  -h, --help            show this help message and exit
  -p PORT, --port PORT  framework listening port
  -H HOSTNAME, --hostname HOSTNAME
                        framework hostname
  -S SOCKET, --socket SOCKET
                        framework unix socket (default ./socket.dir/default)
  -f FLAGS, --flags FLAGS
                        comma separated list of flags
  -A, --abort           abort (fail) currently running job(s)
  -q, --quick           skip method updates and checking workdirs for new jobs
  -w, --just_wait       just wait for running job, don't run any automata
  -F, --fullpath        print full path to jobdirs
  --verbose VERBOSE     verbosity style {no, status, dots, log}
  --quiet               same as --verbose=no
  --horizon HORIZON     time horizon - dates after this are not visible in
                        urd.latest
\end{verbatim}
\end{snugshade}
When the \texttt{run} command starts, it will instruct the Accelerator
to scan all method directories to see if there are any new or changed
methods.  Thereafter, the Accelerator will proceed and scan all source
workdirs to see if any new jobs have been created (by another
Accelerator daemon).  Thereafter, it will execute the build script.

If there are more than one build script with the same name, the one
first in the list of method packages specified in the Accelerator's
configuration file will be used.  This can be overridden by explicitly
specifying the method package and build scrip to run.





\section{Dataset Information}
The \texttt{dsinfo} command gives a compact, but easy to read,
overview of either
\begin{itemize}
\item[] a dataset,
\item[] a chain of datasets, or
\item[] available datasets in a job directory.
\end{itemize}

\subsection{Invocation}
\begin{shell}
ax dsinfo --help
\end{shell}
\begin{snugshade}
\begin{verbatim}
usage: dsinfo [options] dataset(s)

positional arguments:
  dataset

optional arguments:
  -h, --help            show this help message and exit
  -c, --chain           list all datasets in a chain
  -l, --list            list all datasets in a job
  -L, --chainedlist     list all datasets in a job, @@@@@@@@@@@@@@
  -m, --suppress_minmax
                        do not print min/max column values
  -n, --suppress_columns
                        do not print columns
  -q, --suppress_errors
                        silently ignores bad input datasets/jobids
  -s, --slices          list relative number of lines per slice in sorted
                        order
  -S, --chainedslices   same as -s but for full chain
\end{verbatim}
\end{snugshade}

The \texttt{dataset} option is either a \textsl{dataset}, when used
with the \texttt{-s}, \texttt{-S}, and \texttt{-c} options, or a
\textsl{jobid} when used with \texttt{-l} option.  Datasets or jobids
could be either names or absolute paths.  Examples of valid datasets
are \texttt{test-2}, \texttt{test-2/default}, and
\texttt{/home/wdirs/test/test2/dsx}.  Of these, only \texttt{test-2}
is a valid jobid.  Here are all options
\begin{snugshade}
\begin{tabular}{p{4cm}p{9cm}}
  \texttt{-h}\hspace{3cm}\texttt{---help} & show help message and exit.\\[4ex]
  \texttt{-q}\hspace{3cm}\texttt{---quiiet} & Silently ignore any error.\\
\end{tabular}
\end{snugshade}
When \texttt{dsinfo} is fed with dataset(s)
\begin{snugshade}
\begin{tabular}{p{4cm}p{9cm}}
  \texttt{-c}\hspace{3cm}\texttt{---chain} & Print name and number of
  lines for all datasets in the chain.\\[4ex]
  \texttt{-s}\hspace{3cm}\texttt{---slices} & Print absolute and
  relative number of lines per slice for the input dataset.\\[4ex]
  \texttt{-S}\hspace{3cm}\texttt{---chain} & Same as \texttt{-s}, but
  data is for the whole chain of datasets.\\
\end{tabular}
\end{snugshade}
When \texttt{dsinfo} is fed with jobid(s)
\begin{snugshade}
\begin{tabular}{p{4cm}p{9cm}}
  \texttt{-l}\hspace{3cm}\texttt{---list} & Print the name and number
  of lines of all datasets in the input jobid.\\
\end{tabular}
\end{snugshade}

\begin{verbatim}
WIDE37-1/default
    Parent: None
    Hashlabel: None
    Columns:
        'a string'       ascii
        'gauss float'    float64
        'gauss number'   number
        'large number'   number
        'small integer'  int32
        'small number'   number
    6 columns
    1,000,000 lines
    Chain length 2, from WIDE37-0 to WIDE37-1
    0: WIDE37-0/default (1,000,000)
    1: WIDE37-1/default (1,000,000)
    Balance, lines per slice, full chain:
          0:   2.70% (54,056)   13:   2.70% (54,054)   26:   2.70% (54,054)
          1:   2.70% (54,054)   14:   2.70% (54,054)   27:   2.70% (54,054)
          2:   2.70% (54,054)   15:   2.70% (54,054)   28:   2.70% (54,054)
          3:   2.70% (54,054)   16:   2.70% (54,054)   29:   2.70% (54,054)
          4:   2.70% (54,054)   17:   2.70% (54,054)   30:   2.70% (54,054)
          5:   2.70% (54,054)   18:   2.70% (54,054)   31:   2.70% (54,054)
          6:   2.70% (54,054)   19:   2.70% (54,054)   32:   2.70% (54,054)
          7:   2.70% (54,054)   20:   2.70% (54,054)   33:   2.70% (54,054)
          8:   2.70% (54,054)   21:   2.70% (54,054)   34:   2.70% (54,054)
          9:   2.70% (54,054)   22:   2.70% (54,054)   35:   2.70% (54,054)
         10:   2.70% (54,054)   23:   2.70% (54,054)   36:   2.70% (54,054)
         11:   2.70% (54,054)   24:   2.70% (54,054)
         12:   2.70% (54,054)   25:   2.70% (54,054)
    Max to average ratio: 1.000
    2,000,000 total lines in chain
\end{verbatim}

Example invocation
\begin{shell}
% ./dsinfo.py test-20
\end{shell}
The argument can be one or more jobids or dataset ids.  If the
argument is a jobid, it is assumed that the dataset name is
\texttt{default}.  If there are more than one dataset in the job, a
list of dataset names will be returned.

Note also that it may be convenient to use the shells feature....
\begin{shell}
% dsinfo -l -q test-{0..99}
\end{shell}
to scan for datasets in the 100 first jobids.  Adding the \texttt{-q}
option will make \texttt{dsinfo} suppress the warning messages for
those jobids that do not contain datasets.





\section{Find Data in Dataset}
The \texttt{dsgrep} command is a parallel grep for datasets that is
used to look at data stored in datasets or dataset chains.
\subsection{Invocation}
\begin{shell}
ax dsgrep --help
\end{shell}

\begin{snugshade}
\begin{verbatim}
usage: dsgrep [options] pattern ds [ds [...]] [column [column [...]]

  positional arguments:
  pattern
  dataset
  columns

  optional arguments:
  -h, --help         show this help message and exit
  -c, --chain        Follow dataset chains
  -i, --ignore-case  Case insensitive pattern
\end{verbatim}
\end{snugshade}

The \texttt{pattern} is a regular expression and \texttt{ds} are datasets.  For example
\begin{shell}
% dsgrep.py Alice test-0 test-1/special name
\end{shell}
Will look for the string \texttt{Alice} in the \texttt{name} column of
the two datasets \texttt{text-0} and \texttt{test-1/special}.
Optional arguments are
\begin{snugshade}
  \begin{tabular}{p{4cm}p{9cm}}
      \texttt{-h}\hspace{3cm}\texttt{--help} & show help message and exit\\[4ex]
      \texttt{-c}\hspace{3cm}\texttt{--chain} & follow dataset chains\\[4ex]
      \texttt{-i}\hspace{3cm}\texttt{--ignore-case} & Case insensitive pattern\\[4ex]
  \end{tabular}
\end{snugshade}
Strings and columns with special characters have to be quoted.



\subsection{Abuse dsgrep to show datasets}
The data in a dataset may be printed to \texttt{stdout} by
\texttt{grep}ing using a regexp that always matches, like this
\begin{shell}
% ./dsgrep.py . test-0 | less
\end{shell}





\section{The Urd Job Database Server}
Urd is run as a stand-alone server.
\subsection{Invocation}
Start Urd by
\begin{shell}
ax urd
\end{shell}
These are the options
\begin{shell}
ax urd --help
\end{shell}
\begin{snugshade}
\begin{verbatim}
usage: urd [-h] [--port PORT] [--path PATH]

optional arguments:
-h, --help   show this help message and exit
--port PORT  server port (default: 8080)
--path PATH  database directory (can be relative to project directory)
(default: ./urd.db)
\end{verbatim}
\end{snugshade}
Read about running and configuring Urd on page~\ref{urd_howto_run}



\subsection{Authorization to Urd}
Authorisation to Urd could be set in the \texttt{URD\_AUTH}
environment variable.  A common way to invoke the run command with Urd
authorisation is like this
\begin{shell}
% URD_AUTH=user:passwd ax run [script]
\end{shell}
Note that the purpose of the authentication is
actually \textsl{identification}.  It is used to get write access to
certain Urd lists.  Nothing more.


