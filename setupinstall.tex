%%%%%%%%%%%%%%%%%%%%%%%%%%%%%%%%%%%%%%%%%%%%%%%%%%%%%%%%%%%%%%%%%%%%%%%%%%%%
%                                                                          %
% Copyright (c) 2019-2021 Anders Berkeman                                  %
%                                                                          %
% Licensed under the Apache License, Version 2.0 (the "License");          %
% you may not use this file except in compliance with the License.         %
% You may obtain a copy of the License at                                  %
%                                                                          %
%  http://www.apache.org/licenses/LICENSE-2.0                              %
%                                                                          %
% Unless required by applicable law or agreed to in writing, software      %
% distributed under the License is distributed on an "AS IS" BASIS,        %
% WITHOUT WARRANTIES OR CONDITIONS OF ANY KIND, either express or implied. %
% See the License for the specific language governing permissions and      %
% limitations under the License.                                           %
%                                                                          %
%%%%%%%%%%%%%%%%%%%%%%%%%%%%%%%%%%%%%%%%%%%%%%%%%%%%%%%%%%%%%%%%%%%%%%%%%%%%

\section{Install the Accelerator}

\subsection{Using the \texttt{pip} command}
The easiest way to install the Accelerator is by fetching it from the
PyPi repository
\begin{shell}
pip install accelerator
\end{shell}
Some prefer to install to a virtual environment and do something in
line with the following
\begin{shell}
python3 -m venv accvenv
source accvenv/bin/activate
pip install accelerator
\end{shell}
This will install the Accelerator to the \texttt{accvenv} virtual
environment.  Now, use for example the following command
\begin{shell}
ax --help
\end{shell}
to check that the installation worked.  The next step is to set up a
project.



\section{Set up a New Project}
In order to run, the Accelerator needs to have these things in place
\begin{itemize}
\item[] at least one \textsl{workdir} to store data in,
\item[] most likely a new \textsl{method package} directory to store new code in, and
\item[] a \textsl{configuration file} to set things up.
\end{itemize}
This is all taken care of by the \texttt{init} command.
\begin{shell}
ax init
\end{shell}
For more information see section~\ref{sec:initialisation}.  A typical
project setup will look like this
\begin{shell}
myproject/
    accelerator.conf
    dev/
        (methods.conf)
        a_method.py
        build.py
\end{shell}
where methods are stored in the \texttt{dev} directory.
\texttt{methods.conf} is only required when not using
\texttt{auto-discover}, see section{sec:configfile}.



\section{Run the Tests}
A rather extensive test suite is included in the Accerator
installation.  To run this, enable the test package in the
configuration file:
\begin{verbatim}
method packages:
    ...
    accelerator.test_methods
\end{verbatim}
start the server using
\begin{python}
ax server
\end{python}
and in another shell start the test
\begin{python}
ax run tests
\end{python}
Since the tests include testing of different character encodings, you may end up with a
\begin{verbatim}
Exception: Failed to enable numeric_comma, please install at least one
of the following locales: da_DK nb_NO nn_NO sv_SE fi_FI en_ZA es_ES
es_MX fr_FR ru_RU de_DE nl_NL it_IT
\end{verbatim}
On a Debian-based machine, locales can be configured using
\begin{shell}
dpkg-reconfigure locales 
\end{shell}
which has to be run with root privileges.



\section{Server Configuration File}
\label{sec:configfile}
The configuration file specifies which method packages and workdirs
that are available for a project.  A template configuration file can
be generated using the \texttt{init} command as described in
section~\ref{sec:initialisation}.  Below is an example of a configuration file.
%\begin{leftbar}
\begin{snugshade}
\begin{verbatim}
# The configuration is a collection of key value pairs.
#
# Values are specified as
# key: value
# or for several values
# key:
#       value 1
#       value 2
#       ...
# (any leading whitespace is ok)
#
# Use ${VAR} or ${VAR=DEFAULT} to use environment variables.

slices: 23
workdirs:
        test /zbd/workdirs/test
        import ${HOME}/workdirs/import
        live wdirs/live

# Target workdir defaults to the first workdir, but you can override it.
# target workdir: dev
# (this is where jobs without a workdir override are built)

method packages:
        dev auto-discover
        accelerator.standard_methods
#       accelerator.test_methods   

urd: http://localhost:9000

result directory: ${HOME}/accelerator/results
input directory: /zbd/data/backblaze

# If you want to run methods on different python interpreters you can
# specify names for other interpreters here, and put that name after
# the method in methods.conf.
# You automatically get four names for the interpreter that started
# the server: DEFAULT, 3, 3.7 and 3.7.3 (adjusted to the actual
# version used). You can override these here, except DEFAULT.
# interpreters:
#       2.7 /path/to/python2.7
#       test /path/to/beta/python  
\end{verbatim}
\end{snugshade}%
%\end{leftbar}%

The configuration file above specifies 23 slices and three workdirs,
called \texttt{test}, \texttt{import}, and \texttt{live}.  The
\texttt{test} workdir is specified using an absolute path, the
\texttt{import} workdir is specified relative to the user's home
directory using the shell environment variable \texttt{\$HOME}, and
the \texttt{live} workdir is specified using a path relative to
the location of the configuration file itself.

The workdir that is specified first is the \textsl{target workdir},
where jobs are written to by default.  All other specified workdirs
will by default only be used for reading.  Any of the workdirs
specified could be written to, though, using the
\texttt{set\_workdir=} option to the \texttt{build} command, as
described on page~\ref{sec:set_workdir}.

Methods packages available for use are the \texttt{standard\_methods} bundled
with the Accelerator, and methods defined in the
directory \texttt{dev} (if defined in \texttt{dev/methods.conf}).

\starttabletwo

\RPtwo \texttt{slices} & Number of slices used for the project.\\[1ex]

\RPtwo \texttt{workdirs} & A list of paths to workdir directories.  At
least one workdir needs to be defined.  All workdirs that are used
together must have the same number of slices.  It is possible to use
shell environment variables such as \texttt{\$\{HOME\}} when
specifying workdirs.  Path starting with a slash (\texttt{/}) are
absolute paths, all other paths are relative to the location of the
configuration file itself.

Unless overridden by the \texttt{target workdir}, the first workdir in
the list will be the default \textsl{target workdir} that is used for
all writing.  Other specified workdirs will only be read from, unless
overrided by the \texttt{build} call as described on
page~\ref{sec:set_workdir}.\\[1ex]

\RPtwo \texttt{target workdir} & Name of the \textsl{target workdir}.
If specified this overrides the first item in the \texttt{workdirs}
list.\\[1ex]

\RPtwo \texttt{method packages} & A list of directories containing
methods.  These will be the only directories where the Accelerator can
``see'' methods.  \texttt{standard\_methods} is bundled with the
Accelerator and is commonly used.

A package specified alone on a row indicates that its corresponding
directory contains a file named \texttt{methods.conf} that speficy
which methods in the package that can be executed.

If the package is specified with the string \texttt{auto-discover},
like in the example above, any method in the package can be executed
by the Accelerator, and no \texttt{methods.conf} file is required.  It
can still be present in order to specify which interpreters to use on
a method level though.\\[1ex]

\RPtwo \texttt{urd} & If present, an URL to the Urd server.\\[1ex]

\RPtwo \texttt{result directory} & A common path that is available to
all jobs.  Use the \texttt{job.link\_result()}-function to create
symbolic links from files in job directories to this directory.  Just
like workdirs, this path is either absolute or relative to the
location of the configuration file.  \\[1ex]

\pagebreak

\RPtwo \texttt{input directory} & Default root path for
\texttt{csvimport}.  This is to avoid rebuilds of imports if input
files are moved to another directory.  (This typically happens when
setting up a similar system on another physical machine.)  See
section~\ref{sec:INPUT_DIR} on how to get access to
\texttt{input\_directory} from any method.  Just like workdirs, this
path is either absolute or relative to the location of the
configuration file.  \\[1ex]

\RPtwo \texttt{interpreters} & Name and path to python executables.
These are used in \texttt{methods.conf} to specify specific Python
versions (or virtual environments) for individual methods.  If
unspecified, methods will be executed using the same binary that runs
the Accelerator's server process.\\[1ex]

\stoptabletwo

It is possible to assign values in the configuration file using shell
environment variables.  In the example above, workdirs are specified
relative to \texttt{\$\{HOME\}}, for example.  In general, the
assignment is \texttt{\$\{VAR=DEFAULT\}}.



\section{Setting up a Standalone Urd-server}
\label{sec:urd_setup}
The main server program will start a local Urd server by default.
This server is for local use only.  If the Urd server should be used
to share information between users, a standalone server needs to be
set up.  This is how to set up a standalone Urd server:

\begin{enumerate}
\item First, create a new directory for the Urd-server.
\item Run \texttt{ax init} in this directory.
\item Copy existing (or create new \texttt{accelerator.conf} to this directory.
\item Modify the \texttt{urd server} line like this
\begin{verbatim}
  urd: localhost:30025
\end{verbatim}
This will setup a server listening on port 30025.  This is only an example.
\item Edit the \texttt{urd.db/passwd file} according to
  section~\ref{sec:urdpasswd} if password write protection is wanted.
\item Start the server by
  \begin{shell}
    ax urd-server
  \end{shell}
  or, if no passwords are required
  \begin{shell}
    ax urd-server --allow-passwordless
  \end{shell}
\item Done!
\end{enumerate}



\subsection{The Urd Database}
The Urd database is stored in the directory \texttt{urd.db} and has
the following structure
\begin{verbatim}
database_root/
    passwd
    database/
        user1/
            list1
            list2
        user2/
            list3
\end{verbatim}
Each list is a transaction log.  Nothing is ever deleted or modified.
New information is appended.



\subsection{The \texttt{passwd} file}
\label{sec:urdpasswd}
The \texttt{passwd} file stores write access authentication.  The file
format is straightforward, each line is a user--password pair as follows
\begin{verbatim}
user:password
\end{verbatim}
For example, if the file contains the following line
\begin{verbatim}
ab:secret
\end{verbatim}
A build script run like this
\begin{shell}
URD_AUTH=ab:secret ax run script
\end{shell}
will have write access to all lists belonging to the user \texttt{ab},
such as for example the \texttt{ab/test} and \texttt{ab/import} lists.
But it can not write to lists belonging to other users, such as
\texttt{cd/import}.  It can always read all lists, though.



\section{Setting up a Standalone Board-server}
The main server program will start a local \texttt{board} server by
default.  This behaviour can be disabled in the Accelerator's
configuration file.  In that case, to start the board server manually,
use the
\begin{python}
ax board
\end{python}
command.
